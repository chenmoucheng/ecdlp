%
%
% This is LLNCS.DEM the demonstration file of
% the LaTeX macro package from Springer-Verlag
% for Lecture Notes in Computer Science,
% version 2.4 for LaTeX2e as of 16. April 2010
%
\documentclass{llncs}
%
\usepackage{algpseudocode,amsfonts,amsmath,amssymb,mathrsfs,multirow,latexsym}
%
\DeclareMathOperator{\ord}{ord}
\DeclareMathOperator{\res}{Res}
\DeclareMathOperator{\sgn}{sgn}
\newcommand{\F}[1]{\ensuremath{\mathbb F_{#1}}}
%\def\F{{\mathbb F}}
\def\Fqn{{\mathbb F}_{p^n}}
\def\Fqm{{\mathbb F}_{p^m}}
\def\qed{\hfill $\Box$}
\long\def\comment#1{}
%
\begin{document}
%
\title{On the computational complexity of ECDLP for elliptic curves in
  various forms}
%
%\titlerunning{}
%
%\author{Kenta~Kodera \and Atsuko~Miyaji \and Chen-Mou~Cheng}
%
%\authorrunning{}
%
%\institute{Osaka University, Japan}

\maketitle              % typeset the title of the contribution

\begin{abstract}
%
  The security of elliptic curve cryptography is closely related to
  the computational complexity of solving the elliptic curve discrete
  logarithm problem (ECDLP).
%
  Today, the best practical attacks against ECDLP are
  exponential-time, generic discrete logarithm algorithms such as
  Pollard's rho method~\cite{1978-pollard-kangaroo}.
%
  Recently, there is a line of research on index calculus algorithms
  for ECDLP started by Semaev, Gaudry, and
  Diem~\cite{DBLP:journals/iacr/Semaev04,DBLP:journals/jsc/Gaudry09,DBLP:journals/moc/Diem11}.
%
  Under certain heuristic assumptions, such algorithms could lead to
  subexponential attacks to ECDLP in some
  cases~\cite{DBLP:conf/eurocrypt/FaugerePPR12,DBLP:journals/iacr/PetitQ12,DBLP:conf/iwsec/HuangPST13}.
%
  In this paper, we consider the complexity of solving ECDLP for
  elliptic curves in various forms---including
  Hessian~\cite{DBLP:conf/ches/Smart01},
  Montgomery~\cite{1987-montgomery}, (twisted)
  Edwards~\cite{DBLP:journals/iacr/BernsteinL07,DBLP:journals/iacr/BernsteinBJLP08},
  and Weierstrass using index calculus algorithms.
%
  The research question we would like to answer is: Using index
  calculus algorithm, is there any significant difference in the
  computational complexity of solving ECDLP for elliptic curves in
  various forms?
%
  We will provide some insights and empirical evidence showing an
  affirmative answer in this paper.
%
  \keywords{ECDLP, index calculus, elliptic curves cryptography,
    security evaluation}
\end{abstract}

%
% Introduction
%

\section{Introduction}
%
In recent years, elliptic curve cryptography is gaining momentum in
deployment, as it can achieve the same level of security as RSA using
much shorter keys and ciphertexts.
%
The security of elliptic curve cryptography is closely related to the
complexity of solving the elliptic curve discrete logarithm problem
(ECDLP).
%
Let $p$ be a prime number, and $E$, a nonsingular elliptic curve over
\F{p^n}, the finite field of $p^n$ elements for some positive integer
$n$.
%
That is, $E$ is a plane algebraic curve defined by the equation
$y^2=x^3+ax+b$ for $a,b\in\F{p^n}$ and $\Delta=-16(4a^3+27b^2)\neq 0$.
%
Along with a point at infinity $\mathcal O$, the set of rational
points $E(\F{p^n})$ forms an abelian group with $\mathcal O$ being the
identity.
%
Given $P\in E(\F{p^n})$ and $Q\in\langle P\rangle$, ECDLP is the
problem of finding an integer $\alpha$ such that $Q=\alpha P$.

Today, the best practical attacks against ECDLP are exponential-time,
generic discrete logarithm algorithms such as Pollard's rho
method~\cite{1978-pollard-kangaroo}.
%
However, recently there is a line of research on index calculus
algorithms for ECDLP started by Semaev, Gaudry, and
Diem~\cite{DBLP:journals/iacr/Semaev04,DBLP:journals/jsc/Gaudry09,DBLP:journals/moc/Diem11}.
%
Under certain heuristic assumptions, such algorithms could lead to
subexponential attacks to ECDLP in some
cases~\cite{DBLP:conf/eurocrypt/FaugerePPR12,DBLP:journals/iacr/PetitQ12,DBLP:conf/iwsec/HuangPST13}.
%
The interested reader is referred to a survey paper by Galbraith and
Gaudry for a more comprehensive and in-depth account of the recent
development of ECDLP algorithms along various
directions~\cite{DBLP:journals/dcc/GalbraithG16}.

In this paper, we consider the complexity of solving ECDLP for
elliptic curves in various forms---including
Hessian~\cite{DBLP:conf/ches/Smart01},
Montgomery~\cite{1987-montgomery}, twisted
Edwards~\cite{DBLP:journals/iacr/BernsteinBJLP08}, and
Weierstrass---over optimal extension fields
(OEFs)~\cite{DBLP:conf/crypto/BaileyP98} using index calculus
algorithms.
%
Recently, elliptic curves of various forms such as the (Montgomery)
Curve25519~\cite{DBLP:conf/pkc/Bernstein06} have been drawing a lot of
attention in deployment, partly because some of them allow fast
implementation that is secure against timing-based side-channel
attacks.
%
Furthermore, we can construct these curves not only over prime fields
such as \F{2^{255} - 19} as used in Curve25519 but also extension
fields.
%
An OEF is an extension field from a prime field \F p with $p$ close to
$2^8, 2^{16}, 2^{32}, 2^{64}, \ldots$
%
Such primes fit nicely into the processor words of 8, 16, 32, or
64-bit microprocessors and hence are particularly suitable for
software implementation, allowing for efficient utilization of fast
integer arithmetics on modern
microprocessors~\cite{DBLP:conf/crypto/BaileyP98}.
%
As we will see, our experimental results show quite significant
difference in the computational complexity of solving ECDLP for
elliptic curves in various forms over OEF.

The rest of this paper is organized as follows.
%
In Section~\ref{sec:index-calculus-ecdlp}, we will give an high-level
overview of the index calculus algorithm for attacking ECDLP.
%
In Section~\ref{sec:montgomery-symmetry}, we will describe curves of
various forms and how we exploit the symmetry for speeding up index
calculus on them.

%
% Index calculus for ECDLP
%

\section{Index calculus for ECDLP}
%
\label{sec:index-calculus-ecdlp}
%
Let $E$ be an elliptic curve defined over a finite field \F{p^n}.
%
For cryptographic applications, we are mostly interested in a
prime-order subgroup generated by a rational point $P\in E(\F{p^n})$.
%
To find an integer $\alpha$ such that $Q=\alpha P$ for
$Q\in\langle P\rangle$ using an index calculus algorithm, one
typically works as follows.
%
\begin{enumerate}
%
\item Determine a \emph{factor base} $\mathcal F\subset E(\F{p^n})$.
%
\item Collect a set $\mathcal R$ of \emph{relations} by decomposing
  random points $a_iP+b_iQ$ into a sum of points from $\mathcal F$,
  i.e.,
  \[ \mathcal
    R=\left\{a_iP+b_iQ=\sum_{j=1}^mP_{i,j}:P_{i,j}\in\mathcal
      F\right\} \]
%
\item When $|\mathcal R|\approx|\mathcal F|$, eliminate the righthand
  side using linear algebra to obtain an equation in the form
  $aP+bQ=\mathcal O$, and $\alpha=-a/b\bmod\ord(P)$.
%
\end{enumerate} 

\subsection{Semaev's summation polynomials}
%
\label{sec:summation-polynomials}
%
As we can see, an important step in index calculus algorithms for
solving ECDLP is point decomposition on an elliptic curve.
%
It is straightforward that if two points sum to zero on a Weierstrass
curve, then their $x$-coordinates must be equal.
%
Let us now consider the simplest nontrivial case where three points on
a Weierstrass curve $y^2=x^3+ax+b$ sum to the point at infinity.
%
Let
\[ Z=\left\{\begin{aligned}
      (x_1,y_1,x_2,y_2,x_3,y_3)&\in\F{p^n}^6:(x_i,y_i)\in E(\F{p^n}),i=1,2,3; \\
      & (x_1,y_1)+(x_2,y_2)+(x_3,y_3)=\mathcal O
    \end{aligned} \right\}. \]
%
Clearly, $Z$ is in the variety of the ideal
$I\subset\F{p^n}[X_1,Y_1,X_2,Y_2,X_3,Y_3]$ generated by
\[ \left\{\begin{aligned}
      &  (X_3 - X_1)(Y_2 - Y_1) - (X_2 - X_1)(Y_3 - Y_1),\\
      & Y_i^2 - (X_i^3 + aX_i + b),i=1,2,3
    \end{aligned}\right\}. \]
%
Now let $J=I\cap\F{p^n}[X_1,X_2,X_3]$.
%
Using MAGMA's \texttt{EliminationIdeal} function, we obtain that $J$
is actually a principal ideal generated by the polynomial
$(X_2 - X_3)(X_1 - X_3)(X_1 - X_2)f_3$, where
%
\begin{align*}
  f_3 = & X_1^2X_2^2 - 2X_1^2X_2X_3 + X_1^2X_3^2 - 2X_1X_2^2X_3 - 2X_1X_2X_3^2 - 2aX_1X_2 - 2aX_1X_3 \\
        & - 4bX_1 + X_2^2X_3^2 - 2aX_2X_3 - 4bX_2 - 4bX_3 + a^2.
\end{align*}
%
Clearly, the linear factors of the generator correspond to the
degenerated case where two or more points are the same or of opposite
signs, and $f_3$ is the 3rd \emph{summation polynomial}, that is, the
summation polynomial for three distinct points summing to zero.

Starting from the 2nd and 3rd summation polynomials, one can
recursively obtain the subsequent summation polynomials via taking
resultants.
%
This is the observation Semaev made in his seminal
work~\cite{DBLP:journals/iacr/Semaev04}.
%
In short, his proposal is to consider factor bases of the following
form:
\[ \mathcal F=\Big\{(x,y)\in E(\F{p^n}):x\in
  V\subset\F{p^n})\Big\}, \] where $V$ is a subset of \F{p^n}.
%
Note that this factor base is invariant under negation.
%
That is, $P_i\in\mathcal F$ implies $-P_i\in\mathcal F$.
%
In this case, we will have about $|\mathcal F|/2$ (trivial) relations
$P_i+(-P_i)=\mathcal O$ for free, so we just need to find about
$|\mathcal F|/2$ nontrivial relations.
%
In general, we will only discuss factor bases that are invariant under
negation, so by abuse of language, both $\mathcal F$ and $\mathcal F$
modulo negation may be referred to as a factor base in the rest of
this paper.

\subsection{Weil restriction}
%
Restricting $x$-coordinates of the points in factor base to a subset
of \F{p^n} is important from a viewpoint of polynomial system solving.
%
Take $f_3$ as an example.
%
When decomposing a random point $aP+bQ$, we first substitute its
$x$-coordinate into say $X_3$, projecting $J$ onto $\F{p^n}[X_1,X_2]$.
%
The dimension of the variety of this ideal is nonzero.
%
Therefore, we would like to pose some restrictions on $X_1$ and $X_2$
to reduce the dimension to zero and make the solving time more
manageable.

When looking for solutions to a polynomial in $\F{p^n}[X]$ in \F{p^n},
we can view it as a commutative affine algebra
$\mathcal A=\F{p^n}[X]/(X^{p^n} -
X)\cong\F{p^n}[X_1,\ldots,X_n]/(X_1^p - X_1,\ldots,X_n^p - X_n)$.
%
This can be done by identifying the indeterminate $X$ as
$X_1\theta_1+\cdots+X_n\theta_n$, where $(\theta_1,\ldots,\theta_n)$
is a basis for \F{p^n} over \F p.
%
Hence, a polynomial $f=\sum a_iX^i\in\F{p^n}[X]$ can be identified as
a polynomial $f_1\theta_1+\cdots+f_n\theta_n$, where
$f_1,\ldots,f_n\in\mathcal A'=\F p[X_1,\ldots,X_n]/(X_1^p -
X_1,\ldots,X_n^p - X_n)$, by appropriately sending any coefficient
$a_i\in\F{p^n}$ to $a_i^{(1)}\theta_1+\cdots+a_i^{(n)}\theta_n$ for
$a_i^{(1)},\ldots,a_i^{(n)}\in\F p$.
%
Therefore, an equation $f=0$ over \F{p^n} will give rise to a system
of equations $f_1=\cdots=f_n=0$ over \F p.
%
This technique is known as the \emph{Weil restriction} and is used in
the Gaudry-Diem attack, in which the factor base is chosen to consist
of points whose $x$-coordinates lie in a subspace $V$ of \F{p^n} over
\F p~\cite{DBLP:journals/jsc/Gaudry09,DBLP:journals/moc/Diem11}.

\subsection{Exploiting symmetry}
%
\label{sec:exploit-symmetry}
%
Naturally, the symmetric group $S_m$ acts on a point decomposition
$P_1+\ldots+P_m$ because elliptic curve groups are abelian.
%
As noted by Gaudry in his seminal
work~\cite{DBLP:journals/jsc/Gaudry09}, we can therefore rewrite the
variables $x_1,\ldots,x_m\in\F{p^n}$ by elementary symmetric
polynomials $e_1,\ldots,e_m$, where $e_1=\sum x_i$,
$e_2=\sum_{i\neq j}x_ix_j$,
$e_3=\sum_{i\neq j,i\neq k,j\neq k}x_ix_jx_k$, etc.
%
Such rewriting can reduce the degree of summation polynomials and
significantly speed up point
decomposition~\cite{DBLP:conf/eurocrypt/FaugerePPR12,DBLP:conf/iwsec/HuangPST13}.

One might be able to exploit additional symmetry using actions of
other groups, e.g., when the factor base is invariant under addition
of small torsion points~\cite{DBLP:conf/eurocrypt/FaugereHJRV14}.
%
For example, consider a relation $R$ under the action of addition of a
2-torsion point $T_2$:
\begin{align*}
  R & = P_1+\cdots+P_n \\
    & =
      (P_1+u_1T_2)+\cdots+(P_{n-1}+u_{n-1}T_2)+\left(P_n+\left(\sum_{i=1}^{n-1}u_i\right)T_2\right).
\end{align*}
%
Clearly this holds for any $u_1,\ldots,u_{n-1}\in\{0,1\}$, and
$P_i\in\mathcal F$ implies that $P_i+u_iT_2\in\mathcal F$ when the
factor base $\mathcal F$ is invariant under addition of $T_2$.

Of course such kind of speed-up is curve-specific in nature.
%
Furthermore, even if the factor base is invariant under addition of
small torsion points, we may or may not be able to exploit such
symmetry to speed up point decomposition depending on whether the
action of addition of these small torsion points is ``easy to handle
in the polynomial system solving
process''~\cite{DBLP:conf/eurocrypt/FaugereHJRV14}.

Finally, this potential speed-up will be counteracted by
\emph{decrease in decomposition probability}, which seems to have been
largely ignored in the literature.
%
For example, Faug\`ere, Gaudry, Hout, and Renault \emph{did not} take
it into account in their estimation of the cost of a complete ECDLP
attack on twisted Edwards or twisted Jacobi intersections
curves~\cite{DBLP:conf/eurocrypt/FaugereHJRV14}.
%
Specifically in the Section 5.3 of their paper, they explicitly stated
that ``[the] probability to decompose a point [into a sum of $n$
points from the factor base] is $\frac{1}{n!}$'' for curves in various
forms, regardless of whether the factor base is invariant under
addition of small torsion points.
%
However, this decomposition probability will decrease due to the
symmetry brought by additional group action on a factor base
$\mathcal F$, similar to the fact that the probability decreases by a
factor of $n!$ due to the symmetry of the symmetric group $S_n$ acting
on $\mathcal F$.
%
For example, the decomposition probability should further decrease by
a factor of $2^{n-1}$ in the earlier example of this section.

%
% Montgomery curves
%

%------------------------------
\section{Montgomery curves}
%------------------------------------
\label{sec:montgomery-symmetry}
%
A Montgomery curve $M_{A,B}$ over \F{p^n} for $p\neq 2$ is defined by
the equation \begin{equation}
  By^2=x^3+Ax^2+x \label{eq:montgomery-curve} \end{equation} for
$A,B\in\F{p^n}$ such that $A\neq\pm 2$, $B\neq 0$, and
$B(A^2-4)\neq 0$~\cite{1987-montgomery}.
%
For $P=(x,y)\in M_{A,B}$, $-P=(x,-y)$.
%
Furthermore, the addition and doubling formulae for
$(x_3,y_3)=(x_1,y_1)+(x_2,y_2)$ are given as follows.
%
\begin{itemize}
\item When $x_1\neq x_2$:
  \begin{align*}
    x_3 & = B\left(\frac{y_2 - y_1} {x_2 - x_1}\right)^2 - A - x_1 - x_2 = \frac{B(x_2y_1 - x_1y_2)^2} {x_1x_2(x_2 - x_1)^2} \\
    y_3 & = \frac{(2x_1 + x_2 + A)(y_2 - y_1)} {x_2 - x_1} - \frac{B(y_2 - y_1)^3} {(x_2 - x_1)^3} - y_1
  \end{align*}
\item When $x_1=x_2$:
  \begin{align*}
    x_3 & = \frac{(x_1^2 - 1)^2} {4x_1(x_1^2 + Ax_1 + 1)}  \\
    y_3 & = \frac{(2x_1 + x_1 + A)(3x_1^2 + 2Ax_1 + 1)} {2By_1} - \frac{B(3x_1^2 + 2Ax_1 + 1)^3} {(2By_1)^3} - y_1
  \end{align*}
\end{itemize}
%
It was noted by Montgomery himself in his original paper that such
curves can give rise to efficient scalar multiplication
algorithms~\cite{1987-montgomery}.
%
Consider a random point $P\in M_{A,B}(\F{p^n})$ and let
$nP=(X_n:Y_n:Z_n)$ in projective coordinate for any integer $n$.
%
Then:
%
\begin{align*}
  X_{m+n} & = Z_{m-n}[(X_m - Z_m)(X_n + Z_n) + (X_m + Z_m)(X_n - Z_n)]^2 \\
  Z_{m+n} & = X_{m-n}[(X_m - Z_m)(X_n + Z_n) - (X_m + Z_m)(X_n - Z_n)]^2
\end{align*}
%
In particular, when $m=n$:
\begin{align*}
  X_{2n} & = (X_n + Z_n)^2(X_n - Z_n)^2 \\
  Z_{2n} & = (4X_nZ_n)\left((X_n - Z_n)^2 + ((A+2)/4)(4X_nZ_n)\right) \\
  4X_nZ_n & = ( X_n + Z_n)^2 - (X_n - Z_n)^2
\end{align*}
%
In this way, scalar multiplication on Montgomery curve can be
performed without using $y$-coordinates, leading to fast
implementation.

\subsection{Summation polynomials for Montgomery curves}

Following Semaev's approach~\cite{DBLP:journals/iacr/Semaev04}, we can
construct summation polynomials for Montgomery curves as follows.
%
Like Weierstrass curves, the 2nd summation polynomial for Montgomery
curves is simply $f_{M,2} = X_1 - X_2$.
%
Now consider $P,Q\in M_{A, B}$ where $P=(x_1, y_1)$, $Q=(x_2, y_2)$.
%
Let $P+Q=(x_3, y_3)$ and $P-Q=(x_4, y_4)$.
%
By addition formula, we have
\[ x_3 = \frac{B(x_2y_1 - x_1y_2)^2} {x_1x_2(x_2 - x_1)^2},
  x_4 =\frac{B(x_2y_1 - x_1y_2)^2} {x_1x_2(x_2 + x_1)^2}. \]
%
It follows that
%
\begin{align*}
  x_3 + x_4&=\frac{2\left((x_1 + x_2)(x_1x_2 + 1) + 2Ax_1x_2\right)}{(x_1 - x_2)^2},\text{ and} \\
  x_3x_4&=\frac{(1 - x_1x_2)^2}{(x_1 - x_2)^2}.
\end{align*}
%
Using the relationship between the roots of a quadratic polynomial and
its coefficients, we obtain
\[ (x_1 - x_2)^2x^2 - 2\left((x_1 + x_2)(x_1x_2 + 1) +
    2Ax_1x_2\right)x + (1 - x_1x_2)^2. \]
%
From here, we can obtain for Montgomery curve the 3rd summation
polynomial:
\[ f_{M,3}(X_1,X_2,X_3) = (X_1 - X_2)^2X_3^2 - 2\left((X_1 +
    X_2)(X_1X_2 + 1) + 2AX_1X_2\right)X_3 + (1-X_1X_2)^2, \]
%
as well as the subsequent summation polynomials via taking resultants:
\[ f_{M,m}(X_1,\ldots,X_m) =
  \res_X\left(f_{m-k}(X_1,\ldots,X_{m-k-1},X),f_{k+2}(X_{m-k},\ldots,X_m,X)\right). \]
%--------------------------------
\subsection{Action of 2-torsion points for Montgomery curves} \label{subsec:TSPL}
%------------------------
A Montgomery curve always contains an affine 2-torsion point $T_2$.
%
Since $T_2+T_2=2T_2=\mathcal O$, it follows that $-T_2=T_2$.
%
If we write $T_2=(x,y)$, then we can see that $y=0$ in order for
$-T_2=T_2$, as $p\neq 2$.
%
Substituting $y=0$ into Equation~(\ref{eq:montgomery-curve}),
we get an equation $x^3+Ax^2+x=0$.
%
The lefthand side factors into $x(x^2+Ax+1)=0$, so we get \[
  x=0,\frac{-A\pm\sqrt{A^2 - 4}}{2}. \]
%
Therefore, the set of rational points on a Montgomery curve includes
at least two 2-torsion points, namely, $\mathcal O$ and $(0,0)$.
%
The other 2-torsion points may or may not be rational, so we will
first focus on the 2-torsion point $(0,0)$.
%
Substituting $(x_2,y_2)=(0,0)$ into the addition formula for
Montgomery curves, we get that for any point $P=(x,y)\in M_{A,B}$,
$P+(0,0)=(1/x,y')$ for some $y'$.

Following the approach outlined by Faug\`ere, Gaudry, Hout, and
Renault~\cite{DBLP:conf/eurocrypt/FaugereHJRV14}, we consider a
relation $R:=P_1+\cdots+P_m$ under the action of addition of 2-torsion
points $T_2=(0,0)$:
\[ R' :=
  (P_1+u_1T_2)+(P_2+u_2T_2)+\cdots+(P_{m-1}+u_{m-1}T_2)+\left(P_m+\left(\sum_{i=1}^{m-1}u_i\right)T_2\right). \]
%
Clearly if $R$ holds, then $R'$ also holds for any
$u_1,\ldots,u_{m-1}\in\{0,1\}$, and
$D_m=(\mathbb{Z}/2\mathbb{Z})^{m-1}\rtimes S_m$ acts on the summation
polynomial $f_m$.
%
However, this assumes that the factor base
$\mathcal F=\{(x,y)\in E(\F{p^n}):x\in V\subset\F{p^n}\}$ is invariant
under addition of the 2-torsion point, which is indeed the case for
binary Edwards curves but not necessarily true for Montgomery curves
unless $V$ is closed under taking multiplicative inverses.
%
In other words, this means that for Montgomery curves, $V$ needs to be
a \emph{subfield} of $\F{p^n}$, i.e., $V=\F{p^\ell}$ for some integer
$\ell$ that divides $n$.
%
In this case, $f_m$ is invariant under the action of
$x_i\mapsto 1/x_i$.
%
Unfortunately, such an action is not linear and hence not easy to
handle in polynomial system solving process.


%
% Hessian curves
%

\section{Hessian and other curves}
\label{sec:hessian}

\subsection{Hessian curves}
%
A Hessian curve $H_d$ over \F{p^n} for $p^n=2\bmod 3$ is defined by
the equation \begin{equation}
  x^3+y^3+1=3dxy \label{eq:hessian-curve} \end{equation} for
$d\in\F{p^n}$ such that $27d^3\neq 1$~\cite{DBLP:conf/ches/Smart01}.
%
For $P=(x,y)\in H_d$, $-P=(y,x)$.
%
Furthermore, the addition and doubling formulae for
$(x_3,y_3)=(x_1,y_1)+(x_2,y_2)$ are given as follows.
%
\begin{itemize}
\item When $(x_1,y_1)\neq(x_2,y_2)$:
  \begin{align*}
    x_3 & = \frac{y_1^2x_2 - y_2^2x_1}{x_2y_2 - x_1y_1} \\
    y_3 & = \frac{x_1^2y_2 - x_2^2y_1}{x_2y_2 - x_1y_1}
  \end{align*}
\item When $(x_1,y_1)=(x_2,y_2)$:
  \begin{align*}
    x_3 & = \frac{y_1(1 - x_1^3)}{x_1^3 - y_1^3} \\
    y_3 & = \frac{x_1(y_1^3 - 1)}{x_1^3 - y_1^3}
  \end{align*}
\end{itemize}

\subsection{Summation polynomials for Hessian curves}

Following a similar approach outlined by Galbraith and
Gebregiyorgis~\cite{DBLP:conf/indocrypt/GalbraithG14}, we can
construct summation polynomials for Hessian curves as follows.
%
First, we introduce a new variable $T=X+Y$, which is invariant under
negation of a point.
%
The 2nd summation polynomial for Hessian curves is simply
$f_{H,2} = T_1 - T_2$.
%
Now let
\[ Z=\left\{\begin{aligned}
      (x_1,y_1,t_1,&x_2,y_2,t_2,x_3,y_3,t_3)\in\F{p^n}^9:(x_i,y_i)\in H_d(\F{p^n}),i=1,2,3; \\
      & (x_1,y_1)+(x_2,y_2)+(x_3,y_3)=\mathcal O; x_i+y_i=t_i,i=1,2,3
    \end{aligned} \right\}. \]
%
Clearly, $Z$ is in the variety of the ideal
$I\subset\F{p^n}[X_1,Y_1,T_1,X_2,Y_2,T_2,X_3,Y_3,T_3]$ generated by
\[ \left\{\begin{aligned}
      &  (X_3 - X_1)(Y_2 - Y_1) - (X_2 - X_1)(Y_3 - Y_1), \\
      & X_i^3 + Y_i^3 + 1 - 3dX_iY_i,i=1,2,3, \\
      & X_i + Y_i - T_i,i=1,2,3
    \end{aligned}\right\}. \]
%
After removing the degenerate factors, we can obtain for Hessian curve
the 3rd summation polynomial:
\begin{align*}
  f_{H,3}(T_1,T_2,T_3) = & T_1^2T_2^2T_3 + dT_1^2T_2^2 + T_1^2T_2T_3^2
                           + dT_1^2T_2T_3 + dT_1^2T_3^2 - T_1^2 + \\
                         & T_1T_2^2T_3^2 + dT_1T_2^2T_3 + dT_1T_2T_3^2
                           + 3d^2T_1T_2T_3 + 2T_1T_2 + 2T_1T_3 + 2dT_1
                           + \\
                         & dT_2^2T_3^2 - T_2^2 + 2T_2T_3 + 2dT_2 - T_3^2 + 2dT_3 + 3d^2,
\end{align*}
%
as well as the subsequent summation polynomials via taking resultants:
\[ f_{H,m}(T_1,\ldots,T_m) =
  \res_T\left(f_{H,m-k}(T_1,\ldots,T_{m-k-1},T),f_{H,k+2}(T_{m-k},\ldots,T_m,T)\right). \]

A Hessian curve can contain an affine 2-torsion point $T_2$.
%
Since $T_2+T_2=2T_2=\mathcal O$, it follows that $-T_2=T_2$.
%
If we write $T_2=(x,y)$, then we can see that $x=y$ in order for
$-T_2=T_2$, as $-T_2=(y,x)$.
%
Substituting $x=y$ into Equation~(\ref{eq:hessian-curve}), we get an
equation $2x^3-3dx^2+1=0$.
%
Therefore, a Hessian curve $H_d(\F{p^n})$ has a 2-torsion point
$(\zeta,\zeta)$ if the polynomial $2X^3 - 3dX^2 + 1$ has a root
$\zeta$ in $\F{p^n}$.
%
In this case, the addition of this 2-torsion point to a point $(x,y)$
would give another point $(x',y')$ where
\[ \left\{\begin{aligned}
x' = & \frac{\zeta y^2 - \zeta^2x}{\zeta^2 - xy}, \\
y' = & \frac{\zeta x^2 - \zeta^2y}{\zeta^2 - xy}.
\end{aligned}\right. \]
%
Obviously, the factor base is not invariant under the action of this
2-torsion point in general.
%
Therefore, it is not clearly how to exploit such symmetry for Hessian
curves.

\subsection{(Twisted) Edwards and Weierstrass curves}

Faug\`ere, Gaudry, Hout, and Renault studied the point decomposition
problem on twisted Edwards, twisted Jacobi intersections, and
Weierstrass curves~\cite{DBLP:conf/eurocrypt/FaugereHJRV14}.
%
For the sake of completeness, we include some basic facts about
(twisted) Edwards curves here.
%
%
An Edwards curve over \F{p^n} for $p\neq 2$ is defined by the
equation \begin{equation*}
  x^2+y^2=1+dx^2y^2 \label{eq:edwards-curve} \end{equation*} for some
$d\in\F{p^n}$~\cite{DBLP:journals/iacr/BernsteinL07}.
%
A twisted Edwards curve $tE_{a',d'}$ over \F{p^n} for $p\neq 2$ is
defined by the equation \begin{equation}
  a'x^2+y^2=1+d'x^2y^2 \label{eq:twisted-edwards-curve} \end{equation}
for some $a',d'\in\F{p^n}$~\cite{DBLP:journals/iacr/BernsteinBJLP08}.
%
A twisted Edwards curve is a quadratic twist of an Edwards curve by
$a_0=1/(a'-d')$.
%
For $P=(x,y)\in H_d$, $-P=(-x,y)$.
%
Furthermore, the addition and doubling formulae for
$(x_3,y_3)=(x_1,y_1)+(x_2,y_2)$ are given as follows.
%
\begin{itemize}
\item When $(x_1,y_1)\neq(x_2,y_2)$:
  \begin{align*}
    x_3 & = \frac{x_1y_2 + y_1x_2}{1 + d'x_1x_2y_1y_2} \\
    y_3 & = \frac{y_1y_2 - a'x_1x_2}{1 - d'x_1x_2y_1y_2}
  \end{align*}
\item When $(x_1,y_1)=(x_2,y_2)$:
  \begin{align*}
    x_3 & = \frac{2x_1y_1}{1 + d'x_1^2y_1^2} \\
    y_3 & = \frac{y_1^2 - a'x_1^2}{1 - d'x_1^2y_1^2}
  \end{align*}
\end{itemize}
 
%
% (twisted) Edwards curve
%

%------------------------------
\subsection{(Twisted) Edwards curves}
%------------------------------------
\label{sec:twisted-edwards}


Faug\`ere, Gaudry, Hout, and Renault studied PDP on twisted Edwards,
twisted Jacobi intersections, and Weierstrass
curves~\cite{DBLP:journals/joc/FaugereGHR14}.
%
For the sake of completeness, we include some basic facts about
(twisted) Edwards curves here.
%
%
An Edwards curve over \F{p^n} for $p\neq 2$ is defined by the
equation \begin{equation*}
  x^2+y^2=1+dx^2y^2 \label{eq:edwards-curve} \end{equation*} for
$d\in\F{p^n}$~\cite{DBLP:journals/iacr/BernsteinL07}.
%
A twisted Edwards curve $tE_{a',d'}$ over \F{p^n} for $p\neq 2$ is
defined by the equation \begin{equation}
  a'x^2+y^2=1+d'x^2y^2 \label{eq:twisted-edwards-curve} \end{equation}
for $a',d'\in\F{p^n}$~\cite{DBLP:journals/iacr/BernsteinBJLP08}.
%
A twisted Edwards curve is a quadratic twist of an Edwards curve by
$a_0=1/(a'-d')$.
%
For $P=(x,y)\in tE_{a',d'}$, $-P=(-x,y)$.
%
Furthermore, the addition and doubling formulae for
$(x_3,y_3)=(x_1,y_1)+(x_2,y_2)$ are given as follows.
%
\begin{itemize}
\item When $(x_1,y_1)\neq(x_2,y_2)$:
  \begin{align*}
    x_3 & = \frac{x_1y_2 + y_1x_2}{1 + d'x_1x_2y_1y_2} \\
    y_3 & = \frac{y_1y_2 - a'x_1x_2}{1 - d'x_1x_2y_1y_2}
  \end{align*}
\item When $(x_1,y_1)=(x_2,y_2)$:
  \begin{align*}
    x_3 & = \frac{2x_1y_1}{1 + d'x_1^2y_1^2} \\
    y_3 & = \frac{y_1^2 - a'x_1^2}{1 - d'x_1^2y_1^2}
  \end{align*}
\end{itemize}
%
As given by Faug\`ere, Gaudry, Hout, and
Renault~\cite{DBLP:journals/joc/FaugereGHR14}, the 3rd summation
polynomial for twisted Edwards curve is
%
\begin{align*}
  f_{tE, 3}(Y_1,Y_2,Y_3) = & \left(Y_1^2Y_2^2 - Y_1^2 - Y_2^2 + \frac{a}{d}\right)Y_3^2  + 2\frac{d-a}{d}Y_1Y_2Y_3 +
                             \frac{a}{d}\left(Y_1^2 + Y_2^2 - 1\right)
                             - Y_1^2Y_2^2.
\end{align*}
%
As usual, subsequent summation polynomials can be obtained by using
resultants.



\section{What summation polynomials are easier to solve?}
\label{sec:twisted-edwards-summation-polynomial}

In this section, we present the main result of this paper, namely,
some insights into what kind of summation polynomials are easier to
solve.
%
For example, the summation polynomials for (twisted) Edwards form of
elliptic curves seem easier to solve compared with those for
Weierstrass or Montgomery forms.
%
The explanation offered by Faug\`ere, Gaudry, Hout, and Renault is
``due to the smaller degree appearing in the computation of Gr\"obner
basis of $\mathscr S_{D_n}$ in comparison with the Weierstrass case,''
cf.~Section~4.1.1 of their
paper~\cite{DBLP:journals/joc/FaugereGHR14}.
%
Unfortunately, as will be detailed in Section~\ref{sec:experiment},
this \emph{cannot} explain the difference in solving time between
(twisted) Edwards and Montgomery forms, as the highest degrees
appearing in the computation of Gr\"obner bases are \emph{the same}
for these two forms.

We offer a simpler explanation to this difference by counting the
number of terms in the summation polynomials for curves in different
forms.
%
We will show that the summation polynomials for (twisted) Edwards form
of elliptic curves \emph{mainly} have terms of \emph{even} powers.
%
The set of terms of even powers are closed under multiplication, so
intuitively polynomials mainly consisting of even-power terms also
``generate'' such kind of polynomials after taking resultants.
%
We believe that this kind of summation polynomials are easier to
solve, which is main reason for the efficiency gain observed in the
case of (twisted) Edwards curves.

We shall make such intuition precise in Theorem~\ref{th:main}, but
before we state the main result, we need to define some terminology
for ease of exposition.
%
When a multivariate polynomial is regarded as a univariate polynomial
in one of its variables $T$, we say that the coefficient $a_i$ of a
term $a_iT^i$ is \emph{of even power} or simply an \emph{even-power
  coefficient} if $i$ is even; that it is \emph{of odd power} or
simply an \emph{odd-power coefficient} otherwise.
%
Note that these coefficients are themselves multivariate polynomials
in one fewer variables.
%
Also, we take 0 as an even number.

Similarly, we say that a monomial $m=\prod_i^n x_i^{e_i},e_i\geq 0$ in
a multivariate polynomial in $n$ variables is \emph{of even power} or
simply an \emph{even-power monomial} if $\sum_i e_i$ is even; that it
is \emph{of odd power} or simply an \emph{odd-power monomial}
otherwise.
%
In contrast, a monomial is \emph{of homogeneous even parity} if all
$e_i$ are even; it is \emph{of homogeneous odd parity} if all $e_i$
are odd.
%
A monomial is \emph{of homogeneous parity} if it is either of
homogeneous even parity or of homogeneous odd parity.
%
Note that the definition of monomials of homogeneous odd parity
depends on the number of variables in the polynomial, which is not the
case for monomials of homogeneous even parity, as 0 is even in our
definition.
%
For example, the monomial $x_1x_2$ is a monomial of homogeneous odd
parity in a polynomial in $x_1$ and $x_2$ but not so in another
polynomial in $x_1,\ldots,x_n$ for $n>2$.
%
\begin{theorem}
  % 
  \label{th:main}
  % 
  Let $\mathcal E$ be a family of elliptic curves such that its 3rd
  summation polynomial $f_{3,\mathcal E}(X_1,X_2,X_3)$ is of degree 2
  in each variable $X_i$ and consists only of monomials of homogeneous
  parity.
  % 
  Let $g_{m,\mathcal E}$ be the polynomial corresponding to the PDP of
  $m$-th order for $\mathcal E$ as described in
  Section~\ref{sec:summation-polynomial}.
  % 
  That is,
  $g_{m,\mathcal E}(X_1,\ldots,X_m)=f_{m+1,\mathcal
    E}(X_1,\ldots,X_m,x)$, where $x$ is a constant depending on the
  point to be decomposed.
  % 
  \begin{enumerate}
    % 
  \item If $m$ is even, then $g_{m,\mathcal E}$ has no monomials of
    odd power.
    % 
  \item If $m$ is odd, then $g_{m,\mathcal E}$ has some but not all
    monomials of odd power.
    % 
  \end{enumerate}
  % 
\end{theorem}
%
Among the four forms of elliptic curves presented in the last two
sections, only the (twisted) Edwards form satisfies the premises of
Theorem~\ref{th:main}.
%
As we will see in Section~\ref{sec:experiment}, the solving time for
the summation polynomials of (twisted) Edwards form is thus
significantly faster than that of the other forms.

We will prove Theorem~\ref{th:main} in the rest of this section, for
which we will need the following lemmas.
%
\begin{lemma}
  % 
  \label{th:resultant}
  % 
  Let $f_1(T_1,\ldots,T_r,T)=a_0 + a_1T + \cdots + a_mT^m$ and
  $f_2(T_1,\ldots,T_r,T)=b_0 + b_1T + \cdots + b_nT^n$ be two
  polynomials in $r+1$ variables, where $a_i$ and $b_i$ are
  polynomials in $T_1,\ldots,T_r$.
  % 
  Let $f(T_1,\ldots,T_r)=\res_T(f_1,f_2)$ be the resultant of $f_1$
  and $f_2$ regarded as two univariate polynomials in $T$.
  % 
  If both $m$ and $n$ are even, then every term of $f$ is a product of
  an even number or none of the odd-power coefficients of $f_1$ and
  $f_2$, some or none of the even-power coefficients of $f_1$ and
  $f_2$, and $\pm 1$.
  % 
  Specifically, the odd-power coefficients $a_{2k+1}$ and $b_{2k+1}$
  of $f_1$ and $f_2$, respectively, appear in total an even number of
  times in each term of $f$.
  % 
\end{lemma}
%
\begin{proof}
  % 
  The resultant $\res_T(f_1,f_2)$ of $f_1$ and $f_2$ is the
  determinant of the following $(m+n)\times(m+n)$ matrix $S$:
  % 
  \begin{equation*}
    S = \begin{bmatrix}
      a_m & a_{m-1} & \ldots & & a_0 & & &  \\
      & a_m & a_{m-1} & \ldots & & a_0  & &  \\
      & & \ddots & & & & \ddots &  \\
      & & & a_m & a_{m-1} & \ldots & & a_0  \\
      b_n & b_{n-1} & \ldots & & b_0 & & &  \\
      & b_n & b_{n-1} & \ldots & & b_0  & &  \\
      & & \ddots & & & & \ddots &  \\
      & & & b_n & b_{n-1} & \ldots & & b_0
    \end{bmatrix}.
  \end{equation*}
  % 
  We denote as $s_{ij}$ the entry at the $i$-th row and $j$-th column
  of $S$ for $1\leq i,j\leq m+n$.
  % 
  Since both $m$ and $n$ are even, an even-power coefficient $a_{2k}$
  or $b_{2k}$ will appear in $s_{ij}$ for which the sum of indices
  $i+j$ is even.
  % 
  Similarly, an odd-power coefficient $a_{2k+1}$ or $b_{2k+1}$ will
  appear in $s_{ij}$ for which the sum of indices $i+j$ is odd.
  % 
  Now recall that the determinant of $S$ is defined as
  \[ \sum_{\sigma\in S_{n+m}}\sgn(\sigma)s_{1,\sigma(1)}\cdot
    s_{2,\sigma(2)}\cdots s_{m+n,\sigma(m+n)}. \]
  % 
  We note that the sum of the indices of the factors is
  \[ \sum_i^{m+n}i+\sigma(i)=(m+n)(m+n+1), \] which is always even.
  % 
  Therefore, the odd-power coefficients must appear an even number of
  times, thus completing the proof.
  % 
  \qed
  % 
\end{proof}
%
\begin{lemma}
  % 
  \label{th:summation-polynomial}
  %
  Let $\mathcal E$ be a family of elliptic curves such that its 3rd
  summation polynomial $f_{3,\mathcal E}(X_1,X_2,X_3)$ is of degree 2
  in each variable $X_i$ and consists only of monomials of homogeneous
  parity.
  % 
  Then any subsequent summation polynomial
  $f_{m,\mathcal E}(X_1,\ldots,X_m)$ for $m>3$ consists only of
  monomials of homogeneous parity.
  % 
\end{lemma}
% 
\begin{proof}
  %
  As the summation polynomial $f_{m+1,\mathcal E}$ for
  $m\geq 3$ is defined recursively from $f_{m,\mathcal E}$ and
  $f_{3,\mathcal E}$ via taking resultants
  \[ f_{m+1,\mathcal E}(X_1,\dots,X_{m+1}) = \res_X\left(f_{m,\mathcal
        E}(X_1,\dots,X_{m-1},X),f_{3,\mathcal
        E}(X_m,X_{m+1},X)\right), \]
  % 
  we shall prove this lemma by induction on $m$.
  %
  Let
  $f_{m,\mathcal
    E}(X_1,\ldots,X_{m-1},X)=a_{2^{m-2}}X^{2^{m-2}}+\cdots+a_1X+a_0$
  and $f_{3,\mathcal E}(X_m,X_{m+1},X)=b_2X^2+b_1X+b_0$.
  % 
  By the premises that $f_{3,\mathcal E}$ consists only of monomials
  of homogeneous parity, $b_0$ and $b_2$ must consist only of
  monomials (in $X_m$ and $X_{m+1}$) of homogeneous even parity.
  % 
  Furthermore, $b_1=X_mX_{m+1}$.
  % 
  This is because the definition of monomials of homogeneous odd
  parity depends on the number of variables.
  % 
  In this case, the only such monomial is $X_mX_{m+1}X$.

  Now consider a term $c_kX_{m+1}^k$ of
  \[ f_{m+1,\mathcal
      E}(X_1,\ldots,X_m,X_{m+1})=c_{2^{m-1}}X_{m+1}^{2^{m-1}}+\cdots+c_1X_{m+1}+c_0 \]
  as a univariate polynomial in $X_{m+1}$.
  % 
  \begin{enumerate}
    %
  \item If $k$ is odd, then $b_1$ must appear an odd number of times
    in $c_kX_{m+1}^k$ to produce an odd power of $X_{m+1}$ because
    $b_0$ and $b_2$ can only produce even powers of $X_{m+1}$.
    % 
    \emph{As $b_1=X_mX_{m+1}$, $c_k$ must consist only of odd powers
      of $X_m$.}

    By induction hypothesis, $f_{m,\mathcal E}$ consists only of
    monomials of homogeneous parity.
    %
    Now consider an odd-power coefficient $a_{2\ell+1}$ of
    $f_{m,\mathcal E}$ as univariate polynomial in $X$.
    %
    This means that $a_{2\ell+1}$ consists only of monomials in
    $X_1,\ldots,X_{m-1}$ of homogeneous odd parity.
    % 
    By Lemma~\ref{th:resultant}, these odd-power coefficients appear
    in total an odd number of times in $c_kX_{m+1}^k$.
    % 
    \emph{Therefore, $c_k$ must consist only of odd powers of
      $X_1,\ldots,X_{m-1}$, and $c_kX_{m+1}^k$ is a monomial of
      homogeneous odd parity.}
    %
  \item By changing ``odd'' to ``even'' in the above argument, we
    conclude that \emph{for $k$ even, $c_kX_{m+1}^k$ is a monomial of
      homogeneous even parity.}
      % 
  \end{enumerate}
  % 
  \qed
  % 
\end{proof}

We are now ready to complete the proof of Theorem~\ref{th:main}.
%
By Lemma~\ref{th:summation-polynomial},
$g_{m,\mathcal E}(X_1,\ldots,X_m)=f_{m+1,\mathcal
  E}(X_1,\ldots,X_m,x)$ consists only of monomials of homogeneous
parity.
%
Obviously, the monomials of homogeneous even parity will remain of
even power after the substitution of $x$, as the exponent of $x$ must
be even as well.
%
If $m$ is even, then the monomials of homogeneous odd parity in
$f_{m+1,\mathcal E}$ will be an even-power monomial after the
substitution of $x$.
%
For odd $m$, the monomials of homogeneous odd parity will be of even
power after the substitution of $x$.
%
However, as $f_{m+1,\mathcal E}$ consists only of monomials of
homogeneous parity, it is impossible for some odd-power monomials to
appear in $g_{m,\mathcal E}$ after the substitution of $x$, thus
completing the proof of Theorem~\ref{th:main}.

%
% Isomorphisms among curves in various forms
%

% \section{Isomorphisms among curves in various forms}
% \label{sec:isomorphism}

%
To make an apple-to-apple comparison, we use curves in different forms
but are nonetheless isomorphic to one another over \F{p^n}, as
explained in Section~\ref{sec:isomorphism}.
%
That is, $H(\F{p^n})\cong W(\F{p^n})\cong M(\F{p^n})\cong tE(\F{p^n})$
as groups.
%
In other words, the curves in different forms have the same
j-invariant, and we consider the ECDLP in the same (largest)
prime-order subgroup in each of the experiments.
%
We will also state whether the factor base will be invariant under the
action of addition of 2-torsion points for each of the four forms
under investigation.

We start from a Hessian curve $H_d$ satisfying $x^3 + y^3 + 1 = 3dxy$
for $d\in\F{p^n}$ such that the number of its rational points
$\#H_d(\F{p^n})$ is divisible by 12.
%
As we have seen in Section~\ref{sec:hessian-t2}, the factor base of $H_d$
is in general not invariant under the addition of 2-torsion points.
%
From $H_d$, we can obtain an isomorphic Weierstrass curve $W_{a,b}$
satisfying $y^2 = x^3 + ax + b$ for $a = - 27d(d^3 + 8)$ and
$b = 54(d^6 - 20d^3 - 8)$~\cite{DBLP:conf/ches/Smart01}.
%
The isomorphism $\phi_{W,H}$ from $W_{a,b}(\F{p^n})$ to $H_d(\F{p^n})$
is defined over $\F{p^n}$ and is given by sending $(u,v)\in W_{a,b}$
to $(x,y)\in H_d$ where
\[ \left\{\begin{aligned}
x = & \frac{36(d^3 - 1) - v}{6(u + 9d^2)} - \frac{d}{2}, \\
y = & \frac{36(d^3 - 1) + v}{6(u + 9d^2)} - \frac{d}{2}.
\end{aligned}\right. \]
%
The inverse $\phi_{H,W}$ is given by
\[ \left\{\begin{aligned}
u = & \frac{12(d^3 - 1)}{d + x + y} - 9d^2, \\
v = & \frac{36(d^3 - 1)(y - x)}{d + x + y}.
\end{aligned}\right. \]
%
The factor base of $W_{a,b}$ is in general not invariant under the
addition of 2-torsion points~\cite{DBLP:journals/joc/FaugereGHR14}.

With a high probability, we can obtain a Montgomery curve $M_{A,B}$
satisfying $By^2 = x^3 + Ax^2 + x$ from $W_{a,b}$ by solving the
following equations
%
\[ \left\{\begin{aligned}
a = & \frac{3 - A^2}{3B^2}, \\
b = & \frac{2A^3 - 9A}{27B^3}.
\end{aligned}\right. \]
%
The isomorphism $\phi_{W,M}$ is defined over
$\F{p^n}$ and is given by sending $(u,v)\in W_{a,b}$ to
$(x,y)\in M_{A,B}$ where $x = Bu - 1/3A$ and $y = Bv$.
%
The inverse $\phi_{M,W}$ can be obtained by equation solving.
%
As we have seen in Section~\ref{sec:montgomery-symmetry}, the factor
base is invariant under the addition of a particular 2-torsion point
$(0,0)$, though we are not able to exploit this symmetry in general.

Finally, we can obtain a twisted Edwards curve $tE_{a',d'}$ satisfying
$a'x^2 + y^2 = 1 + d'x^2y^2$ from $M_{A,B}$ by taking
\[ \left\{\begin{aligned}
a' = & \frac{A + 2}{B}, \\
d' = & \frac{A - 2}{B}.
\end{aligned}\right. \]
%
Again we let $a_0=1/(a' - d')$ be the amount of quadratic twist.
%
The isomorphism $\phi_{W,tE}$ is defined over $\F{p^n}$ and given by
sending $(u,v)\in W_{a,b}$ to $(x,y)\in tE_{a',d'}$ where
\[ \left\{\begin{aligned}
x = & \frac{2a_0u}{v}, \\
y = & \frac{u - a_0}{u + a_0}.
\end{aligned}\right. \]
%
The inverse $\phi_{tE,W}$ is given by
\[ \left\{\begin{aligned}
u = & \frac{a_0(1 + y)}{1 - y}, \\
v = & \frac{2a_0^2(1 + y)}{x(1 - y)}.
\end{aligned}\right. \]
%
As shown by Faug\`ere, Gaudry, Hout, and
Renault~\cite{DBLP:journals/joc/FaugereGHR14}, the factor base is
invariant under the addition of the 2-torsion point $(0,-1)$.

\section{Experiment}
\label{sec:experiment}

We have conducted a set of experiments to compare the difficulty of
solving ECDLP for four different curves: Hessian($H$), Weierstrass($W$),
Montgomery($M$), and twisted Edwards($tE$) over \F{p^n}.
%

\subsection{Experimental conditions}
\label{subsec:conditions}

%
% Isomorphisms among curves in various forms
%

% \section{Isomorphisms among curves in various forms}
% \label{sec:isomorphism}

%
To make an apple-to-apple comparison, we use curves in different forms
but are nonetheless isomorphic to one another over \F{p^n}, as
explained in Section~\ref{sec:isomorphism}.
%
That is, $H(\F{p^n})\cong W(\F{p^n})\cong M(\F{p^n})\cong tE(\F{p^n})$
as groups.
%
In other words, the curves in different forms have the same
j-invariant, and we consider the ECDLP in the same (largest)
prime-order subgroup in each of the experiments.
%
We will also state whether the factor base will be invariant under the
action of addition of 2-torsion points for each of the four forms
under investigation.

We start from a Hessian curve $H_d$ satisfying $x^3 + y^3 + 1 = 3dxy$
for $d\in\F{p^n}$ such that the number of its rational points
$\#H_d(\F{p^n})$ is divisible by 12.
%
As we have seen in Section~\ref{sec:hessian-t2}, the factor base of $H_d$
is in general not invariant under the addition of 2-torsion points.
%
From $H_d$, we can obtain an isomorphic Weierstrass curve $W_{a,b}$
satisfying $y^2 = x^3 + ax + b$ for $a = - 27d(d^3 + 8)$ and
$b = 54(d^6 - 20d^3 - 8)$~\cite{DBLP:conf/ches/Smart01}.
%
The isomorphism $\phi_{W,H}$ from $W_{a,b}(\F{p^n})$ to $H_d(\F{p^n})$
is defined over $\F{p^n}$ and is given by sending $(u,v)\in W_{a,b}$
to $(x,y)\in H_d$ where
\[ \left\{\begin{aligned}
x = & \frac{36(d^3 - 1) - v}{6(u + 9d^2)} - \frac{d}{2}, \\
y = & \frac{36(d^3 - 1) + v}{6(u + 9d^2)} - \frac{d}{2}.
\end{aligned}\right. \]
%
The inverse $\phi_{H,W}$ is given by
\[ \left\{\begin{aligned}
u = & \frac{12(d^3 - 1)}{d + x + y} - 9d^2, \\
v = & \frac{36(d^3 - 1)(y - x)}{d + x + y}.
\end{aligned}\right. \]
%
The factor base of $W_{a,b}$ is in general not invariant under the
addition of 2-torsion points~\cite{DBLP:journals/joc/FaugereGHR14}.

With a high probability, we can obtain a Montgomery curve $M_{A,B}$
satisfying $By^2 = x^3 + Ax^2 + x$ from $W_{a,b}$ by solving the
following equations
%
\[ \left\{\begin{aligned}
a = & \frac{3 - A^2}{3B^2}, \\
b = & \frac{2A^3 - 9A}{27B^3}.
\end{aligned}\right. \]
%
The isomorphism $\phi_{W,M}$ is defined over
$\F{p^n}$ and is given by sending $(u,v)\in W_{a,b}$ to
$(x,y)\in M_{A,B}$ where $x = Bu - 1/3A$ and $y = Bv$.
%
The inverse $\phi_{M,W}$ can be obtained by equation solving.
%
As we have seen in Section~\ref{sec:montgomery-symmetry}, the factor
base is invariant under the addition of a particular 2-torsion point
$(0,0)$, though we are not able to exploit this symmetry in general.

Finally, we can obtain a twisted Edwards curve $tE_{a',d'}$ satisfying
$a'x^2 + y^2 = 1 + d'x^2y^2$ from $M_{A,B}$ by taking
\[ \left\{\begin{aligned}
a' = & \frac{A + 2}{B}, \\
d' = & \frac{A - 2}{B}.
\end{aligned}\right. \]
%
Again we let $a_0=1/(a' - d')$ be the amount of quadratic twist.
%
The isomorphism $\phi_{W,tE}$ is defined over $\F{p^n}$ and given by
sending $(u,v)\in W_{a,b}$ to $(x,y)\in tE_{a',d'}$ where
\[ \left\{\begin{aligned}
x = & \frac{2a_0u}{v}, \\
y = & \frac{u - a_0}{u + a_0}.
\end{aligned}\right. \]
%
The inverse $\phi_{tE,W}$ is given by
\[ \left\{\begin{aligned}
u = & \frac{a_0(1 + y)}{1 - y}, \\
v = & \frac{2a_0^2(1 + y)}{x(1 - y)}.
\end{aligned}\right. \]
%
As shown by Faug\`ere, Gaudry, Hout, and
Renault~\cite{DBLP:journals/joc/FaugereGHR14}, the factor base is
invariant under the addition of the 2-torsion point $(0,-1)$.


As explained in Section~\ref{sec:index-calculus-ecdlp}, we focus on
the PDP computation in these experiments, as the other bottleneck
computation, the linear algebra step is already well understood.
%
We focus on the bottleneck computation in PDP, namely, the cost of the
F4 algorithm for computing Gr\"obner bases for polynomial systems
obtained after rewriting using the elementary symmetric polynomials
and applying the Weil restriction technique to summation polynomials.
%
This way we will be taking advantage of the symmetry of $S_m$ acting
on point decompositions.
% 
However, we \emph{did not} exploit the symmetry of group actions on
point decomposition except the action of $S_m$.
%
This is because we want to compare the \emph{intrinsic} computational
complexity of solving PDP for various curves and hence can only
consider the symmetry that is present in \emph{all} curves.
%
Exploiting further curve-specific symmetry whenever possible will
result in further speed-up, but it would be independent of our
findings here.

All our experiment are done using the MAGMA computation algebra system
(version 2.23-1) on a single core of an Intel Xeon CPU E7-4830 v4
running at 2~GHz.
%
The main cost metrics are running time, Matcost, and the maximum step
degree reached during the execution of the F4 algorithm.
%
The last metric is usually referred to as the ``degree of regularity''
in the literature~\cite{DBLP:conf/indocrypt/GalbraithG14}, which
provides an upper bound for the size of the Macaulay submatrices
involved in the process.
%
The ``Matcost'' is a number output by the MAGMA implementation of the
F4 algorithm and provides an estimate of the linear algebra cost
during the execution of the F4 algorithm.

Lastly, we also focus on ``rank'', which represents the number of 
linearly independent relations we can get once successfully 
solving a PDP instance for each of the four curves.
%
It is also important factor to consider complexity of solving PDP
because rank decide the number of solving PDP which we need 
during the whole sequence of index-calculus attacks.


\subsection{Experimental results}

We present our experimental results for the case of $n=5$.
%
Here our factor base is to take $V$ as the base field \F p of \F{p^n}.
%

\begin{table}[!h]
\centering
\caption{$m=3$}
\label{tb:m=3}
\begin{tabular}{llrrrr}
\hline
\multicolumn{1}{|c|}{$q$}                    & \multicolumn{1}{l|}{Curve}       & \multicolumn{1}{l|}{Time} & \multicolumn{1}{l|}{Dreg} & \multicolumn{1}{l|}{Matcost} & \multicolumn{1}{l|}{Rank} \\ \hline
\multicolumn{1}{|l|}{\multirow{4}{*}{251}} & \multicolumn{1}{l|}{Hessian}     & \multicolumn{1}{r|}{0}    & \multicolumn{1}{r|}{6}    & \multicolumn{1}{r|}{41420.4} & \multicolumn{1}{r|}{1}    \\ \cline{2-6} 
\multicolumn{1}{|l|}{}                     & \multicolumn{1}{l|}{Weierstrass} & \multicolumn{1}{r|}{0}    & \multicolumn{1}{r|}{6}    & \multicolumn{1}{r|}{42132.0} & \multicolumn{1}{r|}{1}    \\ \cline{2-6} 
\multicolumn{1}{|l|}{}                     & \multicolumn{1}{l|}{Montgomery}  & \multicolumn{1}{r|}{0}    & \multicolumn{1}{r|}{6}    & \multicolumn{1}{r|}{61127.9} & \multicolumn{1}{r|}{4}    \\ \cline{2-6} 
\multicolumn{1}{|l|}{}                     & \multicolumn{1}{l|}{tEdwards}    & \multicolumn{1}{r|}{0}    & \multicolumn{1}{r|}{6}    & \multicolumn{1}{r|}{6308.4}  & \multicolumn{1}{r|}{4}    \\ \hline \vspace{-3mm}
                                           &                                  & \multicolumn{1}{l}{}      & \multicolumn{1}{l}{}      & \multicolumn{1}{l}{}         & \multicolumn{1}{l}{}      \\ \hline
\multicolumn{1}{|l|}{\multirow{4}{*}{239}} & \multicolumn{1}{l|}{Hessian}     & \multicolumn{1}{r|}{0}    & \multicolumn{1}{r|}{6}    & \multicolumn{1}{r|}{42336.8} & \multicolumn{1}{r|}{1}    \\ \cline{2-6} 
\multicolumn{1}{|l|}{}                     & \multicolumn{1}{l|}{Weierstrass} & \multicolumn{1}{r|}{0}    & \multicolumn{1}{r|}{6}    & \multicolumn{1}{r|}{41259}   & \multicolumn{1}{r|}{1}    \\ \cline{2-6} 
\multicolumn{1}{|l|}{}                     & \multicolumn{1}{l|}{Montgomery}  & \multicolumn{1}{r|}{0}    & \multicolumn{1}{r|}{6}    & \multicolumn{1}{r|}{61239}   & \multicolumn{1}{r|}{4}    \\ \cline{2-6} 
\multicolumn{1}{|l|}{}                     & \multicolumn{1}{l|}{tEdwards}    & \multicolumn{1}{r|}{0}    & \multicolumn{1}{r|}{6}    & \multicolumn{1}{r|}{6308.36} & \multicolumn{1}{r|}{4}    \\ \hline
\end{tabular}
\end{table}

\begin{table}[!h]
\centering
\caption{$m=4$}
\label{tb:m=4}
\begin{tabular}{clrrrr}
\hline
\multicolumn{1}{|c|}{$q$}                  & \multicolumn{1}{l|}{Curve}       & \multicolumn{1}{l|}{Time}  & \multicolumn{1}{l|}{Dreg} & \multicolumn{1}{l|}{Matcost}     & \multicolumn{1}{l|}{Rank} \\ \hline
\multicolumn{1}{|l|}{\multirow{4}{*}{251}} & \multicolumn{1}{l|}{Hessian}     & \multicolumn{1}{r|}{3.459} & \multicolumn{1}{r|}{19}   & \multicolumn{1}{r|}{12069800000} & \multicolumn{1}{r|}{1}    \\ \cline{2-6} 
\multicolumn{1}{|l|}{}                     & \multicolumn{1}{l|}{Weierstrass} & \multicolumn{1}{r|}{3.659} & \multicolumn{1}{r|}{19}   & \multicolumn{1}{r|}{12066400000} & \multicolumn{1}{r|}{1}    \\ \cline{2-6} 
\multicolumn{1}{|l|}{}                     & \multicolumn{1}{l|}{Montgomery}  & \multicolumn{1}{r|}{3.280} & \multicolumn{1}{r|}{18}   & \multicolumn{1}{r|}{11401700000} & \multicolumn{1}{r|}{5}    \\ \cline{2-6} 
\multicolumn{1}{|l|}{}                     & \multicolumn{1}{l|}{tEdwards}    & \multicolumn{1}{r|}{0.119} & \multicolumn{1}{r|}{18}   & \multicolumn{1}{r|}{54102900}    & \multicolumn{1}{r|}{5}    \\ \hline
 \vspace{-3mm}                                          &                                  & \multicolumn{1}{l}{}       & \multicolumn{1}{l}{}      & \multicolumn{1}{l}{}             & \multicolumn{1}{l}{}      \\ \hline
\multicolumn{1}{|l|}{\multirow{4}{*}{239}} & \multicolumn{1}{l|}{Hessian}     & \multicolumn{1}{r|}{3.990} & \multicolumn{1}{r|}{19}   & \multicolumn{1}{r|}{12066100000} & \multicolumn{1}{r|}{1}    \\ \cline{2-6} 
\multicolumn{1}{|l|}{}                     & \multicolumn{1}{l|}{Weierstrass} & \multicolumn{1}{r|}{3.680} & \multicolumn{1}{r|}{19}   & \multicolumn{1}{r|}{12064700000} & \multicolumn{1}{r|}{1}    \\ \cline{2-6} 
\multicolumn{1}{|l|}{}                     & \multicolumn{1}{l|}{Montgomery}  & \multicolumn{1}{r|}{3.489} & \multicolumn{1}{r|}{18}   & \multicolumn{1}{r|}{11399100000} & \multicolumn{1}{r|}{5}    \\ \cline{2-6} 
\multicolumn{1}{|l|}{}                     & \multicolumn{1}{l|}{tEdwards}    & \multicolumn{1}{r|}{0.150} & \multicolumn{1}{r|}{18}   & \multicolumn{1}{r|}{54093000}    & \multicolumn{1}{r|}{5}    \\ \hline
\end{tabular}
\end{table}


% \begin{table}[!h]
% \centering
% \caption{$m=3$}
% \label{tb:m=3}
% \begin{tabular}{llrrr}
% \hline
% \multicolumn{1}{|l|}{$q$}                  & \multicolumn{1}{l|}{Curve}       & \multicolumn{1}{l|}{Time} & \multicolumn{1}{l|}{Dreg} & \multicolumn{1}{l|}{Matcost} \\ \hline
% \multicolumn{1}{|l|}{\multirow{4}{*}{251}} & \multicolumn{1}{l|}{Hessian}     & \multicolumn{1}{r|}{0}    & \multicolumn{1}{r|}{6}    & \multicolumn{1}{r|}{41420.4} \\ \cline{2-5} 
% \multicolumn{1}{|l|}{}                     & \multicolumn{1}{l|}{Weierstrass} & \multicolumn{1}{r|}{0}    & \multicolumn{1}{r|}{6}    & \multicolumn{1}{r|}{42132.0} \\ \cline{2-5} 
% \multicolumn{1}{|l|}{}                     & \multicolumn{1}{l|}{Montgomery}  & \multicolumn{1}{r|}{0}    & \multicolumn{1}{r|}{6}    & \multicolumn{1}{r|}{61127.9} \\ \cline{2-5} 
% \multicolumn{1}{|l|}{}                     & \multicolumn{1}{l|}{tEdwards}    & \multicolumn{1}{r|}{0}    & \multicolumn{1}{r|}{6}    & \multicolumn{1}{r|}{6308.4}  \\ \hline \vspace{-3mm}
%                                            &                                  &                           &                           &                              \\ \hline
% \multicolumn{1}{|l|}{\multirow{4}{*}{239}} & \multicolumn{1}{l|}{Hessian}     & \multicolumn{1}{r|}{0}    & \multicolumn{1}{r|}{6}    & \multicolumn{1}{r|}{42336.8} \\ \cline{2-5} 
% \multicolumn{1}{|l|}{}                     & \multicolumn{1}{l|}{Weierstrass} & \multicolumn{1}{r|}{0}    & \multicolumn{1}{r|}{6}    & \multicolumn{1}{r|}{41259.0} \\ \cline{2-5} 
% \multicolumn{1}{|l|}{}                     & \multicolumn{1}{l|}{Montgomery}  & \multicolumn{1}{r|}{0}    & \multicolumn{1}{r|}{6}    & \multicolumn{1}{r|}{61239.0} \\ \cline{2-5} 
% \multicolumn{1}{|l|}{}                     & \multicolumn{1}{l|}{tEdwards}    & \multicolumn{1}{r|}{0}    & \multicolumn{1}{r|}{6}    & \multicolumn{1}{r|}{6308.4}  \\ \hline
% \end{tabular}
% \end{table}


% \begin{table}[!h]
% \centering
% \caption{$m=4$}
% \label{tb:m=4}
% \begin{tabular}{llrrr}
% \hline
% \multicolumn{1}{|l|}{$q$}                  & \multicolumn{1}{l|}{Curve}       & \multicolumn{1}{l|}{Time}  & \multicolumn{1}{l|}{Dreg} & \multicolumn{1}{l|}{Matcost}     \\ \hline
% \multicolumn{1}{|l|}{\multirow{4}{*}{251}} & \multicolumn{1}{l|}{Hessian}     & \multicolumn{1}{r|}{3.459} & \multicolumn{1}{r|}{19}   & \multicolumn{1}{r|}{12069800000} \\ \cline{2-5} 
% \multicolumn{1}{|l|}{}                     & \multicolumn{1}{l|}{Weierstrass} & \multicolumn{1}{r|}{3.659} & \multicolumn{1}{r|}{19}   & \multicolumn{1}{r|}{12066400000} \\ \cline{2-5} 
% \multicolumn{1}{|l|}{}                     & \multicolumn{1}{l|}{Montgomery}  & \multicolumn{1}{r|}{3.280} & \multicolumn{1}{r|}{18}   & \multicolumn{1}{r|}{11401700000} \\ \cline{2-5} 
% \multicolumn{1}{|l|}{}                     & \multicolumn{1}{l|}{tEdwards}    & \multicolumn{1}{r|}{0.119} & \multicolumn{1}{r|}{18}   & \multicolumn{1}{r|}{54102900}    \\ \hline \vspace{-3mm}
%                                            &                                  &                            &                           &                                  \\ \hline
% \multicolumn{1}{|l|}{\multirow{4}{*}{239}} & \multicolumn{1}{l|}{Hessian}     & \multicolumn{1}{r|}{3.990} & \multicolumn{1}{r|}{19}   & \multicolumn{1}{r|}{12066100000} \\ \cline{2-5} 
% \multicolumn{1}{|l|}{}                     & \multicolumn{1}{l|}{Weierstrass} & \multicolumn{1}{r|}{3.680} & \multicolumn{1}{r|}{19}   & \multicolumn{1}{r|}{12064700000} \\ \cline{2-5} 
% \multicolumn{1}{|l|}{}                     & \multicolumn{1}{l|}{Montgomery}  & \multicolumn{1}{r|}{3.489} & \multicolumn{1}{r|}{18}   & \multicolumn{1}{r|}{11399100000} \\ \cline{2-5} 
% \multicolumn{1}{|l|}{}                     & \multicolumn{1}{l|}{tEdwards}    & \multicolumn{1}{r|}{0.150} & \multicolumn{1}{r|}{18}   & \multicolumn{1}{r|}{54093000}    \\ \hline
% \end{tabular}
% \end{table}

We can clearly see that the PDP solving time and matcost for twisted 
Edwards curve is much faster and smaller than the other three curves 
for different $q$ and $m$.
%
In terms of Dreg, degree for Montgomery and twisted Edwards curve are 
smaller than other curves in the case of $m=4$.
%
Also, we can see that Rank for Hessian and Weierstrass curve is 
1 in all cases.
%
On the other hand, the Rank for Montgomery and twisted Edwards curve
is larger, that is 4 and 5 respectively in the case of $m=3$ and 4.
%
% We further analyze which terms are missing from the summation
% polynomials for curves in different forms.
% %
% We classify terms into odd vs.~even degrees.
% %
% Recall that the polynomials are expressed using elementary symmetric polynomials 
% $e_1, \dots ,e_m$.
% %
% Therefore, a monomial $e_i^j$ has odd degree only if both $i$ and $j$
% are odd.
% %
% For $m=2$, it is interesting that there are no monomials of odd degree
% only for twisted Edwards curve.
% %
% In other words, there are no terms originated from monomial such as
% $e_1$ or $e_1e_2$.
% %
% This is because the addition formula for twisted Edwards curve, based
% on which the summation polynomial is derived, consists of only even
% terms.
% %
% Therefore, when one of the 3 variables in the summation polynomial is
% substituted with the point we want to decompose, the odd-degree terms
% all vanish.

% For $m=3$, we use summation polynomial with 4 variables, which is
% obtained by taking the resultant of two summation polynomials 
% with 3 variables.
% %
% Therefore, there are some odd terms such as $e_3$ or $e_2^2e_3$.
% %
% For $m=4$, there is also no terms with odd degree.
% %
% Through calculating resultant recursively, two variables are
% eliminated, and it preserves all terms to even degree.








%
% ---- Bibliography ----
%
\bibliographystyle{abbrv}
\bibliography{dblp,local}

\end{document}
