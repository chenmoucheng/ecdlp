%
%
% This is LLNCS.DEM the demonstration file of
% the LaTeX macro package from Springer-Verlag
% for Lecture Notes in Computer Science,
% version 2.4 for LaTeX2e as of 16. April 2010
%
\documentclass{llncs}
%
\usepackage{algpseudocode,amsfonts,amsmath,amssymb}
\usepackage[scale=.78]{geometry}
% \geometry{
%   a4paper,         % or letterpaper
%   textwidth=15cm,  % llncs has 12.2cm
%   textheight=24cm, % llncs has 19.3cm
%   heightrounded,   % integer number of lines
%   hratio=1:1,      % horizontally centered
%   vratio=2:3,      % not vertically centered
% }
\usepackage{latexsym,mathrsfs,multirow}
%
\DeclareMathOperator{\ord}{ord}
\DeclareMathOperator{\res}{Res}
\DeclareMathOperator{\sgn}{sgn}
\newcommand{\F}[1]{\ensuremath{\mathbb F_{#1}}}
%\def\F{{\mathbb F}}
\def\Fqn{{\mathbb F}_{p^n}}
\def\Fqm{{\mathbb F}_{p^m}}
\def\qed{\hfill $\Box$}
\long\def\comment#1{}
%
\begin{document}
%
\title{On the computational complexity of ECDLP for elliptic curves in
  various forms using index calculus}
%
%\titlerunning{}
%
%\author{Kenta~Kodera \and Atsuko~Miyaji \and Chen-Mou~Cheng}
%
%\authorrunning{}
%
%\institute{Osaka University, Japan}

\maketitle              % typeset the title of the contribution

\begin{abstract}
%
  The security of elliptic curve cryptography is closely related to
  the computational complexity of the elliptic curve discrete
  logarithm problem (ECDLP).
%
  Today, the best practical attacks against ECDLP are
  exponential-time, generic discrete logarithm algorithms such as
  Pollard's rho method~\cite{1978-pollard-kangaroo}.
%
  Recently, there is a line of research on index calculus for ECDLP
  started by Semaev, Gaudry, and
  Diem~\cite{DBLP:journals/iacr/Semaev04,DBLP:journals/jsc/Gaudry09,DBLP:journals/moc/Diem11}.
%
  Under certain heuristic assumptions, such algorithms could lead to
  subexponential attacks to ECDLP in some
  cases~\cite{DBLP:conf/eurocrypt/FaugerePPR12,DBLP:journals/iacr/PetitQ12,DBLP:conf/iwsec/HuangPST13}.
%
  In this paper, we investigate the computational complexity of ECDLP
  for elliptic curves in various forms---including
  Hessian~\cite{DBLP:conf/ches/Smart01},
  Montgomery~\cite{1987-montgomery}, (twisted)
  Edwards~\cite{DBLP:journals/iacr/BernsteinL07,DBLP:journals/iacr/BernsteinBJLP08},
  and Weierstrass using index calculus.
%
  The research question we would like to answer is: Using index
  calculus, is there any significant difference in the computational
  complexity of ECDLP for elliptic curves in various forms?
%
  We will provide some empirical evidence and insights showing an
  affirmative answer in this paper.
%
  \keywords{Security evaluation, ECDLP, index calculus, summation
    polynomial, point decomposition problem}
\end{abstract}

%
% Introduction
%

\section{Introduction}
%
In recent years, elliptic curve cryptography is gaining momentum in
deployment, as it can achieve the same level of security as RSA using
much shorter keys and ciphertexts.
%
The security of elliptic curve cryptography is closely related to the
computational complexity of solving the elliptic curve discrete
logarithm problem (ECDLP).
%
Let $p$ be a prime number, and $E$, a nonsingular elliptic curve over
\F{p^n}, the finite field of $p^n$ elements for some positive integer
$n$.
%
That is, $E$ is a plane algebraic curve defined by the equation
$y^2=x^3+ax+b$ for $a,b\in\F{p^n}$ and $\Delta=-16(4a^3+27b^2)\neq 0$.
%
Along with a point at infinity $\mathcal O$, the set of rational
points $E(\F{p^n})$ forms an abelian group with $\mathcal O$ as the
identity element.
%
Given $P\in E(\F{p^n})$ and $Q$ in the subgroup generated by $P$,
ECDLP is the problem of finding an integer $\alpha$ such that
$Q=\alpha P$.

Today, the best practical attacks against ECDLP are exponential-time,
generic discrete logarithm algorithms such as Pollard's rho
method~\cite{1978-pollard-kangaroo}.
%
However, recently there is a line of research on index calculus
algorithms for ECDLP started by Semaev, Gaudry, and
Diem~\cite{DBLP:journals/iacr/Semaev04,DBLP:journals/jsc/Gaudry09,DBLP:journals/moc/Diem11}.
%
Under certain heuristic assumptions, such algorithms could lead to
subexponential attacks to ECDLP in some
cases~\cite{DBLP:conf/eurocrypt/FaugerePPR12,DBLP:journals/iacr/PetitQ12,DBLP:conf/iwsec/HuangPST13}.
%
The interested reader is referred to a survey paper by Galbraith and
Gaudry for a more comprehensive and in-depth account of the recent
development of ECDLP algorithms along various
directions~\cite{DBLP:journals/dcc/GalbraithG16}.

In this paper, we consider the complexity of solving ECDLP for
elliptic curves in various forms---including
Hessian~\cite{DBLP:conf/ches/Smart01},
Montgomery~\cite{1987-montgomery}, (twisted)
Edwards~\cite{DBLP:journals/iacr/BernsteinL07,DBLP:journals/iacr/BernsteinBJLP08},
and Weierstrass using index calculus algorithms.
%
Recently, elliptic curves of various forms such as
Curve25519~\cite{DBLP:conf/pkc/Bernstein06} have been drawing a lot of
attention in deployment, partly because some of them allow fast
implementation that is secure against timing-based side channel
attacks.
%
Furthermore, we can construct these curves not only over prime fields
(such as the field of $2^{255} - 19$ elements used in Curve25519) but
also extension fields.
%
In this paper, we will focus on curves over optimal extension fields
(OEFs)~\cite{DBLP:conf/crypto/BaileyP98}.
%
An OEF is an extension field from a prime field \F p with $p$ close to
$2^8, 2^{16}, 2^{32}, 2^{64}$, etc.
%
Such primes fit nicely into the processor words of 8, 16, 32, or
64-bit microprocessors and hence are particularly suitable for
software implementation, allowing for efficient utilization of fast
integer arithmetics on modern
microprocessors~\cite{DBLP:conf/crypto/BaileyP98}.
%
As we will see, our experimental results show quite significant
difference in the computational complexity of solving ECDLP for
elliptic curves in various forms over OEFs.

The rest of this paper is organized as follows.
%
In Section~\ref{sec:index-calculus-ecdlp}, we will give a brief
introduction on solving ECDLP using index calculus algorithms.
%
In Section~\ref{sec:montgomery-symmetry} and
\ref{sec:hessian-edwards}, we will present how we can attack ECDLP
using index calculus algorithms for elliptic curves in Montgomery,
Hessian, and (twisted) Edwards forms.
%
In Section~\ref{sec:twisted-edwards-summation-polynomial}, we further
analyze and characterize a family of elliptic curves whose summation
polynomials are ``easier'' to solve.
%
In Section~\ref{sec:isomorphism}, we describe the set of isomorphic
curves that we are going to experiment with.
%
Finally, we present the experimental results and conclude this paper
in Section~\ref{sec:experiment}.
%
% Index calculus for ECDLP
%

\section{Previous works}
\label{sec:previous-work}

\subsection{Index calculus for ECDLP}
%
\label{sec:index-calculus-ecdlp}
%
Let $E$ be an elliptic curve defined over a finite field \F{p^n}.
%
For cryptographic applications, we are mostly interested in a
prime-order subgroup generated by a rational point $P\in E(\F{p^n})$.
%
Here we first give a high-level overview of a typical index calculus
algorithm for finding an integer $\alpha$ such that $Q=\alpha P$ for
$Q\in\langle P\rangle$.
%
\begin{enumerate}
%
\item Determine a \emph{factor base} $\mathcal F\subset E(\F{p^n})$.
%
\item Collect a set $\mathcal R$ of \emph{relations} by decomposing
  random points $a_iP+b_iQ$ into a sum of points from $\mathcal F$,
  i.e.,
  \[ \mathcal R=\left\{a_iP+b_iQ=\sum_jP_{i,j}:P_{i,j}\in\mathcal
      F\right\}. \]
%
\item When $|\mathcal R|\approx|\mathcal F|$, eliminate the righthand
  side using linear algebra to obtain an equation in the form
  $aP+bQ=\mathcal O$, and $\alpha=-a/b\bmod\ord(P)$.
%
\end{enumerate}
%
The last step of linear algebra is relatively well studied in the
literature, so we will focus on the subproblem in the second step,
namely, the point decomposition problem (PDP) on an elliptic curve in
the rest of this paper.
%
\begin{definition}[Point Decomposition Problem of $m$-th order]
\label{def:pdp}
% 
Given a rational point $R\in E(\F{p^n})$ on an elliptic curve $E$ and
a factor base $\mathcal F\subset E(\F{p^n})$, find, if they exist,
$P_1,\dots,P_m\in\mathcal F$ such that \[ R=P_1 + \cdots + P_m. \]
%
\end{definition}

\subsection{Semaev's summation polynomials}
%
\label{sec:summation-polynomial}
%
We can solve PDP by considering when a set of points sum to zero on an
elliptic curve.
%
It is straightforward that if two points sum to zero on an elliptic
curve $E: y^2=x^3+ax+b$ in Weierstrass form, then their
$x$-coordinates must be equal.
%
Let us now consider the simplest yet nontrivial case where three
points on $E$ sum to zero.
%
Let
\[ Z=\left\{\begin{aligned}
      (x_1,y_1,x_2,y_2,x_3,y_3)&\in\F{p^n}^6:(x_i,y_i)\in E(\F{p^n}),i=1,2,3; \\
      & (x_1,y_1)+(x_2,y_2)+(x_3,y_3)=\mathcal O
    \end{aligned} \right\}. \]
%
Clearly, $Z$ is in the variety of the ideal
$I\subset\F{p^n}[X_1,Y_1,X_2,Y_2,X_3,Y_3]$ generated by
\[ \left\{\begin{aligned}
      & Y_i^2 - (X_i^3 + aX_i + b),i=1,2,3; \\
      &  (X_3 - X_1)(Y_2 - Y_1) - (X_2 - X_1)(Y_3 - Y_1)\\
    \end{aligned}\right\}. \]
%
Now let $J=I\cap\F{p^n}[X_1,X_2,X_3]$.
%
Using MAGMA's \texttt{EliminationIdeal} function, we obtain that $J$
is actually a principal ideal generated by the polynomial
$(X_2 - X_3)(X_1 - X_3)(X_1 - X_2)f_3$, where
%
\begin{align*}
  f_3 = & X_1^2X_2^2 - 2X_1^2X_2X_3 + X_1^2X_3^2 - 2X_1X_2^2X_3 - 2X_1X_2X_3^2 - 2aX_1X_2 - 2aX_1X_3 \\
        & - 4bX_1 + X_2^2X_3^2 - 2aX_2X_3 - 4bX_2 - 4bX_3 + a^2.
\end{align*}
%
Clearly, the linear factors of this generator correspond to the
degenerated case where two or more points are the same or of opposite
signs, and $f_3$ is the 3rd \emph{summation polynomial}, that is, the
summation polynomial for three distinct points summing to zero.

Starting from the 3rd summation polynomial, we can recursively
construct the subsequent summation polynomials $f_m$ for $m>3$ via
taking resultants.
%
As a result, the degree of each variable in $f_m$ is $2^{m-2}$, which
grows exponentially as $m$.
%
This is the observation Semaev made in his seminal
work~\cite{DBLP:journals/iacr/Semaev04}.
%
In short, his proposal is to consider factor bases of the following
form:
\[ \mathcal F=\Big\{(x,y)\in E(\F{p^n}):x\in V\subset\F{p^n}\Big\}, \]
where $V$ is a subset of \F{p^n}.
%
Then we solve PDP of $m$-th order via solving the corresponding
$(m+1)$-th summation polynomial $f_{m+1}(X_1,\ldots,X_m,\tilde x)=0$,
where $\tilde x$ is the $x$-coordinate of the point to be decomposed.

Note that this factor base is naturally invariant under point
negation.
%
That is, $P_i\in\mathcal F$ implies $-P_i\in\mathcal F$.
%
In this case, we have about $|\mathcal F|/2$ (trivial) relations
$P_i+(-P_i)=\mathcal O$ for free, so we just need to find the other
$|\mathcal F|/2$ nontrivial relations.
%
In general, we will only discuss factor bases that are invariant under
point negation, so by abuse of language, both $\mathcal F$ and
$\mathcal F$ modulo point negation may be referred to as a factor base
in the rest of this paper.

\subsection{Weil restriction}
\label{sec:weil-restriction}
%
Restricting the $x$-coordinates of the points in a factor base to a
subset of \F{p^n} is important from a viewpoint of polynomial system
solving.
%
Take $f_3$ as an example.
%
When decomposing a random point $aP+bQ$, we first substitute its
$x$-coordinate into say $X_3$, projecting the ideal onto
$\F{p^n}[X_1,X_2]$.
%
The dimension of the variety of this ideal is nonzero.
%
Therefore, we would like to pose some restrictions on $X_1$ and $X_2$
to reduce the dimension to zero so that the solving time can be more
manageable.

When looking for solutions to a polynomial
$f=\sum a_iX^i\in\F{p^n}[X]$ in \F{p^n}, we can view $\F{p^n}[X]$ as a
commutative affine algebra
$\mathcal A=\F{p^n}[X]/(X^{p^n} -
X)\cong\F{p^n}[X_1,\ldots,X_n]/(X_1^p - X_1,\ldots,X_n^p - X_n)$.
%
This can be done by identifying the indeterminate $X$ as
$X_1\theta_1+\cdots+X_n\theta_n$, where $(\theta_1,\ldots,\theta_n)$
is a basis for \F{p^n} over \F p.
%
Hence, $f$ can be identified as a polynomial
$f_1\theta_1+\cdots+f_n\theta_n$, where
$f_1,\ldots,f_n\in\mathcal A'=\F p[X_1,\ldots,X_n]/(X_1^p -
X_1,\ldots,X_n^p - X_n)$, by appropriately sending each coefficient
$a_i\in\F{p^n}$ to $a_i^{(1)}\theta_1+\cdots+a_i^{(n)}\theta_n$ for
$a_i^{(1)},\ldots,a_i^{(n)}\in\F p$.
%
Therefore, an equation $f=0$ over \F{p^n} will give rise to a system
of equations $f_1=\cdots=f_n=0$ over \F p.
%
This technique is known as the \emph{Weil restriction} and is used in
the Gaudry-Diem attack, in which the factor base is chosen to consist
of points whose $x$-coordinates lie in a subspace $V$ of \F{p^n} over
\F p~\cite{DBLP:journals/jsc/Gaudry09,DBLP:journals/moc/Diem11}.

\subsection{Exploiting symmetry}
%
\label{sec:exploit-symmetry}
%
Naturally, the symmetric group $S_m$ acts on a point decomposition
$P_1+\ldots+P_m$ because elliptic curve groups are abelian.
%
As noted by Gaudry in his seminal
work~\cite{DBLP:journals/jsc/Gaudry09}, we can therefore rewrite the
variables $x_1,\ldots,x_m\in\F{p^n}$ by elementary symmetric
polynomials $e_1,\ldots,e_m$, where $e_1=\sum x_i$,
$e_2=\sum_{i\neq j}x_ix_j$,
$e_3=\sum_{i\neq j,i\neq k,j\neq k}x_ix_jx_k$, etc.
%
Such rewriting can reduce the degree of summation polynomials and
significantly speed up point
decomposition~\cite{DBLP:conf/eurocrypt/FaugerePPR12,DBLP:conf/iwsec/HuangPST13}.

We might be able to exploit additional symmetry brought by actions of
other groups, e.g., when the factor base is invariant under addition
of small torsion points.
%
For example, consider a decomposition of a point $R$ under the action
of addition of a 2-torsion point $T_2$:
\[ R = P_1+\cdots+P_n =
  (P_1+u_1T_2)+\cdots+(P_{n-1}+u_{n-1}T_2)+\left(P_n+\left(\sum_{i=1}^{n-1}u_i\right)T_2\right). \]
%
Clearly this holds for any $u_1,\ldots,u_{n-1}\in\{0,1\}$, so a
decomposition can give rise to $2^{n-1-1}$ other decompositions.
%
Similar to rewriting using the elementary symmetric polynomials for
the action of $S_m$, we can also take advantage of this additional
symmetry by appropriate
rewriting~\cite{DBLP:journals/joc/FaugereGHR14}.

Naturally, such kind of speed-up is curve-specific.
%
Furthermore, even if the factor base is invariant under additional
group actions, we may or may not be able to exploit such symmetry to
speed up point decomposition depending on whether the action is ``easy
to handle in the polynomial system solving
process''~\cite{DBLP:journals/joc/FaugereGHR14}.


%------------------------------
\subsection{PDP on (twisted) Edwards curves}
%------------------------------------
\label{sec:twisted-edwards}


Faug\`ere, Gaudry, Hout, and Renault studied PDP on twisted Edwards,
twisted Jacobi intersections, and Weierstrass
curves~\cite{DBLP:journals/joc/FaugereGHR14}.
%
For the sake of completeness, we include some of their results here.
%
An Edwards curve over \F{p^n} for $p\neq 2$ is defined by the equation
$x^2+y^2=1+dx^2y^2$ for certain
$d\in\F{p^n}$~\cite{DBLP:journals/iacr/BernsteinL07}.
%
A twisted Edwards curve $tE_{a,d}$ over \F{p^n} for $p\neq 2$ is
defined by the equation $ax^2+y^2=1+dx^2y^2$ for certain
$a,d\in\F{p^n}$~\cite{DBLP:journals/iacr/BernsteinBJLP08}.
%
A twisted Edwards curve is a quadratic twist of an Edwards curve by
$a_0=1/(a-d)$.
%
For $P=(x,y)\in tE_{a,d}$, $-P=(-x,y)$.
%
Furthermore, the addition and doubling formulae for
$(x_3,y_3)=(x_1,y_1)+(x_2,y_2)$ are given as follows.
%
\begin{flalign*}
  \text{When }(x_1,y_1)\neq (x_2,y_2):\left\{\begin{aligned}
      x_3 & = \frac{x_1y_2 + y_1x_2}{1 + dx_1x_2y_1y_2}, \\
      y_3 & = \frac{y_1y_2 - ax_1x_2}{1 - dx_1x_2y_1y_2}.
    \end{aligned}\right. &&
\end{flalign*}
%
\begin{flalign*}
  \text{When }(x_1,y_1)=(x_2,y_2):\left\{\begin{aligned}
      x_3 & = \frac{2x_1y_1}{1 + dx_1^2y_1^2}, \\
      y_3 & = \frac{y_1^2 - ax_1^2}{1 - dx_1^2y_1^2}.
    \end{aligned}\right. &&
\end{flalign*}
%
The 3rd summation polynomial for twisted Edwards curves is
%
\begin{align*}
  f_{tE, 3}(Y_1,Y_2,Y_3) = & \left(Y_1^2Y_2^2 - Y_1^2 - Y_2^2 +
                             \frac{a}{d}\right)Y_3^2  \\
                           & + 2\frac{d-a}{d}Y_1Y_2Y_3 +
                             \frac{a}{d}\left(Y_1^2 + Y_2^2 - 1\right)
                             -
                             Y_1^2Y_2^2~\cite{DBLP:journals/joc/FaugereGHR14}.
\end{align*}
%
Again subsequent summation polynomials are obtained by taking
resultants.



\subsection{Symmetry and decomposition probability}
\label{sec:symmetry-decomposition-probability}

Symmetry brought by group action on point decomposition will
inevitably be accompanied by \emph{decrease in decomposition
  probability}.
%
For example, if a factor base $\mathcal F$ is invariant under addition
of a 2-torsion point, then the decomposition probability for PDP of
$m$-th order should decrease by a factor of $2^{m-1}$.
%
This is due to the same reason that the decomposition probability
decreases by a factor of $m!$ because the symmetric group $S_m$ acts
on $\mathcal F$.

However, this simple fact seems to have been largely ignored in the
literature.
%
For example, Faug\`ere, Gaudry, Hout, and Renault explicitly stated in
Section~5.3 of their paper that ``[the] probability to decompose a
point [into a sum of $n$ points from the factor base] is
$\frac{1}{n!}$'' for twisted Edwards or twisted Jacobi intersections
curves, despite the fact that the factor base is invariant under
addition of 2-torsion points~\cite{DBLP:journals/joc/FaugereGHR14}.
%
At a first glance, this may not seem a problem, as we would expect to
obtain $2^{n-1}$ solutions if we can successfully solve a PDP
instance.
%
(Unfortunately this is also \emph{not true} in general.  We will come
back to it in more detail in Section~\ref{sec:price}.)
% 
However, when estimating the cost of a complete ECDLP attack, they
proposed to \emph{collapse} these $2^{n-1}$ relations into one in
order to reduce the size of the factor base and thus the cost of the
linear algebra, cf.~Remark 5 of the paper.
%
In this case, the decrease in decomposition probability \emph{does}
have an adverse effect, and their estimation for the overall ECDLP
cost ended up being overoptimistic by a factor of at least $2^{n-1}$.


\section{Montgomery and Hessian curves}
\label{sec:montgomery-hessian}
%
% Montgomery curves
%

\subsection{Montgomery curves}
\label{sec:montgomery-symmetry}
%
A Montgomery curve $M_{A,B}$ over \F{p^n} for $p\neq 2$ is defined by
the equation \begin{equation}
  By^2=x^3+Ax^2+x \label{eq:montgomery-curve} \end{equation} for
$A,B\in\F{p^n}$ such that $A\neq\pm 2$, $B\neq 0$, and
$B(A^2-4)\neq 0$~\cite{1987-montgomery}.
%
For $P=(x,y)\in M_{A,B}$, $-P=(x,-y)$.
%
Furthermore, the addition and doubling formulae for
$(x_3,y_3)=(x_1,y_1)+(x_2,y_2)$ are given as follows.
%
\[ \text{When }(x_1,y_1)\neq(x_2,y_2):\left\{\begin{aligned}
      x_3 & = B\left(\frac{y_2 - y_1} {x_2 - x_1}\right)^2 - A - x_1 - x_2 = \frac{B(x_2y_1 - x_1y_2)^2} {x_1x_2(x_2 - x_1)^2}, \\
      y_3 & = \frac{(2x_1 + x_2 + A)(y_2 - y_1)} {x_2 - x_1} -
      \frac{B(y_2 - y_1)^3} {(x_2 - x_1)^3} - y_1.
    \end{aligned}\right. \] \[ \text{When }(x_1,y_1)=(x_2,y_2):\left\{\begin{aligned}
      x_3 & = \frac{(x_1^2 - 1)^2} {4x_1(x_1^2 + Ax_1 + 1)},  \\
      y_3 & = \frac{(2x_1 + x_1 + A)(3x_1^2 + 2Ax_1 + 1)} {2By_1} -
      \frac{B(3x_1^2 + 2Ax_1 + 1)^3} {(2By_1)^3} - y_1.
  \end{aligned}\right. \]
%
It was noted by Montgomery himself in his original paper that such
curves can give rise to efficient scalar multiplication
algorithms~\cite{1987-montgomery}.
%
That is, consider a random point $P\in M_{A,B}(\F{p^n})$ and let
$nP=(X_n:Y_n:Z_n)$ in projective coordinates for some integer $n$.
%
Then:
%
\[ \left\{\begin{aligned}
      X_{m+n} & = Z_{m-n}[(X_m - Z_m)(X_n + Z_n) + (X_m + Z_m)(X_n - Z_n)]^2, \\
      Z_{m+n} & = X_{m-n}[(X_m - Z_m)(X_n + Z_n) - (X_m + Z_m)(X_n -
      Z_n)]^2.
    \end{aligned}\right. \]
%
In particular, when $m=n$:
\[ \left\{\begin{aligned}
      X_{2n} & = (X_n + Z_n)^2(X_n - Z_n)^2, \\
      Z_{2n} & = (4X_nZ_n)\left((X_n - Z_n)^2 + ((A+2)/4)(4X_nZ_n)\right), \\
      4X_nZ_n & = ( X_n + Z_n)^2 - (X_n - Z_n)^2.
    \end{aligned}\right. \]
%
In this way, scalar multiplication on Montgomery curve can be
performed without using $y$-coordinates, leading to fast
implementation.

\subsection{Summation polynomials for Montgomery curves}

Following Semaev's approach~\cite{DBLP:journals/iacr/Semaev04}, we can
construct summation polynomials for Montgomery curves.
%
Like Weierstrass curves, the 2nd summation polynomial for Montgomery
curves is simply $f_{M,2} = X_1 - X_2$.
%
Now consider $P,Q\in M_{A, B}$ where $P=(x_1, y_1)$ and
$Q=(x_2, y_2)$.
%
Let $P+Q=(x_3, y_3)$ and $P-Q=(x_4, y_4)$.
%
By the addition formula, we have
\[ x_3 = \frac{B(x_2y_1 - x_1y_2)^2} {x_1x_2(x_2 - x_1)^2},
  x_4 =\frac{B(x_2y_1 - x_1y_2)^2} {x_1x_2(x_2 + x_1)^2}. \]
%
It follows that
%
\[ \left\{\begin{aligned} x_3 + x_4&=\frac{2\left((x_1 + x_2)(x_1x_2 + 1) + 2Ax_1x_2\right)}{(x_1 - x_2)^2}, \\
      x_3x_4&=\frac{(1 - x_1x_2)^2}{(x_1 - x_2)^2}.
    \end{aligned}\right. \]
%
Using the relationship between the roots of a quadratic polynomial and
its coefficients, we obtain
\[ (x_1 - x_2)^2x^2 - 2\left((x_1 + x_2)(x_1x_2 + 1) +
    2Ax_1x_2\right)x + (1 - x_1x_2)^2. \]
%
From here, we can obtain for Montgomery curve the 3rd summation
polynomial:
\[ f_{M,3}(X_1,X_2,X_3) = (X_1 - X_2)^2X_3^2 - 2\left((X_1 +
    X_2)(X_1X_2 + 1) + 2AX_1X_2\right)X_3 + (1-X_1X_2)^2, \]
%
as well as the subsequent summation polynomials via taking resultants:
\[ f_{M,m}(X_1,\ldots,X_m) =
  \res_X\left(f_{M,m-k}(X_1,\ldots,X_{m-k-1},X),f_{M,k+2}(X_{m-k},\ldots,X_m,X)\right). \]

  
%--------------------------------
\subsection{Small torsion points on Montgomery
  curves} \label{subsec:TSPL}
%------------------------
A Montgomery curve always contains an affine 2-torsion point $T_2$.
%
Since $T_2+T_2=2T_2=\mathcal O$, it follows that $-T_2=T_2$.
%
If we write $T_2=(x,y)$, then we can see that $y=0$ in order for
$-T_2=T_2$, as $p\neq 2$.
%
Substituting $y=0$ into Equation~(\ref{eq:montgomery-curve}),
we get an equation $x^3+Ax^2+x=0$.
%
The lefthand side factors into $x(x^2+Ax+1)=0$, so we get \[
  x=0,\frac{-A\pm\sqrt{A^2 - 4}}{2}. \]
%
Therefore, the set of rational points over the definition field
$F_{p^n}$ of a Montgomery curve includes at least two 2-torsion
points, namely, $\mathcal O$ and $(0,0)$.
%
The other 2-torsion points may or may not be rational, so we will
focus on $(0,0)$ in this paper.
%
Substituting $(x_2,y_2)=(0,0)$ into the addition formula for
Montgomery curves, we get that for any point $P=(x,y)\in M_{A,B}$,
$P+(0,0)=(1/x,-y/x^2)$.

To be able to exploit the symmetry of addition of $T_2=(0,0)$, we need
to choose the factor base
$\mathcal F=\{(x,y)\in E(\F{p^n}):x\in V\subset\F{p^n}\}$ invariant
under addition of $T_2$.
%
This means that $V$ needs to be closed under taking multiplicative
inverses.
%
In other words, $V$ needs to be a \emph{subfield} of $\F{p^n}$, i.e.,
$V=\F{p^\ell}$ for some integer $\ell$ that divides $n$.
%
In this case, $f_m$ is invariant under the action of
$x_i\mapsto 1/x_i$.
%
Unfortunately, such an action is not linear and hence not easy to
handle in polynomial system solving.
%
How to take advantage of such kind of symmetry in PDP is still an open
research problem.


%
% Hessian curves
%

\section{Hessian and other curves}
\label{sec:hessian}

\subsection{Hessian curves}
%
A Hessian curve $H_d$ over \F{p^n} for $p^n=2\bmod 3$ is defined by
the equation \begin{equation}
  x^3+y^3+1=3dxy \label{eq:hessian-curve} \end{equation} for
$d\in\F{p^n}$ such that $27d^3\neq 1$~\cite{DBLP:conf/ches/Smart01}.
%
For $P=(x,y)\in H_d$, $-P=(y,x)$.
%
Furthermore, the addition and doubling formulae for
$(x_3,y_3)=(x_1,y_1)+(x_2,y_2)$ are given as follows.
%
\begin{itemize}
\item When $(x_1,y_1)\neq(x_2,y_2)$:
  \begin{align*}
    x_3 & = \frac{y_1^2x_2 - y_2^2x_1}{x_2y_2 - x_1y_1} \\
    y_3 & = \frac{x_1^2y_2 - x_2^2y_1}{x_2y_2 - x_1y_1}
  \end{align*}
\item When $(x_1,y_1)=(x_2,y_2)$:
  \begin{align*}
    x_3 & = \frac{y_1(1 - x_1^3)}{x_1^3 - y_1^3} \\
    y_3 & = \frac{x_1(y_1^3 - 1)}{x_1^3 - y_1^3}
  \end{align*}
\end{itemize}

\subsection{Summation polynomials for Hessian curves}

Following a similar approach outlined by Galbraith and
Gebregiyorgis~\cite{DBLP:conf/indocrypt/GalbraithG14}, we can
construct summation polynomials for Hessian curves as follows.
%
First, we introduce a new variable $T=X+Y$, which is invariant under
negation of a point.
%
The 2nd summation polynomial for Hessian curves is simply
$f_{H,2} = T_1 - T_2$.
%
Now let
\[ Z=\left\{\begin{aligned}
      (x_1,y_1,t_1,&x_2,y_2,t_2,x_3,y_3,t_3)\in\F{p^n}^9:(x_i,y_i)\in H_d(\F{p^n}),i=1,2,3; \\
      & (x_1,y_1)+(x_2,y_2)+(x_3,y_3)=\mathcal O; x_i+y_i=t_i,i=1,2,3
    \end{aligned} \right\}. \]
%
Clearly, $Z$ is in the variety of the ideal
$I\subset\F{p^n}[X_1,Y_1,T_1,X_2,Y_2,T_2,X_3,Y_3,T_3]$ generated by
\[ \left\{\begin{aligned}
      &  (X_3 - X_1)(Y_2 - Y_1) - (X_2 - X_1)(Y_3 - Y_1), \\
      & X_i^3 + Y_i^3 + 1 - 3dX_iY_i,i=1,2,3, \\
      & X_i + Y_i - T_i,i=1,2,3
    \end{aligned}\right\}. \]
%
After removing the degenerate factors, we can obtain for Hessian curve
the 3rd summation polynomial:
\begin{align*}
  f_{H,3}(T_1,T_2,T_3) = & T_1^2T_2^2T_3 + dT_1^2T_2^2 + T_1^2T_2T_3^2
                           + dT_1^2T_2T_3 + dT_1^2T_3^2 - T_1^2 + \\
                         & T_1T_2^2T_3^2 + dT_1T_2^2T_3 + dT_1T_2T_3^2
                           + 3d^2T_1T_2T_3 + 2T_1T_2 + 2T_1T_3 + 2dT_1
                           + \\
                         & dT_2^2T_3^2 - T_2^2 + 2T_2T_3 + 2dT_2 - T_3^2 + 2dT_3 + 3d^2,
\end{align*}
%
as well as the subsequent summation polynomials via taking resultants:
\[ f_{H,m}(T_1,\ldots,T_m) =
  \res_T\left(f_{H,m-k}(T_1,\ldots,T_{m-k-1},T),f_{H,k+2}(T_{m-k},\ldots,T_m,T)\right). \]

A Hessian curve can contain an affine 2-torsion point $T_2$.
%
Since $T_2+T_2=2T_2=\mathcal O$, it follows that $-T_2=T_2$.
%
If we write $T_2=(x,y)$, then we can see that $x=y$ in order for
$-T_2=T_2$, as $-T_2=(y,x)$.
%
Substituting $x=y$ into Equation~(\ref{eq:hessian-curve}), we get an
equation $2x^3-3dx^2+1=0$.
%
Therefore, a Hessian curve $H_d(\F{p^n})$ has a 2-torsion point
$(\zeta,\zeta)$ if the polynomial $2X^3 - 3dX^2 + 1$ has a root
$\zeta$ in $\F{p^n}$.
%
In this case, the addition of this 2-torsion point to a point $(x,y)$
would give another point $(x',y')$ where
\[ \left\{\begin{aligned}
x' = & \frac{\zeta y^2 - \zeta^2x}{\zeta^2 - xy}, \\
y' = & \frac{\zeta x^2 - \zeta^2y}{\zeta^2 - xy}.
\end{aligned}\right. \]
%
Obviously, the factor base is not invariant under the action of this
2-torsion point in general.
%
Therefore, it is not clearly how to exploit such symmetry for Hessian
curves.

\subsection{(Twisted) Edwards and Weierstrass curves}

Faug\`ere, Gaudry, Hout, and Renault studied the point decomposition
problem on twisted Edwards, twisted Jacobi intersections, and
Weierstrass curves~\cite{DBLP:conf/eurocrypt/FaugereHJRV14}.
%
For the sake of completeness, we include some basic facts about
(twisted) Edwards curves here.
%
%
An Edwards curve over \F{p^n} for $p\neq 2$ is defined by the
equation \begin{equation*}
  x^2+y^2=1+dx^2y^2 \label{eq:edwards-curve} \end{equation*} for some
$d\in\F{p^n}$~\cite{DBLP:journals/iacr/BernsteinL07}.
%
A twisted Edwards curve $tE_{a',d'}$ over \F{p^n} for $p\neq 2$ is
defined by the equation \begin{equation}
  a'x^2+y^2=1+d'x^2y^2 \label{eq:twisted-edwards-curve} \end{equation}
for some $a',d'\in\F{p^n}$~\cite{DBLP:journals/iacr/BernsteinBJLP08}.
%
A twisted Edwards curve is a quadratic twist of an Edwards curve by
$a_0=1/(a'-d')$.
%
For $P=(x,y)\in H_d$, $-P=(-x,y)$.
%
Furthermore, the addition and doubling formulae for
$(x_3,y_3)=(x_1,y_1)+(x_2,y_2)$ are given as follows.
%
\begin{itemize}
\item When $(x_1,y_1)\neq(x_2,y_2)$:
  \begin{align*}
    x_3 & = \frac{x_1y_2 + y_1x_2}{1 + d'x_1x_2y_1y_2} \\
    y_3 & = \frac{y_1y_2 - a'x_1x_2}{1 - d'x_1x_2y_1y_2}
  \end{align*}
\item When $(x_1,y_1)=(x_2,y_2)$:
  \begin{align*}
    x_3 & = \frac{2x_1y_1}{1 + d'x_1^2y_1^2} \\
    y_3 & = \frac{y_1^2 - a'x_1^2}{1 - d'x_1^2y_1^2}
  \end{align*}
\end{itemize}
 
\section{Experiment}
\label{sec:experiment}

We have conducted a set of experiments to compare the difficulty of
solving ECDLP for four different curves: Hessian($H$), Weierstrass($W$),
Montgomery($M$), and twisted Edwards($tE$) over \F{p^n}.
%

\subsection{Experimental conditions}
\label{subsec:conditions}

%
% Isomorphisms among curves in various forms
%

% \section{Isomorphisms among curves in various forms}
% \label{sec:isomorphism}

%
To make a fair comparison, we use curves in different forms but are
nonetheless isomorphic to one another over \F{p^n}.
%
That is, $H(\F{p^n})\cong W(\F{p^n})\cong M(\F{p^n})\cong tE(\F{p^n})$
as groups, and we consider ECDLP in the same largest prime-order
subgroup.
%
We will also explicitly state whether the factor base is invariant
under addition of 2-torsion points for each of the four forms under
investigation.

We start from a Hessian curve $H_d$ satisfying $x^3 + y^3 + 1 = 3dxy$
for $d\in\F{p^n}$ such that the number of its rational points
$\#H_d(\F{p^n})$ is divisible by 12.
%
As we have seen in Section~\ref{sec:hessian-t2}, the factor base of
$H_d$ is in general not invariant under addition of 2-torsion points.
%
From $H_d$, we can obtain an isomorphic Weierstrass curve $W_{a,b}$
satisfying $y^2 = x^3 + ax + b$ for $a = - 27d(d^3 + 8)$ and
$b = 54(d^6 - 20d^3 - 8)$~\cite{DBLP:conf/ches/Smart01}.
%
The isomorphism $\phi_{W,H}$ from $W_{a,b}(\F{p^n})$ to $H_d(\F{p^n})$
is defined over $\F{p^n}$ and is given by sending $(u,v)\in W_{a,b}$
to $(x,y)\in H_d$, where
\[ \left\{\begin{aligned}
      x = & \frac{36(d^3 - 1) - v}{6(u + 9d^2)} - \frac{d}{2}, \\
      y = & \frac{36(d^3 - 1) + v}{6(u + 9d^2)} - \frac{d}{2}.
    \end{aligned}\right. \text{The inverse $\phi_{H,W}$ is given by }
  \left\{\begin{aligned}
      u = & \frac{12(d^3 - 1)}{d + x + y} - 9d^2, \\
      v = & \frac{36(d^3 - 1)(y - x)}{d + x + y}.
    \end{aligned}\right. \]
%
The factor base of $W_{a,b}$ is in general not invariant under
addition of 2-torsion points~\cite{DBLP:journals/joc/FaugereGHR14}.

With a high probability, we can obtain a Montgomery curve $M_{A,B}$
satisfying $By^2 = x^3 + Ax^2 + x$ from $W_{a,b}$ by solving the
following equations
%
\[ \left\{\begin{aligned}
a = & \frac{3 - A^2}{3B^2}, \\
b = & \frac{2A^3 - 9A}{27B^3}.
\end{aligned}\right. \]
%
The isomorphism $\phi_{W,M}$ is defined over $\F{p^n}$ and is given by
sending $(u,v)\in W_{a,b}$ to $(x,y)\in M_{A,B}$ for $x = Bu - 1/3A$
and $y = Bv$.
%
The inverse $\phi_{M,W}$ can be obtained by equation solving.
%
As we have seen in Section~\ref{sec:montgomery-symmetry}, the factor
base is invariant under addition of a particular 2-torsion point
$(0,0)$, though we are not able to exploit this symmetry in general.

Finally, we can obtain a twisted Edwards curve $tE_{a',d'}$ satisfying
$a'x^2 + y^2 = 1 + d'x^2y^2$ from $M_{A,B}$ by taking $a' = (A + 2)/B$
and $d' = (A - 2)/B$.
%
Again we let $a_0=1/(a' - d')$ be the amount of quadratic twist.
%
The isomorphism $\phi_{W,tE}$ is defined over $\F{p^n}$ and given by
sending $(u,v)\in W_{a,b}$ to $(x,y)\in tE_{a',d'}$, where
\[ \left\{\begin{aligned}
      x = & \frac{2a_0u}{v}, \\
      y = & \frac{u - a_0}{u + a_0}.
    \end{aligned}\right. \text{The inverse $\phi_{tE,W}$ is given by }
  \left\{\begin{aligned}
      u = & \frac{a_0(1 + y)}{1 - y}, \\
      v = & \frac{2a_0^2(1 + y)}{x(1 - y)}.
    \end{aligned}\right. \]
% 
As shown by Faug\`ere, Gaudry, Hout, and
Renault~\cite{DBLP:journals/joc/FaugereGHR14}, the factor base is
invariant under addition of the 2-torsion point $(0,-1)$.


As explained in Section~\ref{sec:index-calculus-ecdlp}, we focus on
the PDP computation in these experiments, as the other bottleneck
computation, the linear algebra step is already well understood.
%
We focus on the bottleneck computation in PDP, namely, the cost of the
F4 algorithm for computing Gr\"obner bases for polynomial systems
obtained after rewriting using the elementary symmetric polynomials
and applying the Weil restriction technique to summation polynomials.
%
This way we will be taking advantage of the symmetry of $S_m$ acting
on point decompositions.
% 
However, we \emph{did not} exploit the symmetry of group actions on
point decomposition except the action of $S_m$.
%
This is because we want to compare the \emph{intrinsic} computational
complexity of solving PDP for various curves and hence can only
consider the symmetry that is present in \emph{all} curves.
%
Exploiting further curve-specific symmetry whenever possible will
result in further speed-up, but it would be independent of our
findings here.

All our experiment are done using the MAGMA computation algebra system
(version 2.23-1) on a single core of an Intel Xeon CPU E7-4830 v4
running at 2~GHz.
%
The main cost metrics are running time, Matcost, and the maximum step
degree reached during the execution of the F4 algorithm.
%
The last metric is usually referred to as the ``degree of regularity''
in the literature~\cite{DBLP:conf/indocrypt/GalbraithG14}, which
provides an upper bound for the size of the Macaulay submatrices
involved in the process.
%
The ``Matcost'' is a number output by the MAGMA implementation of the
F4 algorithm and provides an estimate of the linear algebra cost
during the execution of the F4 algorithm.

Lastly, we also focus on ``rank'', which represents the number of 
linearly independent relations we can get once successfully 
solving a PDP instance for each of the four curves.
%
It is also important factor to consider complexity of solving PDP
because rank decide the number of solving PDP which we need 
during the whole sequence of index-calculus attacks.


\subsection{Experimental results}

We present our experimental results for the case of $n=5$.
%
Here our factor base is to take $V$ as the base field \F p of \F{p^n}.
%

\begin{table}[!h]
\centering
\caption{$m=3$}
\label{tb:m=3}
\begin{tabular}{llrrrr}
\hline
\multicolumn{1}{|c|}{$q$}                    & \multicolumn{1}{l|}{Curve}       & \multicolumn{1}{l|}{Time} & \multicolumn{1}{l|}{Dreg} & \multicolumn{1}{l|}{Matcost} & \multicolumn{1}{l|}{Rank} \\ \hline
\multicolumn{1}{|l|}{\multirow{4}{*}{251}} & \multicolumn{1}{l|}{Hessian}     & \multicolumn{1}{r|}{0}    & \multicolumn{1}{r|}{6}    & \multicolumn{1}{r|}{41420.4} & \multicolumn{1}{r|}{1}    \\ \cline{2-6} 
\multicolumn{1}{|l|}{}                     & \multicolumn{1}{l|}{Weierstrass} & \multicolumn{1}{r|}{0}    & \multicolumn{1}{r|}{6}    & \multicolumn{1}{r|}{42132.0} & \multicolumn{1}{r|}{1}    \\ \cline{2-6} 
\multicolumn{1}{|l|}{}                     & \multicolumn{1}{l|}{Montgomery}  & \multicolumn{1}{r|}{0}    & \multicolumn{1}{r|}{6}    & \multicolumn{1}{r|}{61127.9} & \multicolumn{1}{r|}{4}    \\ \cline{2-6} 
\multicolumn{1}{|l|}{}                     & \multicolumn{1}{l|}{tEdwards}    & \multicolumn{1}{r|}{0}    & \multicolumn{1}{r|}{6}    & \multicolumn{1}{r|}{6308.4}  & \multicolumn{1}{r|}{4}    \\ \hline \vspace{-3mm}
                                           &                                  & \multicolumn{1}{l}{}      & \multicolumn{1}{l}{}      & \multicolumn{1}{l}{}         & \multicolumn{1}{l}{}      \\ \hline
\multicolumn{1}{|l|}{\multirow{4}{*}{239}} & \multicolumn{1}{l|}{Hessian}     & \multicolumn{1}{r|}{0}    & \multicolumn{1}{r|}{6}    & \multicolumn{1}{r|}{42336.8} & \multicolumn{1}{r|}{1}    \\ \cline{2-6} 
\multicolumn{1}{|l|}{}                     & \multicolumn{1}{l|}{Weierstrass} & \multicolumn{1}{r|}{0}    & \multicolumn{1}{r|}{6}    & \multicolumn{1}{r|}{41259}   & \multicolumn{1}{r|}{1}    \\ \cline{2-6} 
\multicolumn{1}{|l|}{}                     & \multicolumn{1}{l|}{Montgomery}  & \multicolumn{1}{r|}{0}    & \multicolumn{1}{r|}{6}    & \multicolumn{1}{r|}{61239}   & \multicolumn{1}{r|}{4}    \\ \cline{2-6} 
\multicolumn{1}{|l|}{}                     & \multicolumn{1}{l|}{tEdwards}    & \multicolumn{1}{r|}{0}    & \multicolumn{1}{r|}{6}    & \multicolumn{1}{r|}{6308.36} & \multicolumn{1}{r|}{4}    \\ \hline
\end{tabular}
\end{table}

\begin{table}[!h]
\centering
\caption{$m=4$}
\label{tb:m=4}
\begin{tabular}{clrrrr}
\hline
\multicolumn{1}{|c|}{$q$}                  & \multicolumn{1}{l|}{Curve}       & \multicolumn{1}{l|}{Time}  & \multicolumn{1}{l|}{Dreg} & \multicolumn{1}{l|}{Matcost}     & \multicolumn{1}{l|}{Rank} \\ \hline
\multicolumn{1}{|l|}{\multirow{4}{*}{251}} & \multicolumn{1}{l|}{Hessian}     & \multicolumn{1}{r|}{3.459} & \multicolumn{1}{r|}{19}   & \multicolumn{1}{r|}{12069800000} & \multicolumn{1}{r|}{1}    \\ \cline{2-6} 
\multicolumn{1}{|l|}{}                     & \multicolumn{1}{l|}{Weierstrass} & \multicolumn{1}{r|}{3.659} & \multicolumn{1}{r|}{19}   & \multicolumn{1}{r|}{12066400000} & \multicolumn{1}{r|}{1}    \\ \cline{2-6} 
\multicolumn{1}{|l|}{}                     & \multicolumn{1}{l|}{Montgomery}  & \multicolumn{1}{r|}{3.280} & \multicolumn{1}{r|}{18}   & \multicolumn{1}{r|}{11401700000} & \multicolumn{1}{r|}{5}    \\ \cline{2-6} 
\multicolumn{1}{|l|}{}                     & \multicolumn{1}{l|}{tEdwards}    & \multicolumn{1}{r|}{0.119} & \multicolumn{1}{r|}{18}   & \multicolumn{1}{r|}{54102900}    & \multicolumn{1}{r|}{5}    \\ \hline
 \vspace{-3mm}                                          &                                  & \multicolumn{1}{l}{}       & \multicolumn{1}{l}{}      & \multicolumn{1}{l}{}             & \multicolumn{1}{l}{}      \\ \hline
\multicolumn{1}{|l|}{\multirow{4}{*}{239}} & \multicolumn{1}{l|}{Hessian}     & \multicolumn{1}{r|}{3.990} & \multicolumn{1}{r|}{19}   & \multicolumn{1}{r|}{12066100000} & \multicolumn{1}{r|}{1}    \\ \cline{2-6} 
\multicolumn{1}{|l|}{}                     & \multicolumn{1}{l|}{Weierstrass} & \multicolumn{1}{r|}{3.680} & \multicolumn{1}{r|}{19}   & \multicolumn{1}{r|}{12064700000} & \multicolumn{1}{r|}{1}    \\ \cline{2-6} 
\multicolumn{1}{|l|}{}                     & \multicolumn{1}{l|}{Montgomery}  & \multicolumn{1}{r|}{3.489} & \multicolumn{1}{r|}{18}   & \multicolumn{1}{r|}{11399100000} & \multicolumn{1}{r|}{5}    \\ \cline{2-6} 
\multicolumn{1}{|l|}{}                     & \multicolumn{1}{l|}{tEdwards}    & \multicolumn{1}{r|}{0.150} & \multicolumn{1}{r|}{18}   & \multicolumn{1}{r|}{54093000}    & \multicolumn{1}{r|}{5}    \\ \hline
\end{tabular}
\end{table}


% \begin{table}[!h]
% \centering
% \caption{$m=3$}
% \label{tb:m=3}
% \begin{tabular}{llrrr}
% \hline
% \multicolumn{1}{|l|}{$q$}                  & \multicolumn{1}{l|}{Curve}       & \multicolumn{1}{l|}{Time} & \multicolumn{1}{l|}{Dreg} & \multicolumn{1}{l|}{Matcost} \\ \hline
% \multicolumn{1}{|l|}{\multirow{4}{*}{251}} & \multicolumn{1}{l|}{Hessian}     & \multicolumn{1}{r|}{0}    & \multicolumn{1}{r|}{6}    & \multicolumn{1}{r|}{41420.4} \\ \cline{2-5} 
% \multicolumn{1}{|l|}{}                     & \multicolumn{1}{l|}{Weierstrass} & \multicolumn{1}{r|}{0}    & \multicolumn{1}{r|}{6}    & \multicolumn{1}{r|}{42132.0} \\ \cline{2-5} 
% \multicolumn{1}{|l|}{}                     & \multicolumn{1}{l|}{Montgomery}  & \multicolumn{1}{r|}{0}    & \multicolumn{1}{r|}{6}    & \multicolumn{1}{r|}{61127.9} \\ \cline{2-5} 
% \multicolumn{1}{|l|}{}                     & \multicolumn{1}{l|}{tEdwards}    & \multicolumn{1}{r|}{0}    & \multicolumn{1}{r|}{6}    & \multicolumn{1}{r|}{6308.4}  \\ \hline \vspace{-3mm}
%                                            &                                  &                           &                           &                              \\ \hline
% \multicolumn{1}{|l|}{\multirow{4}{*}{239}} & \multicolumn{1}{l|}{Hessian}     & \multicolumn{1}{r|}{0}    & \multicolumn{1}{r|}{6}    & \multicolumn{1}{r|}{42336.8} \\ \cline{2-5} 
% \multicolumn{1}{|l|}{}                     & \multicolumn{1}{l|}{Weierstrass} & \multicolumn{1}{r|}{0}    & \multicolumn{1}{r|}{6}    & \multicolumn{1}{r|}{41259.0} \\ \cline{2-5} 
% \multicolumn{1}{|l|}{}                     & \multicolumn{1}{l|}{Montgomery}  & \multicolumn{1}{r|}{0}    & \multicolumn{1}{r|}{6}    & \multicolumn{1}{r|}{61239.0} \\ \cline{2-5} 
% \multicolumn{1}{|l|}{}                     & \multicolumn{1}{l|}{tEdwards}    & \multicolumn{1}{r|}{0}    & \multicolumn{1}{r|}{6}    & \multicolumn{1}{r|}{6308.4}  \\ \hline
% \end{tabular}
% \end{table}


% \begin{table}[!h]
% \centering
% \caption{$m=4$}
% \label{tb:m=4}
% \begin{tabular}{llrrr}
% \hline
% \multicolumn{1}{|l|}{$q$}                  & \multicolumn{1}{l|}{Curve}       & \multicolumn{1}{l|}{Time}  & \multicolumn{1}{l|}{Dreg} & \multicolumn{1}{l|}{Matcost}     \\ \hline
% \multicolumn{1}{|l|}{\multirow{4}{*}{251}} & \multicolumn{1}{l|}{Hessian}     & \multicolumn{1}{r|}{3.459} & \multicolumn{1}{r|}{19}   & \multicolumn{1}{r|}{12069800000} \\ \cline{2-5} 
% \multicolumn{1}{|l|}{}                     & \multicolumn{1}{l|}{Weierstrass} & \multicolumn{1}{r|}{3.659} & \multicolumn{1}{r|}{19}   & \multicolumn{1}{r|}{12066400000} \\ \cline{2-5} 
% \multicolumn{1}{|l|}{}                     & \multicolumn{1}{l|}{Montgomery}  & \multicolumn{1}{r|}{3.280} & \multicolumn{1}{r|}{18}   & \multicolumn{1}{r|}{11401700000} \\ \cline{2-5} 
% \multicolumn{1}{|l|}{}                     & \multicolumn{1}{l|}{tEdwards}    & \multicolumn{1}{r|}{0.119} & \multicolumn{1}{r|}{18}   & \multicolumn{1}{r|}{54102900}    \\ \hline \vspace{-3mm}
%                                            &                                  &                            &                           &                                  \\ \hline
% \multicolumn{1}{|l|}{\multirow{4}{*}{239}} & \multicolumn{1}{l|}{Hessian}     & \multicolumn{1}{r|}{3.990} & \multicolumn{1}{r|}{19}   & \multicolumn{1}{r|}{12066100000} \\ \cline{2-5} 
% \multicolumn{1}{|l|}{}                     & \multicolumn{1}{l|}{Weierstrass} & \multicolumn{1}{r|}{3.680} & \multicolumn{1}{r|}{19}   & \multicolumn{1}{r|}{12064700000} \\ \cline{2-5} 
% \multicolumn{1}{|l|}{}                     & \multicolumn{1}{l|}{Montgomery}  & \multicolumn{1}{r|}{3.489} & \multicolumn{1}{r|}{18}   & \multicolumn{1}{r|}{11399100000} \\ \cline{2-5} 
% \multicolumn{1}{|l|}{}                     & \multicolumn{1}{l|}{tEdwards}    & \multicolumn{1}{r|}{0.150} & \multicolumn{1}{r|}{18}   & \multicolumn{1}{r|}{54093000}    \\ \hline
% \end{tabular}
% \end{table}

We can clearly see that the PDP solving time and matcost for twisted 
Edwards curve is much faster and smaller than the other three curves 
for different $q$ and $m$.
%
In terms of Dreg, degree for Montgomery and twisted Edwards curve are 
smaller than other curves in the case of $m=4$.
%
Also, we can see that Rank for Hessian and Weierstrass curve is 
1 in all cases.
%
On the other hand, the Rank for Montgomery and twisted Edwards curve
is larger, that is 4 and 5 respectively in the case of $m=3$ and 4.
%
% We further analyze which terms are missing from the summation
% polynomials for curves in different forms.
% %
% We classify terms into odd vs.~even degrees.
% %
% Recall that the polynomials are expressed using elementary symmetric polynomials 
% $e_1, \dots ,e_m$.
% %
% Therefore, a monomial $e_i^j$ has odd degree only if both $i$ and $j$
% are odd.
% %
% For $m=2$, it is interesting that there are no monomials of odd degree
% only for twisted Edwards curve.
% %
% In other words, there are no terms originated from monomial such as
% $e_1$ or $e_1e_2$.
% %
% This is because the addition formula for twisted Edwards curve, based
% on which the summation polynomial is derived, consists of only even
% terms.
% %
% Therefore, when one of the 3 variables in the summation polynomial is
% substituted with the point we want to decompose, the odd-degree terms
% all vanish.

% For $m=3$, we use summation polynomial with 4 variables, which is
% obtained by taking the resultant of two summation polynomials 
% with 3 variables.
% %
% Therefore, there are some odd terms such as $e_3$ or $e_2^2e_3$.
% %
% For $m=4$, there is also no terms with odd degree.
% %
% Through calculating resultant recursively, two variables are
% eliminated, and it preserves all terms to even degree.








\section{Analysis, discussion, and concluding remarks}
\label{sec:analysis}

\subsection{What makes PDP on (twisted) Edwards curves easier to
  solve?}

As we have seen in Section~\ref{sec:experiment-result}, PDP on
(twisted) Edwards curves seems easier to solve than on other curves.
%
The explanation offered by Faug\`ere, Gaudry, Hout, and Renault is
``due to the smaller degree appearing in the computation of Gr\"obner
basis of $\mathscr S_{D_n}$ in comparison with the Weierstrass case,''
cf.~Section~4.1.1 of their
paper~\cite{DBLP:journals/joc/FaugereGHR14}.
%
Unfortunately, this \emph{cannot} explain the difference in solving
time between (twisted) Edwards and Montgomery forms, as the highest
degrees appearing in the computation of Gr\"obner bases are \emph{the
  same} for these two forms.
%
Therefore, there must be other reasons, which we will explore further
in this section.
%
Table~\ref{tb:terms} shows the sparsity before and after Weil
restriction for each of the four curves under investigation in the
case of $m=2,3,4$.
%
As there are $n$ equations after Weil restriction, we are only showing
the average numbers.
%
We also include the maximum number of terms for a polynomial of degree
$2^{(m+1)-2}$ in $m$ variables.
%

\begin{table}[!h]
\centering
\caption{number of terms (experimental/theoretical) in polynomial systems before and after Weil decent}
\label{tb:terms}
\begin{tabular}{llllllll}
\hline
\multicolumn{1}{|l|}{\multirow{2}{*}{$m$}} & \multicolumn{1}{l|}{\multirow{2}{*}{Curve}} & \multicolumn{3}{l|}{before Weil decent}                                                    & \multicolumn{3}{l|}{after Weil decent}                                                           \\ \cline{3-8} 
\multicolumn{1}{|l|}{}                     & \multicolumn{1}{l|}{}                       & \multicolumn{1}{l|}{total}   & \multicolumn{1}{l|}{odd}     & \multicolumn{1}{l|}{even}    & \multicolumn{1}{l|}{total}     & \multicolumn{1}{l|}{odd}       & \multicolumn{1}{l|}{even}      \\ \hline
\multicolumn{1}{|l|}{\multirow{4}{*}{2}}   & \multicolumn{1}{l|}{Hessian}                & \multicolumn{1}{l|}{6/6}     & \multicolumn{1}{l|}{2/2}     & \multicolumn{1}{l|}{4/4}     & \multicolumn{1}{l|}{5.2/6}     & \multicolumn{1}{l|}{2/2}       & \multicolumn{1}{l|}{3.2/4}     \\ \cline{2-8} 
\multicolumn{1}{|l|}{}                     & \multicolumn{1}{l|}{Weierstrass}            & \multicolumn{1}{l|}{6/6}     & \multicolumn{1}{l|}{2/2}     & \multicolumn{1}{l|}{4/4}     & \multicolumn{1}{l|}{5.2/6}     & \multicolumn{1}{l|}{2/2}       & \multicolumn{1}{l|}{3.2/4}     \\ \cline{2-8} 
\multicolumn{1}{|l|}{}                     & \multicolumn{1}{l|}{Montgomery}             & \multicolumn{1}{l|}{6/6}     & \multicolumn{1}{l|}{2/2}     & \multicolumn{1}{l|}{4/4}     & \multicolumn{1}{l|}{5.2/6}     & \multicolumn{1}{l|}{2/2}       & \multicolumn{1}{l|}{3.2/4}     \\ \cline{2-8} 
\multicolumn{1}{|l|}{}                     & \multicolumn{1}{l|}{tEdwards}               & \multicolumn{1}{l|}{4/6}     & \multicolumn{1}{l|}{0/2}     & \multicolumn{1}{l|}{4/4}     & \multicolumn{1}{l|}{3.2/6}     & \multicolumn{1}{l|}{0/2}       & \multicolumn{1}{l|}{3.2/4}     \\ \hline \vspace{-3mm}
                                           &                                             &                              &                              &                              &                                &                                &                                \\ \hline
\multicolumn{1}{|l|}{\multirow{4}{*}{3}}   & \multicolumn{1}{l|}{Hessian}                & \multicolumn{1}{l|}{35/35}   & \multicolumn{1}{l|}{16/16}   & \multicolumn{1}{l|}{19/19}   & \multicolumn{1}{l|}{34.2/35}   & \multicolumn{1}{l|}{16/16}     & \multicolumn{1}{l|}{18.2/19}   \\ \cline{2-8} 
\multicolumn{1}{|l|}{}                     & \multicolumn{1}{l|}{Weierstrass}            & \multicolumn{1}{l|}{35/35}   & \multicolumn{1}{l|}{16/16}   & \multicolumn{1}{l|}{19/19}   & \multicolumn{1}{l|}{34/35}     & \multicolumn{1}{l|}{16/16}     & \multicolumn{1}{l|}{18/19}     \\ \cline{2-8} 
\multicolumn{1}{|l|}{}                     & \multicolumn{1}{l|}{Montgomery}             & \multicolumn{1}{l|}{35/35}   & \multicolumn{1}{l|}{16/16}   & \multicolumn{1}{l|}{19/19}   & \multicolumn{1}{l|}{33.4/35}   & \multicolumn{1}{l|}{16/16}     & \multicolumn{1}{l|}{17.4/19}   \\ \cline{2-8} 
\multicolumn{1}{|l|}{}                     & \multicolumn{1}{l|}{tEdwards}               & \multicolumn{1}{l|}{25/35}   & \multicolumn{1}{l|}{6/16}    & \multicolumn{1}{l|}{19/19}   & \multicolumn{1}{l|}{23.4/35}   & \multicolumn{1}{l|}{6/16}      & \multicolumn{1}{l|}{17.4/19}   \\ \hline  \vspace{-3mm}
                                           &                                             &                              &                              &                              &                                &                                &                                \\ \hline
\multicolumn{1}{|l|}{\multirow{4}{*}{4}}   & \multicolumn{1}{l|}{Hessian}                & \multicolumn{1}{l|}{495/495} & \multicolumn{1}{l|}{240/240} & \multicolumn{1}{l|}{255/255} & \multicolumn{1}{l|}{493.2/495} & \multicolumn{1}{l|}{239.4/240} & \multicolumn{1}{l|}{253.8/255} \\ \cline{2-8} 
\multicolumn{1}{|l|}{}                     & \multicolumn{1}{l|}{Weierstrass}            & \multicolumn{1}{l|}{495/495} & \multicolumn{1}{l|}{240/240} & \multicolumn{1}{l|}{255/255} & \multicolumn{1}{l|}{492/495}   & \multicolumn{1}{l|}{238.4/240} & \multicolumn{1}{l|}{253.6/255} \\ \cline{2-8} 
\multicolumn{1}{|l|}{}                     & \multicolumn{1}{l|}{Montgomery}             & \multicolumn{1}{l|}{495/495} & \multicolumn{1}{l|}{240/240} & \multicolumn{1}{l|}{255/255} & \multicolumn{1}{l|}{492.2/495} & \multicolumn{1}{l|}{239.2/240} & \multicolumn{1}{l|}{253/255}   \\ \cline{2-8} 
\multicolumn{1}{|l|}{}                     & \multicolumn{1}{l|}{tEdwards}               & \multicolumn{1}{l|}{255/495} & \multicolumn{1}{l|}{0/240}   & \multicolumn{1}{l|}{255/255} & \multicolumn{1}{l|}{253/495}   & \multicolumn{1}{l|}{0/240}     & \multicolumn{1}{l|}{253/255}   \\ \hline
\end{tabular}
\end{table}


% \begin{table}[!h]
% \centering
% \caption{\#terms in polynomial systems before/after Weil decent}
% \label{tb:terms}
% \begin{tabular}{llllllll}
% \hline
% \multicolumn{1}{|l|}{m}                  & \multicolumn{1}{l|}{Curve}       & \multicolumn{1}{l|}{\begin{tabular}[c]{@{}l@{}}before\\ Weil decent\end{tabular}} & \multicolumn{1}{l|}{odd}     & \multicolumn{1}{l|}{even}    & \multicolumn{1}{l|}{\begin{tabular}[c]{@{}l@{}}after \\ Weil decent\end{tabular}} & \multicolumn{1}{l|}{odd}           & \multicolumn{1}{l|}{even}            \\ \hline
% \multicolumn{1}{|l|}{\multirow{4}{*}{2}} & \multicolumn{1}{l|}{Hessian}     & \multicolumn{1}{l|}{6/6}                                                          & \multicolumn{1}{l|}{2/2}     & \multicolumn{1}{l|}{4/4}     & \multicolumn{1}{l|}{58/65}                                                        & \multicolumn{1}{l|}{10/10}         & \multicolumn{1}{l|}{48/55}           \\ \cline{2-8} 
% \multicolumn{1}{|l|}{}                   & \multicolumn{1}{l|}{Weierstrass} & \multicolumn{1}{l|}{6/6}                                                          & \multicolumn{1}{l|}{2/2}     & \multicolumn{1}{l|}{4/4}     & \multicolumn{1}{l|}{58/65}                                                        & \multicolumn{1}{l|}{10/10}         & \multicolumn{1}{l|}{48/55}           \\ \cline{2-8} 
% \multicolumn{1}{|l|}{}                   & \multicolumn{1}{l|}{Montgomery}  & \multicolumn{1}{l|}{6/6}                                                          & \multicolumn{1}{l|}{2/2}     & \multicolumn{1}{l|}{4/4}     & \multicolumn{1}{l|}{58/65}                                                        & \multicolumn{1}{l|}{10/10}         & \multicolumn{1}{l|}{48/55}           \\ \cline{2-8} 
% \multicolumn{1}{|l|}{}                   & \multicolumn{1}{l|}{tEdwards}    & \multicolumn{1}{l|}{4/6}                                                          & \multicolumn{1}{l|}{0/2}     & \multicolumn{1}{l|}{4/4}     & \multicolumn{1}{l|}{28/65}                                                        & \multicolumn{1}{l|}{5/10}          & \multicolumn{1}{l|}{23/55}           \\ \hline \vspace{-5mm}
%                                          &                                  &                                                                                   &                              &                              &                                                                                   &                                    &                                      \\ \hline
% \multicolumn{1}{|l|}{\multirow{4}{*}{3}} & \multicolumn{1}{l|}{Hessian}     & \multicolumn{1}{l|}{35/35}                                                        & \multicolumn{1}{l|}{16/16}   & \multicolumn{1}{l|}{19/19}   & \multicolumn{1}{l|}{3856/3875}                                                    & \multicolumn{1}{l|}{695/695}       & \multicolumn{1}{l|}{3161/3180}       \\ \cline{2-8} 
% \multicolumn{1}{|l|}{}                   & \multicolumn{1}{l|}{Weierstrass} & \multicolumn{1}{l|}{35/35}                                                        & \multicolumn{1}{l|}{16/16}   & \multicolumn{1}{l|}{19/19}   & \multicolumn{1}{l|}{3854/3875}                                                    & \multicolumn{1}{l|}{694/695}       & \multicolumn{1}{l|}{3160/3180}       \\ \cline{2-8} 
% \multicolumn{1}{|l|}{}                   & \multicolumn{1}{l|}{Montgomery}  & \multicolumn{1}{l|}{35/35}                                                        & \multicolumn{1}{l|}{16/16}   & \multicolumn{1}{l|}{19/19}   & \multicolumn{1}{l|}{3838/3875}                                                    & \multicolumn{1}{l|}{695/695}       & \multicolumn{1}{l|}{3143/3180}       \\ \cline{2-8} 
% \multicolumn{1}{|l|}{}                   & \multicolumn{1}{l|}{tEdwards}    & \multicolumn{1}{l|}{25/35}                                                        & \multicolumn{1}{l|}{6/16}    & \multicolumn{1}{l|}{19/19}   & \multicolumn{1}{l|}{2235/3875}                                                    & \multicolumn{1}{l|}{550/695}       & \multicolumn{1}{l|}{1685/3180}       \\ \hline \vspace{-5mm}
%                                          &                                  &                                                                                   &                              &                              &                                                                                   &                                    &                                      \\ \hline
% \multicolumn{1}{|l|}{\multirow{4}{*}{4}} & \multicolumn{1}{l|}{Hessian}     & \multicolumn{1}{l|}{495/495}                                                      & \multicolumn{1}{l|}{240/240} & \multicolumn{1}{l|}{255/255} & \multicolumn{1}{l|}{3095460/3108104}                                              & \multicolumn{1}{l|}{700485/701864} & \multicolumn{1}{l|}{2394975/2406240} \\ \cline{2-8} 
% \multicolumn{1}{|l|}{}                   & \multicolumn{1}{l|}{Weierstrass} & \multicolumn{1}{l|}{495/495}                                                      & \multicolumn{1}{l|}{240/240} & \multicolumn{1}{l|}{255/255} & \multicolumn{1}{l|}{3090557/3108104}                                              & \multicolumn{1}{l|}{697526/701864} & \multicolumn{1}{l|}{2393031/2406240} \\ \cline{2-8} 
% \multicolumn{1}{|l|}{}                   & \multicolumn{1}{l|}{Montgomery}  & \multicolumn{1}{l|}{495/495}                                                      & \multicolumn{1}{l|}{240/240} & \multicolumn{1}{l|}{255/255} & \multicolumn{1}{l|}{3101711/3108104}                                              & \multicolumn{1}{l|}{699180/701864} & \multicolumn{1}{l|}{2402531/2406240} \\ \cline{2-8} 
% \multicolumn{1}{|l|}{}                   & \multicolumn{1}{l|}{tEdwards}    & \multicolumn{1}{l|}{255/495}                                                      & \multicolumn{1}{l|}{0/240}   & \multicolumn{1}{l|}{255/255} & \multicolumn{1}{l|}{1548859/3108104}                                              & \multicolumn{1}{l|}{350847/701864} & \multicolumn{1}{l|}{1198012/2406240} \\ \hline
% \end{tabular}
% \end{table}

We can see that total number of terms for twisted Edwards curves is
significantly fewer than that for the other curves in all cases.
%
Naturally, this could lead to smaller matrices and hence faster
solving time with the F4 algorithm.
%
We also note that, except for twisted Edwards curves, the summation
polynomials for the other curves are all 100\% dense without any
missing terms.

% We further analyze which terms are missing from the summation
% polynomials for curves in different forms.
% %
% We classify terms into odd vs.~even degrees.
% %
% From Table~\ref{tb:terms}, missing of only odd terms on twisted Edwards
% curve reduce the number of terms.

%
% (twisted) Edwards curve
%

%------------------------------
\subsection{(Twisted) Edwards curves}
%------------------------------------
\label{sec:twisted-edwards}


Faug\`ere, Gaudry, Hout, and Renault studied PDP on twisted Edwards,
twisted Jacobi intersections, and Weierstrass
curves~\cite{DBLP:journals/joc/FaugereGHR14}.
%
For the sake of completeness, we include some basic facts about
(twisted) Edwards curves here.
%
%
An Edwards curve over \F{p^n} for $p\neq 2$ is defined by the
equation \begin{equation*}
  x^2+y^2=1+dx^2y^2 \label{eq:edwards-curve} \end{equation*} for
$d\in\F{p^n}$~\cite{DBLP:journals/iacr/BernsteinL07}.
%
A twisted Edwards curve $tE_{a',d'}$ over \F{p^n} for $p\neq 2$ is
defined by the equation \begin{equation}
  a'x^2+y^2=1+d'x^2y^2 \label{eq:twisted-edwards-curve} \end{equation}
for $a',d'\in\F{p^n}$~\cite{DBLP:journals/iacr/BernsteinBJLP08}.
%
A twisted Edwards curve is a quadratic twist of an Edwards curve by
$a_0=1/(a'-d')$.
%
For $P=(x,y)\in tE_{a',d'}$, $-P=(-x,y)$.
%
Furthermore, the addition and doubling formulae for
$(x_3,y_3)=(x_1,y_1)+(x_2,y_2)$ are given as follows.
%
\begin{itemize}
\item When $(x_1,y_1)\neq(x_2,y_2)$:
  \begin{align*}
    x_3 & = \frac{x_1y_2 + y_1x_2}{1 + d'x_1x_2y_1y_2} \\
    y_3 & = \frac{y_1y_2 - a'x_1x_2}{1 - d'x_1x_2y_1y_2}
  \end{align*}
\item When $(x_1,y_1)=(x_2,y_2)$:
  \begin{align*}
    x_3 & = \frac{2x_1y_1}{1 + d'x_1^2y_1^2} \\
    y_3 & = \frac{y_1^2 - a'x_1^2}{1 - d'x_1^2y_1^2}
  \end{align*}
\end{itemize}
%
As given by Faug\`ere, Gaudry, Hout, and
Renault~\cite{DBLP:journals/joc/FaugereGHR14}, the 3rd summation
polynomial for twisted Edwards curve is
%
\begin{align*}
  f_{tE, 3}(Y_1,Y_2,Y_3) = & \left(Y_1^2Y_2^2 - Y_1^2 - Y_2^2 + \frac{a}{d}\right)Y_3^2  + 2\frac{d-a}{d}Y_1Y_2Y_3 +
                             \frac{a}{d}\left(Y_1^2 + Y_2^2 - 1\right)
                             - Y_1^2Y_2^2.
\end{align*}
%
As usual, subsequent summation polynomials can be obtained by using
resultants.



\section{Observations of summation polynomial on twisted Edwards
  curve}
\label{sec:twisted-edwards-summation-polynomial}

We classify variables and monomials in from the viewpoint of parity. 
%
We refer to variable whose exponent is even number and odd number 
as respectively \emph{even variable} and \emph{odd variable}
%
Similarly, we refer to monomial whose degree is even number and odd number 
as respectively \emph{even monomial} and \emph{odd monoimal}


\begin{description}
  \item [Lemma 1]~\\
  %
  Considering resultant of two multivariate polynomial $a_1$ and $a_2$,
  you will consider the polynomials as univariate polynomial of the variable
  to eliminate $T$, such as $f_1 = a_mT^m + \dots +a_1T^1 + a_0$ 
  and $f_2 = b_nT^n + \dots +b_1T^1 + b_0$.
  %
  If degree of both of $m$ and $n$ is even, 
  then coefficients of odd monomial, $a_{2k+1}$ and $b_{2k+1}$ for integer $k$,
  is contained even number times in the term of resultant.
  %
\end{description}

\noindent
\emph{Proof.}
%
Resultant of $a_1$ and $a_2$ is 
determinant of following square matrix X of size $m+n$.
%
\begin{equation*}
X =
\begin{bmatrix}
  a_m & a_{m-1} & \ldots & & a_0 & & &  \\
          & a_m & a_{m-1} & \ldots & & a_0  & &  \\
  & & \ddots & & & & \ddots &  \\
    & & & a_m & a_{m-1} & \ldots & & a_0  \\
  b_n & b_{n-1} & \ldots & & b_0 & & &  \\
       & b_n & b_{n-1} & \ldots & & b_0  & &  \\
  & & \ddots & & & & \ddots &  \\
    & & & b_n & b_{n-1} & \ldots & & b_0
\end{bmatrix}
\end{equation*}

Let components in the $i$-th row and $j$-th column in the matrix $X$ 
be represented to $x_{ij}$. 
%
From assumption, $a_m$ and $b_n$ is the coefficients of even monomial
and the number of row which include $a_i$ and $b_i$ is both even.
%
Consequently, coefficients of odd monomials, $a_{2k+1}$ and $b_{2k+1}$ 
for integer $k$, correspond components whose sum of index (i+j) is odd.
%
Also, coefficients of even monomials, $a_{2k}$ and $b_{2k}$ for 
integer $k$, correspond components whose sum of index is even.

Here, determinant of $X$ is defined as
%
\begin{align*}
  %
  \mbox{det}(X) = \sum_{\sigma \in S_{n+m}} \mbox{sgn}(\sigma)x_{1\sigma(1)}
                   x_{2\sigma(2)}\cdots x_{m+n\sigma(m+n)}
\end{align*}
%
Under any permutation, total summation of sum of index remain even number,
that is $2*(1+ \cdots +n)$.
%
Hence, components whose sum of index is odd is contained even number times.
%
Therefore, coefficients of odd monoials is contained even number times
in the term of resultant.
\qed \\

We focus on each variable which is consisted term
is whether even or odd variable.
%
We reffer to the term which consist of only the same parity of variables
as \emph{well-regulated term}.


\begin{description}
  %
  \item [Lemma 2]~\\
  %
  If all terms in 3rd summation polynomial are well-regulated terms,
  all terms in summation polynomial with any variables are also
  well-regulated.
  %
\end{description}

\noindent
\emph{Proof.}
%
We prove that summation polynomial $f_{n}$ for $n\geq3$ satisfy the
proposion by mathematical inducution.

Let $n=3$, from assumption, summation polynomial $f_3(Y_1, Y_2, Y_3)$
satisfy proposion.
%
\begin{align*}
  %
  f_{3}(Y_1, Y_2, Y_3)=a_2Y_3^2 + a_1Y_3 + a_0
  %
\end{align*}
%

Let $n=k$, we assume that $f_{k}(Y_1, \dots , Y_{k})$ satisfy the proposion.
% 
we can express the polynomial as univariate polynomial of $Y_{k}$ 
with $2^{k-2}$ degree. 
%
\begin{align*}
  %
  f_{k}(Y_1, \dots , Y_{k}) & = 
  b_{2^{k-2}}Y_{k}^{2^{k-2}} + b_{2^{k-2}-1}Y_{k}^{2^{k-2}-1} + 
   \cdots + b_2Y_{k}^2 + b_1Y_{k} + b_0
  %
\end{align*}
%
In other words, we assume polynomial $b_{2k}$ consist of only even variables 
and $b_{2k+1}$ consist of only odd variables for integer $k$.

Let $n=k+1$, we will consider about $f_{k+1}(Y_1, \dots , Y_{k}, Y_{k+1})$.
%
From the definition of summation polynomial, we can express the polynomial
by using above coefficient $a_i$ and $b_i$.
%
\begin{align*}
  %
  f_{k+1}(Y_1, \dots , Y_{k}, Y_{k+1}) &= 
   Res\left(f_{k}(Y_1, \dots , Y_{k-1}, T), f_3(Y_{k}, Y_{k+1}, T)\right)\\
  %
   & = Res(b_{2^{k-2}}T^{2^{k-2}} + \cdots + b_1T + b_0,~ a_2T^2 + a_1T + a_0)
  %
\end{align*}
%
Similarly, we can express as univariate polynoimal of $Y_{k+1}$ 
with $2^{k-1}$ degree.
%
\begin{align*}
  f_{k+1}(Y_1, \dots , Y_{k}, Y_{k+1}) =
  c_{2^{k-1}}Y_{k+1}^{2^{k-1}} + c_{2^{k-1}-1}Y_{k+1}^{2^{k-1}-1} + 
   \cdots + c_2Y_{k+1}^2 + c_1Y_{k+1} + c_0
\end{align*}
%
All odd terms include multiplication of coefficient $a_1$ odd times.
%
Because degree of all summation polynomial are even, 
we can consider Lemma 1 in this case.
%
Hence, coefficients of all odd terms $c_{2k+1}$ include 
multiplication of coefficient $b_{2k+1}$ odd times.
%
From the assumption, $b_{2k+1}$ consist of only odd variables.
%
Therefore, all variables in all odd terms are odd variables.
%
Similarly, we can prove proposion in the case of even terms.
%
\qed \\


\begin{description}
  \item [theorem 1]~\\
  %
  Let us think about point decomposition of $m$ point 
  by using summation polynomial with $m+1$ variables, 
  $f_{m+1}(Y_0, \dots, Y_m, Y)$.
  %
  $Y$ is associated with point to decompose 
  and will be substituted by some constant.

  Suppose 3rd summation polynomial consists of only well-regulated terms.
  %
  When $m$ is even number,
  polynomial after substituting consist of only even monomials.

  When $m$ is odd number,
  polynomial after substituting includes some odd monomials.
  %
\end{description}

\noindent
\emph{Proof.}
%
Because max degree of each variables of 
$f_{m+1}(Y_0, \dots, Y_m, Y)$ are $2^{m-1}$,
it can be written as univariate polynoimal of $Y$.
%
\begin{align*}
  a_{2^{m-1}}Y^{2^{m-1}} + a_{2^{m-1}-1}Y^{2^{m-1}-1} + 
  \cdots + a_2Y^2 + a_1Y + a_0
\end{align*}
%
The coefficients $c_i$ are a polynomial 
with $m$ variables ($Y_0, \dots, Y_m$).
%
After substituting of $Y$, summation polynomial is composed of $c_i$.
%
Let $k$ be integers. 
From Lemma 2, all monomials in $a_{2k+1}$ consist of odd degree variables.
%
If $m$ is even number, there are all even monomials in $a_{2k+1}$
because even times summation of odd numbers are even.
%
Similarly, if $m$ is odd number, all monomials in $a_{2k+1}$ are odd monomials.
\qed \\

\begin{description}
  \item [Corollary 1]~\\
    As you see above of this section, 
    in twisted Edwards curve, 3rd summation polynomial consists of
    only well-regulated terms.
    %
    Therefore, twisted Edwards curve is the case of Theorem 1.
\end{description}






\subsection{What price for a highly symmetric factor base?}

Last but not least, we discuss the price needed to pay to have a
highly symmetric factor base $\mathcal F$ that is invariant under
additional group actions than that of the symmetric group $S_m$.
%
We first note that the action of $S_m$ do not give any new relations.
%
As previewed in Section~\ref{sec:symmetry-decomposition-probability},
we would expect that the effect of the decrease in decomposition
probability due to additional symmetry in $\mathcal F$ could be offset
by that of the increase in number of solutions.
%
For example, let us reconsider the group action of addition of $T_2$
in Section~\ref{sec:exploit-symmetry}.
%
If we could get $2^{m-1}$ solutions, then the loss of the factor of
$2^{m-1}$ in decomposition probability would be compensated.
%
This way everything would be the same as if there were no such
symmetry, and we could exploit the additional symmetry at no cost.

Unfortunately, this proposition is \emph{false} in general.
%
Consider an example of $m=4$.
%
Let $Q_i=P_i+ T_2$ for $i=1,2,3,4$.
%
We can write down all $2^{m-1}=8$ possible ways of a point
decomposition under this group action:
%
\[ \begin{aligned}
P_1 + P_2 + P_3 + P_4 = & Q_1 + Q_2 + P_3 + P_4 \\
= Q_1 + P_2 + Q_3 + P_4 = & Q_1 + P_2 + P_3 + Q_4 \\
= P_1 + Q_2 + Q_3 + P_4 = & P_1 + Q_2 + P_3 + Q_4 \\
= P_1 + P_2 + Q_3 + Q_4 = & Q_1 + Q_2 + Q_3 + Q_4.
\end{aligned} \]
%
It is easy to find that we have only 5 linearly independent relations
from these 8 relations, as there are nontrivial linear combinations
summing to zero, e.g.:
\[ (P_1 + P_2 + P_3 + P_4) - (Q_1 + Q_2 + P_3 + P_4) - (P_1 + P_2 +
  Q_3 + Q_4) + (Q_1 + Q_2 + Q_3 + Q_4) = \mathcal O.\]


As explained in Section~\ref{subsec:conditions}, the factor bases for
Montgomery and twisted Edwards curves are invariant under addition of
2-torsion points.
%
For $m=3$, we achieve maximum rank of $2^{m-1}=4$.
%
For $m=4$, as we have explained above, we can only have rank $5$, 
which is strictly less than the maximum possible rank $2^{m-1}=8$.

Finally, we note that we have not exploited any symmetry for Hessian
curves in our experiments.
%
However, the rank for Hessian curves is always 1 in all our
experiments.
%
This shows that the factor base we have chosen for Hessian curves is
\emph{not} invariant under addition of small torsion points, as the
rank would be $>1$ otherwise.







%
% ---- Bibliography ----
%
\bibliographystyle{abbrv}
\bibliography{dblp,local}

\end{document}
