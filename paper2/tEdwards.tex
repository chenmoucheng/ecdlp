%
% (twisted) Edwards curve
%

%------------------------------
\section{(twisted) Edwards curves}
%------------------------------------
\label{sec:twisted-edwards}


Faug\`ere, Gaudry, Hout, and Renault studied the point decomposition
problem on twisted Edwards, twisted Jacobi intersections, and
Weierstrass curves~\cite{DBLP:conf/eurocrypt/FaugereHJRV14}.
%
For the sake of completeness, we include some basic facts about
(twisted) Edwards curves here.
%
%
An Edwards curve over \F{p^n} for $p\neq 2$ is defined by the
equation \begin{equation*}
  x^2+y^2=1+dx^2y^2 \label{eq:edwards-curve} \end{equation*} for some
$d\in\F{p^n}$~\cite{DBLP:journals/iacr/BernsteinL07}.
%
A twisted Edwards curve $tE_{a',d'}$ over \F{p^n} for $p\neq 2$ is
defined by the equation \begin{equation}
  a'x^2+y^2=1+d'x^2y^2 \label{eq:twisted-edwards-curve} \end{equation}
for some $a',d'\in\F{p^n}$~\cite{DBLP:journals/iacr/BernsteinBJLP08}.
%
A twisted Edwards curve is a quadratic twist of an Edwards curve by
$a_0=1/(a'-d')$.
%
For $P=(x,y)\in H_d$, $-P=(-x,y)$.
%
Furthermore, the addition and doubling formulae for
$(x_3,y_3)=(x_1,y_1)+(x_2,y_2)$ are given as follows.
%
\begin{itemize}
\item When $(x_1,y_1)\neq(x_2,y_2)$:
  \begin{align*}
    x_3 & = \frac{x_1y_2 + y_1x_2}{1 + d'x_1x_2y_1y_2} \\
    y_3 & = \frac{y_1y_2 - a'x_1x_2}{1 - d'x_1x_2y_1y_2}
  \end{align*}
\item When $(x_1,y_1)=(x_2,y_2)$:
  \begin{align*}
    x_3 & = \frac{2x_1y_1}{1 + d'x_1^2y_1^2} \\
    y_3 & = \frac{y_1^2 - a'x_1^2}{1 - d'x_1^2y_1^2}
  \end{align*}
\end{itemize}

Summation polynomial on twisted Edwards curve is as follows.
%
\begin{align*}
  f_{tE, 2}(Y_1, Y_2) &= Y_2-Y_1 \\
    %
  f_{tE, 3}(Y_1, Y_2, Y_3) &=
    \left( Y_1^2Y_2^2-Y_1^2-Y_2^2+\frac{a}{d} \right)Y_3^2  \\
    &~~~~~ + 2\frac{d-a}{d}Y_1Y_2Y_3 + \frac{a}{d}\left(Y_1^2+Y_2^2-1\right)-Y_1^2Y_2^2\\
    %
\end{align*}
%
As you have seen, summation polynomial for $m\geq4$ is 
calculated by using resultants.


\subsection{Observations of summation polynomial on twisted Edwards curve}

We classify variables and monomials in from the viewpoint of parity. 
%
We refer to variable whose exponent is even number and odd number 
as respectively \emph{even variable} and \emph{odd variable}
%
Similarly, we refer to monomial whose degree is even number and odd number 
as respectively \emph{even monomial} and \emph{odd monoimal}


\begin{description}
  \item [Lemma 1]~\\
  %
  Considering resultant of two multivariate polynomial $a_1$ and $a_2$,
  you will consider the polynomials as univariate polynomial of the variable
  to eliminate $T$, such as $f_1 = a_mT^m + \dots +a_1T^1 + a_0$ 
  and $f_2 = b_nT^n + \dots +b_1T^1 + b_0$.
  %
  If degree of both of $m$ and $n$ is even, 
  then coefficients of odd monomial, $a_{2k+1}$ and $b_{2k+1}$ for integer $k$,
  is contained even number times in the term of resultant.
  %
\end{description}

\noindent
\emph{Proof.}
%
Resultant of $a_1$ and $a_2$ is 
determinant of following square matrix X of size $m+n$.
%
\begin{equation*}
X =
\begin{bmatrix}
  a_m & a_{m-1} & \ldots & & a_0 & & &  \\
          & a_m & a_{m-1} & \ldots & & a_0  & &  \\
  & & \ddots & & & & \ddots &  \\
    & & & a_m & a_{m-1} & \ldots & & a_0  \\
  b_n & b_{n-1} & \ldots & & b_0 & & &  \\
       & b_n & b_{n-1} & \ldots & & b_0  & &  \\
  & & \ddots & & & & \ddots &  \\
    & & & b_n & b_{n-1} & \ldots & & b_0
\end{bmatrix}
\end{equation*}

Let components in the $i$-th row and $j$-th column in the matrix $X$ 
be represented to $x_{ij}$. 
%
From assumption, $a_m$ and $b_n$ is the coefficients of even monomial
and the number of row which include $a_i$ and $b_i$ is both even.
%
Consequently, coefficients of odd monomials, $a_{2k+1}$ and $b_{2k+1}$ 
for integer $k$, correspond components whose sum of index (i+j) is odd.
%
Also, coefficients of even monomials, $a_{2k}$ and $b_{2k}$ for 
integer $k$, correspond components whose sum of index is even.

Here, determinant of $X$ is defined as
%
\begin{align*}
  %
  \mbox{det}(X) = \sum_{\sigma \in S_{n+m}} \mbox{sgn}(\sigma)x_{1\sigma(1)}
                   x_{2\sigma(2)}\cdots x_{m+n\sigma(m+n)}
\end{align*}
%
Under any permutation, total summation of sum of index remain even number,
that is $2*(1+ \cdots +n)$.
%
Hence, components whose sum of index is odd is contained even number times.
%
Therefore, coefficients of odd monoials is contained even number times
in the term of resultant.
\qed \\

We focus on each variable which is consisted term
is whether even or odd variable.
%
We reffer to the term which consist of only the same parity of variables
as \emph{well-regulated term}.


\begin{description}
  %
  \item [Lemma 2]~\\
  %
  If all terms in 3rd summation polynomial are well-regulated terms,
  all terms in summation polynomial with any variables are also
  well-regulated.
  %
\end{description}

\noindent
\emph{Proof.}
%
We prove that summation polynomial $f_{n}$ for $n\geq3$ satisfy the
proposion by mathematical inducution.

Let $n=3$, from assumption, summation polynomial $f_3(Y_1, Y_2, Y_3)$
satisfy proposion.
%
\begin{align*}
  %
  f_{3}(Y_1, Y_2, Y_3)=a_2Y_3^2 + a_1Y_3 + a_0
  %
\end{align*}
%

Let $n=k$, we assume that $f_{k}(Y_1, \dots , Y_{k})$ satisfy the proposion.
% 
we can express the polynomial as univariate polynomial of $Y_{k}$ 
with $2^{k-2}$ degree. 
%
\begin{align*}
  %
  f_{k}(Y_1, \dots , Y_{k}) & = 
  b_{2^{k-2}}Y_{k}^{2^{k-2}} + b_{2^{k-2}-1}Y_{k}^{2^{k-2}-1} + 
   \cdots + b_2Y_{k}^2 + b_1Y_{k} + b_0
  %
\end{align*}
%
In other words, we assume polynomial $b_{2k}$ consist of only even variables 
and $b_{2k+1}$ consist of only odd variables for integer $k$.

Let $n=k+1$, we will consider about $f_{k+1}(Y_1, \dots , Y_{k}, Y_{k+1})$.
%
From the definition of summation polynomial, we can express the polynomial
by using above coefficient $a_i$ and $b_i$.
%
\begin{align*}
  %
  f_{k+1}(Y_1, \dots , Y_{k}, Y_{k+1}) &= 
   Res\left(f_{k}(Y_1, \dots , Y_{k-1}, T), f_3(Y_{k}, Y_{k+1}, T)\right)\\
  %
   & = Res(b_{2^{k-2}}T^{2^{k-2}} + \cdots + b_1T + b_0,~ a_2T^2 + a_1T + a_0)
  %
\end{align*}
%
Similarly, we can express as univariate polynoimal of $Y_{k+1}$ 
with $2^{k-1}$ degree.
%
\begin{align*}
  f_{k+1}(Y_1, \dots , Y_{k}, Y_{k+1}) =
  c_{2^{k-1}}Y_{k+1}^{2^{k-1}} + c_{2^{k-1}-1}Y_{k+1}^{2^{k-1}-1} + 
   \cdots + c_2Y_{k+1}^2 + c_1Y_{k+1} + c_0
\end{align*}
%
All odd terms include multiplication of coefficient $a_1$ odd times.
%
Because degree of all summation polynomial are even, 
we can consider Lemma 1 in this case.
%
Hence, coefficients of all odd terms $c_{2k+1}$ include 
multiplication of coefficient $b_{2k+1}$ odd times.
%
From the assumption, $b_{2k+1}$ consist of only odd variables.
%
Therefore, all variables in all odd terms are odd variables.
%
Similarly, we can prove proposion in the case of even terms.
%
\qed \\


\begin{description}
  \item [theorem 1]~\\
  %
  Let us think about point decomposition of $m$ point 
  by using summation polynomial with $m+1$ variables, 
  $f_{m+1}(Y_0, \dots, Y_m, Y)$.
  %
  $Y$ is associated with point to decompose 
  and will be substituted by some constant.

  Suppose 3rd summation polynomial consists of only well-regulated terms.
  %
  When $m$ is even number,
  polynomial after substituting consist of only even monomials.

  When $m$ is odd number,
  polynomial after substituting includes some odd monomials.
  %
\end{description}

\noindent
\emph{Proof.}
%
Because max degree of each variables of 
$f_{m+1}(Y_0, \dots, Y_m, Y)$ are $2^{m-1}$,
it can be written as univariate polynoimal of $Y$.
%
\begin{align*}
  a_{2^{m-1}}Y^{2^{m-1}} + a_{2^{m-1}-1}Y^{2^{m-1}-1} + 
  \cdots + a_2Y^2 + a_1Y + a_0
\end{align*}
%
The coefficients $c_i$ are a polynomial 
with $m$ variables ($Y_0, \dots, Y_m$).
%
After substituting of $Y$, summation polynomial is composed of $c_i$.
%
Let $k$ be integers. 
From Lemma 2, all monomials in $a_{2k+1}$ consist of odd degree variables.
%
If $m$ is even number, there are all even monomials in $a_{2k+1}$
because even times summation of odd numbers are even.
%
Similarly, if $m$ is odd number, all monomials in $a_{2k+1}$ are odd monomials.
\qed \\

\begin{description}
  \item [Corollary 1]~\\
    As you see above of this section, 
    in twisted Edwards curve, 3rd summation polynomial consists of
    only well-regulated terms.
    %
    Therefore, twisted Edwards curve is the case of Theorem 1.
\end{description}



