%
% (twisted) Edwards curve
%

\subsection{Missing terms of summation polynomials in (twisted) Edwards curves}
\label{sec:twisted-edwards-summation-polynomial}

% In this section, we present the main result of this paper, namely,
% some insights into what kind of summation polynomials are easier to
% solve.
%
% For example, the summation polynomials for (twisted) Edwards form of
% elliptic curves seem easier to solve compared with those for
% Weierstrass or Montgomery forms.
%
% The explanation offered by Faug\`ere, Gaudry, Hout, and Renault is
% ``due to the smaller degree appearing in the computation of Gr\"obner
% basis of $\mathscr S_{D_n}$ in comparison with the Weierstrass case,''
% cf.~Section~4.1.1 of their
% paper~\cite{DBLP:journals/joc/FaugereGHR14}.
% %
% Unfortunately, as will be detailed in Section~\ref{sec:experiment},
% this \emph{cannot} explain the difference in solving time between
% (twisted) Edwards and Montgomery forms, as the highest degrees
% appearing in the computation of Gr\"obner bases are \emph{the same}
% for these two forms.

% We offer a simpler explanation to this difference by counting the
% number of terms in the summation polynomials for curves in different
% forms.


In this section, we will show that the summation polynomials for
(twisted) Edwards curves \emph{mainly} have terms of \emph{even}
degrees.
%
The set of terms of even degrees is closed under multiplication, so
intuitively this kind of polynomials are easier to solve, which can be
the main reason for the efficiency gain observed in the case of
(twisted) Edwards curves.

We shall make this intuition precise in Theorem~\ref{th:main}, but
before we state the main result, we need to clarify our terminology
for ease of exposition.
%
When a multivariate polynomial is regarded as a univariate polynomial
in one of its variables $T$, we say that the coefficient $a_i$ of a
term $a_iT^i$ is an \emph{even or odd-degree coefficient} depending on
whether $i$ is even or odd, respectively.
%
Note that these coefficients are themselves multivariate polynomials
in one fewer variables.

We say that a monomial $m=\prod_{i=1}^n x_i^{e_i},e_i\geq 0$ in a
multivariate polynomial in $n$ variables is \emph{of even degree} or
simply an \emph{even-degree monomial} if $\sum_i e_i$ is even; that it
is \emph{of odd degree} or simply an \emph{odd-degree monomial}
otherwise.
%
In contrast, a monomial is \emph{of (homogeneous) even parity} if all
$e_i$ are even; it is \emph{of (homogeneous) odd parity} if all $e_i$
are odd.
%
A monomial is \emph{of homogeneous parity} if it is either of
homogeneous even or odd parity.
%
Note that the definition of monomials of odd parity depends on the
total number of variables in the polynomial, which is not the case for
monomials of even parity because we regard 0 as even.
%
For example, the monomial $x_1x_2$ is a monomial of odd parity in a
polynomial in $x_1$ and $x_2$ but not so in another polynomial in
$x_1,\ldots,x_n$ for $n>2$.

By abuse of language, we say that a polynomial is \emph{of even or odd
  parity} if it is a linear combination of monomials of even or odd
parity, respectively; that a polynomial is \emph{of homogeneous
  parity} if it is a linear combination of monomials of homogeneous
parity.
%
The set of polynomials of even parity is closed under polynomial
addition and multiplication and hence forms a subring.
%
On the other hand, a polynomial $f$ in $x_1,\ldots,x_n$ of odd parity
must have the form $\sum_i c_i\left(\prod_{j=1} x_j^{e_{ij}}\right)$,
for $e_{ij}$ odd.
%
Therefore, if $f$ is a polynomial of odd parity and $g$, a polynomial
of even parity, then $fg$ must be of odd parity.

\begin{theorem}
  % 
  \label{th:main}
  % 
  Let $\mathcal E$ be a family of elliptic curves such that its 3rd
  summation polynomial $f_{\mathcal E,3}(X_1,X_2,X_3)$ is of degree 2
  in each variable $X_i$ and of homogeneous parity.
  % 
  Let $g_{\mathcal E,m}$ be the polynomial corresponding to the PDP of
  $m$-th order for $\mathcal E$ as described in
  Section~\ref{sec:summation-polynomial}.
  % 
  That is,
  $g_{\mathcal E,m}(X_1,\ldots,X_m)=f_{\mathcal
    E,m+1}(X_1,\ldots,X_m,x)$, where $x$ is a constant depending on
  the point to be decomposed.
  % 
  \begin{enumerate}
    % 
  \item If $m$ is even, then $g_{\mathcal E,m}$ has no monomials of
    odd degrees.
    % 
  \item If $m$ is odd, then $g_{\mathcal E,m}$ has some but not all
    monomials of odd degrees.
    % 
  \end{enumerate}
  % 
\end{theorem}
%
Among the four forms of elliptic curves that we have investigated in
this paper, only the (twisted) Edwards form satisfies the premises of
Theorem~\ref{th:main}.
%
As we have seen in Section~\ref{sec:experiment}, the PDP solving time
for the (twisted) Edwards form is thus significantly faster than that
for the other forms.

We will prove Theorem~\ref{th:main} in the rest of this section, for
which we will need the following lemmas.
%
\begin{lemma}
  % 
  \label{th:resultant}
  % 
  Let $f_1(T_1,\ldots,T_r,T)=a_0 + a_1T + \cdots + a_mT^m$ and
  $f_2(T_1,\ldots,T_r,T)=b_0 + b_1T + \cdots + b_nT^n$ be two
  polynomials in $r+1$ variables, where $a_i$ and $b_i$ are
  polynomials in $T_1,\ldots,T_r$.
  % 
  Let $f(T_1,\ldots,T_r)=\res_T(f_1,f_2)$ be the resultant of $f_1$
  and $f_2$ regarded as two univariate polynomials in $T$.
  % 
  If both $m$ and $n$ are even, then every monomial of $f$ is a
  product of an even number or none of the odd-degree coefficients of
  $f_1$ and $f_2$ and some or none of the even-degree coefficients of
  $f_1$ and $f_2$.
  % 
  Specifically, the odd-degree coefficients $a_{2k+1}$ and $b_{2k+1}$
  of $f_1$ and $f_2$, respectively, appear in total an even number of
  times in each monomial of $f$.
  % 
\end{lemma}
%
\begin{proof}
  % 
  The resultant $\res_T(f_1,f_2)$ of $f_1$ and $f_2$ is the
  determinant of the following $(m+n)\times(m+n)$ matrix $S$:
  \begin{equation}
    \label{eq:resultant}
    S = \begin{bmatrix}
      a_m & a_{m-1} & \ldots & & a_0 & & &  \\
      & a_m & a_{m-1} & \ldots & & a_0  & &  \\
      & & \ddots & & & & \ddots &  \\
      & & & a_m & a_{m-1} & \ldots & & a_0  \\
      b_n & b_{n-1} & \ldots & & b_0 & & &  \\
      & b_n & b_{n-1} & \ldots & & b_0  & &  \\
      & & \ddots & & & & \ddots &  \\
      & & & b_n & b_{n-1} & \ldots & & b_0
    \end{bmatrix}.
    \begin{tabular}{@{}c@{}}
      $\left.\begin{tabular}{@{}c@{}} \\ \\ \\ \\\end{tabular}\right\} n$\\
      \\
      $\left.\begin{tabular}{@{}c@{}} \\ \\ \\ \\\end{tabular}\right\} m$\\
    \end{tabular}
  \end{equation}
  % 
  We denote as $s_{ij}$ the entry at the $i$-th row and $j$-th column
  of $S$ for $1\leq i,j\leq m+n$.
  % 
  Since both $m$ and $n$ are even, an even-degree coefficient $a_{2k}$
  or $b_{2k}$ will appear in $s_{ij}$ for which the sum of indices
  $i+j$ is even.
  % 
  Similarly, an odd-degree coefficient $a_{2k+1}$ or $b_{2k+1}$ will
  appear in $s_{ij}$ for which the sum of indices $i+j$ is odd.
  % 
  Now recall that the determinant of $S$ is defined as
  \[ \sum_{\sigma\in S_{n+m}}\sgn(\sigma)s_{1,\sigma(1)}\cdot
    s_{2,\sigma(2)}\cdots s_{m+n,\sigma(m+n)}. \]
  % 
  We note that the sum of the indices of any summand is
  \[ \sum_i^{m+n}i+\sigma(i)=(m+n)(m+n+1), \] which is always even.
  % 
  Therefore, the odd-degree coefficients must appear an even number of
  times, thus completing the proof.
  % 
  \qed
  % 
\end{proof}
%
\begin{lemma}
  % 
  \label{th:summation-polynomial}
  %
  Let $\mathcal E$ be a family of elliptic curves such that its 3rd
  summation polynomial $f_{\mathcal E,3}(X_1,X_2,X_3)$ is of degree 2
  in each variable $X_i$ and of homogeneous parity.
  % 
  Then any subsequent summation polynomial
  $f_{\mathcal E,m}(X_1,\ldots,X_m)$ for $m>3$ is of homogeneous
  parity.
  % 
\end{lemma}
% 
\begin{proof}
  %
  As the summation polynomial $f_{\mathcal E,m+1}$ for $m\geq 3$ is
  defined recursively from $f_{\mathcal E,m}$ and $f_{\mathcal E,3}$
  via taking resultants
  \[ f_{\mathcal E,m+1}(X_1,\dots,X_{m+1}) = \res_X\left(f_{\mathcal
        E,m}(X_1,\dots,X_{m-1},X),f_{\mathcal
        E,3}(X_m,X_{m+1},X)\right), \]
  % 
  we shall prove this lemma by induction on $m$.
  %
  Let
  $f_{\mathcal
    E,m}(X_1,\ldots,X_{m-1},X)=a_{2^{m-2}}X^{2^{m-2}}+\cdots+a_1X+a_0$
  and $f_{\mathcal E,3}(X_m,X_{m+1},X)=b_2X^2+b_1X+b_0$.
  % 
  By the premise that $f_{\mathcal E,3}$ is of homogeneous parity,
  $b_0$ and $b_2$ must consist only of monomials (in $X_m$ and
  $X_{m+1}$) of even parity.
  % 
  Furthermore, $b_1=cX_mX_{m+1}$ for some constant $c$.
  % 
  This is because $f_{\mathcal E,3}$ is of degree 2 in each variable,
  for which the only monomial of odd parity is $X_mX_{m+1}X$.

  Now consider a term $c_kX_{m+1}^k$ of
  \[ f_{\mathcal
      E,m+1}(X_1,\ldots,X_m,X_{m+1})=c_{2^{m-1}}X_{m+1}^{2^{m-1}}+\cdots+c_1X_{m+1}+c_0 \]
  as a univariate polynomial in $X_{m+1}$.
  % 
  Again as $f_{\mathcal E,3}$ is of degree 2 in $X$, we have the case
  of $n=2$ in Equation~\ref{eq:resultant}.
    % 
  Now $X_{m+1}$ must come from $b_1$, so we can conclude that
  \[ c_kX_{m+1}^k=\sum_i\alpha_i a_{\beta_i} a_{\gamma_i}
    b_0^{\delta_i} b_2^{\epsilon_i} X_m^kX_{m+1}^k, \] where
  $\alpha_i$ a constant, $\beta_i,\gamma_i\in\{0,\ldots,2^{m-2}\}$,
  and $\delta_i,\epsilon_i$ nonnegative integers such that
  $\delta_i+\epsilon_i+k=2^{m-2}$.
  %
  We will complete the proof by showing as follows that $c_kX_{m+1}^k$
  is a polynomial in $X_1,\ldots,X_{m+1}$ of homogeneous parity for
  all $k$.
  % 
  \begin{enumerate}
    %
  \item If $k$ is even, then by Lemma~\ref{th:resultant},  $\beta_i$
    and $\gamma_i$ are both even or both odd in each summand.
    %
    In either case, the product $a_{\beta_i}a_{\gamma_i}$ is a
    polynomial in $X_1,\ldots,X_{m-1}$ of even parity.
    % 
    It follows that each summand is a polynomial of even parity
    because it is a product of polynomials of even parity.
    % 
    Hence $c_kX_{m+1}^k$ is a polynomial of even parity.
    %
  \item If $k$ is odd, the situation is similar but slightly more
    complicated.
    % 
    By Lemma~\ref{th:resultant}, exactly one of $\beta_i$ and
    $\gamma_i$ is odd in each summand, say $\beta_i$.
    % 
    By induction hypothesis, $a_{\beta_i}$ is a polynomial in
    $X_1,\ldots,X_{m-1}$ of odd parity because it comes from
    $a_{\beta_i} X^{\beta_i}$ in $f_{\mathcal E,m}$.
    %
    It follows that each summand is a polynomial of odd parity because
    it is a product of a polynomial of even parity
    $a_{\gamma_i} b_0^{\delta_i} b_2^{\epsilon_i}$ and a polynomial of
    odd parity $a_{\beta_i} X_m^kX_{m+1}^k$.
    % 
    Hence $c_kX_{m+1}^k$ is a polynomial of odd parity.
    %
  \end{enumerate}
  % 
  \qed
  % 
\end{proof}

By Lemma~\ref{th:summation-polynomial},
$g_{\mathcal E,m}(X_1,\ldots,X_m)=f_{\mathcal
  E,m+1}(X_1,\ldots,X_m,x)$ is of homogeneous parity.
%
Obviously, the monomials of even parity will remain of even degree
after $x$ is substituted.
%
If $m$ is even, then the monomials of odd parity in
$f_{\mathcal E,m+1}$ will become of even degree after $x$ is
substituted because an even number of odd numbers sum to an even
number.
%
Similarly if $m$ is odd, then the monomials of odd parity in
$f_{\mathcal E,m+1}$ will become of odd degree after $x$ is
substituted.
%
However, those odd-degree monomials that are \emph{not} of homogeneous
parity, e.g., $X_1^2X_2$, cannot appear in $g_{\mathcal E,m}$ by
Lemma~\ref{th:summation-polynomial}.
%
This completes the proof of Theorem~\ref{th:main}.
