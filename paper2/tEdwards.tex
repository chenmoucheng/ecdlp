%
% (twisted) Edwards curve
%

%------------------------------
\subsection{(Twisted) Edwards curves}
%------------------------------------
\label{sec:twisted-edwards}


Faug\`ere, Gaudry, Hout, and Renault studied PDP on twisted Edwards,
twisted Jacobi intersections, and Weierstrass
curves~\cite{DBLP:journals/joc/FaugereGHR14}.
%
For the sake of completeness, we include some basic facts about
(twisted) Edwards curves here.
%
%
An Edwards curve over \F{p^n} for $p\neq 2$ is defined by the
equation \begin{equation*}
  x^2+y^2=1+dx^2y^2 \label{eq:edwards-curve} \end{equation*} for
$d\in\F{p^n}$~\cite{DBLP:journals/iacr/BernsteinL07}.
%
A twisted Edwards curve $tE_{a',d'}$ over \F{p^n} for $p\neq 2$ is
defined by the equation \begin{equation}
  a'x^2+y^2=1+d'x^2y^2 \label{eq:twisted-edwards-curve} \end{equation}
for $a',d'\in\F{p^n}$~\cite{DBLP:journals/iacr/BernsteinBJLP08}.
%
A twisted Edwards curve is a quadratic twist of an Edwards curve by
$a_0=1/(a'-d')$.
%
For $P=(x,y)\in tE_{a',d'}$, $-P=(-x,y)$.
%
Furthermore, the addition and doubling formulae for
$(x_3,y_3)=(x_1,y_1)+(x_2,y_2)$ are given as follows.
%
\begin{itemize}
\item When $(x_1,y_1)\neq(x_2,y_2)$:
  \begin{align*}
    x_3 & = \frac{x_1y_2 + y_1x_2}{1 + d'x_1x_2y_1y_2} \\
    y_3 & = \frac{y_1y_2 - a'x_1x_2}{1 - d'x_1x_2y_1y_2}
  \end{align*}
\item When $(x_1,y_1)=(x_2,y_2)$:
  \begin{align*}
    x_3 & = \frac{2x_1y_1}{1 + d'x_1^2y_1^2} \\
    y_3 & = \frac{y_1^2 - a'x_1^2}{1 - d'x_1^2y_1^2}
  \end{align*}
\end{itemize}
%
As given by Faug\`ere, Gaudry, Hout, and
Renault~\cite{DBLP:journals/joc/FaugereGHR14}, the 3rd summation
polynomial for twisted Edwards curve is
%
\begin{align*}
  f_{tE, 3}(Y_1,Y_2,Y_3) = & \left(Y_1^2Y_2^2 - Y_1^2 - Y_2^2 + \frac{a}{d}\right)Y_3^2  + 2\frac{d-a}{d}Y_1Y_2Y_3 +
                             \frac{a}{d}\left(Y_1^2 + Y_2^2 - 1\right)
                             - Y_1^2Y_2^2.
\end{align*}
%
As usual, subsequent summation polynomials can be obtained by using
resultants.



\section{What summation polynomials are easier to solve?}
\label{sec:twisted-edwards-summation-polynomial}

In this section, we present the main result of this paper, namely,
some insights into what kind of summation polynomials are easier to
solve.
%
For example, the summation polynomials for (twisted) Edwards form of
elliptic curves seem easier to solve compared with those for
Weierstrass or Montgomery forms.
%
The explanation offered by Faug\`ere, Gaudry, Hout, and Renault is
``due to the smaller degree appearing in the computation of Gr\"obner
basis of $\mathscr S_{D_n}$ in comparison with the Weierstrass case,''
cf.~Section~4.1.1 of their
paper~\cite{DBLP:journals/joc/FaugereGHR14}.
%
Unfortunately, as will be detailed in Section~\ref{sec:experiment},
this \emph{cannot} explain the difference in solving time between
(twisted) Edwards and Montgomery forms, as the highest degrees
appearing in the computation of Gr\"obner bases are \emph{the same}
for these two forms.

We offer a simpler explanation to this difference by counting the
number of terms in the summation polynomials for curves in different
forms.
%
We will show that the summation polynomials for (twisted) Edwards form
of elliptic curves \emph{mainly} have terms of \emph{even} powers.
%
The set of terms of even powers are closed under multiplication, so
intuitively polynomials mainly consisting of even-power terms also
``generate'' such kind of polynomials after taking resultants.
%
We believe that this kind of summation polynomials are easier to
solve, which is main reason for the efficiency gain observed in the
case of (twisted) Edwards curves.

We shall make such intuition precise in Theorem~\ref{th:main}, but
before we state the main result, we need to define some terminology
for ease of exposition.
%
When a multivariate polynomial is regarded as a univariate polynomial
in one of its variables $T$, we say that the coefficient $a_i$ of a
term $a_iT^i$ is \emph{of even power} or simply an \emph{even-power
  coefficient} if $i$ is even; that it is \emph{of odd power} or
simply an \emph{odd-power coefficient} otherwise.
%
Note that these coefficients are themselves multivariate polynomials
in one fewer variables.
%
Also, we take 0 as an even number.

Similarly, we say that a monomial $m=\prod_i^n x_i^{e_i},e_i\geq 0$ in
a multivariate polynomial in $n$ variables is \emph{of even power} or
simply an \emph{even-power monomial} if $\sum_i e_i$ is even; that it
is \emph{of odd power} or simply an \emph{odd-power monomial}
otherwise.
%
In contrast, a monomial is \emph{of homogeneous even parity} if all
$e_i$ are even; it is \emph{of homogeneous odd parity} if all $e_i$
are odd.
%
A monomial is \emph{of homogeneous parity} if it is either of
homogeneous even parity or of homogeneous odd parity.
%
Note that the definition of monomials of homogeneous odd parity
depends on the number of variables in the polynomial, which is not the
case for monomials of homogeneous even parity, as 0 is even in our
definition.
%
For example, the monomial $x_1x_2$ is a monomial of homogeneous odd
parity in a polynomial in $x_1$ and $x_2$ but not so in another
polynomial in $x_1,\ldots,x_n$ for $n>2$.
%
\begin{theorem}
  % 
  \label{th:main}
  % 
  Let $\mathcal E$ be a family of elliptic curves such that its 3rd
  summation polynomial $f_{3,\mathcal E}(X_1,X_2,X_3)$ is of degree 2
  in each variable $X_i$ and consists only of monomials of homogeneous
  parity.
  % 
  Let $g_{m,\mathcal E}$ be the polynomial corresponding to the PDP of
  $m$-th order for $\mathcal E$ as described in
  Section~\ref{sec:summation-polynomial}.
  % 
  That is,
  $g_{m,\mathcal E}(X_1,\ldots,X_m)=f_{m+1,\mathcal
    E}(X_1,\ldots,X_m,x)$, where $x$ is a constant depending on the
  point to be decomposed.
  % 
  \begin{enumerate}
    % 
  \item If $m$ is even, then $g_{m,\mathcal E}$ has no monomials of
    odd power.
    % 
  \item If $m$ is odd, then $g_{m,\mathcal E}$ has some but not all
    monomials of odd power.
    % 
  \end{enumerate}
  % 
\end{theorem}
%
Among the four forms of elliptic curves presented in the last two
sections, only the (twisted) Edwards form satisfies the premises of
Theorem~\ref{th:main}.
%
As we will see in Section~\ref{sec:experiment}, the solving time for
the summation polynomials of (twisted) Edwards form is thus
significantly faster than that of the other forms.

We will prove Theorem~\ref{th:main} in the rest of this section, for
which we will need the following lemmas.
%
\begin{lemma}
  % 
  \label{th:resultant}
  % 
  Let $f_1(T_1,\ldots,T_r,T)=a_0 + a_1T + \cdots + a_mT^m$ and
  $f_2(T_1,\ldots,T_r,T)=b_0 + b_1T + \cdots + b_nT^n$ be two
  polynomials in $r+1$ variables, where $a_i$ and $b_i$ are
  polynomials in $T_1,\ldots,T_r$.
  % 
  Let $f(T_1,\ldots,T_r)=\res_T(f_1,f_2)$ be the resultant of $f_1$
  and $f_2$ regarded as two univariate polynomials in $T$.
  % 
  If both $m$ and $n$ are even, then every term of $f$ is a product of
  an even number or none of the odd-power coefficients of $f_1$ and
  $f_2$, some or none of the even-power coefficients of $f_1$ and
  $f_2$, and $\pm 1$.
  % 
  Specifically, the odd-power coefficients $a_{2k+1}$ and $b_{2k+1}$
  of $f_1$ and $f_2$, respectively, appear in total an even number of
  times in each term of $f$.
  % 
\end{lemma}
%
\begin{proof}
  % 
  The resultant $\res_T(f_1,f_2)$ of $f_1$ and $f_2$ is the
  determinant of the following $(m+n)\times(m+n)$ matrix $S$:
  % 
  \begin{equation*}
    S = \begin{bmatrix}
      a_m & a_{m-1} & \ldots & & a_0 & & &  \\
      & a_m & a_{m-1} & \ldots & & a_0  & &  \\
      & & \ddots & & & & \ddots &  \\
      & & & a_m & a_{m-1} & \ldots & & a_0  \\
      b_n & b_{n-1} & \ldots & & b_0 & & &  \\
      & b_n & b_{n-1} & \ldots & & b_0  & &  \\
      & & \ddots & & & & \ddots &  \\
      & & & b_n & b_{n-1} & \ldots & & b_0
    \end{bmatrix}.
  \end{equation*}
  % 
  We denote as $s_{ij}$ the entry at the $i$-th row and $j$-th column
  of $S$ for $1\leq i,j\leq m+n$.
  % 
  Since both $m$ and $n$ are even, an even-power coefficient $a_{2k}$
  or $b_{2k}$ will appear in $s_{ij}$ for which the sum of indices
  $i+j$ is even.
  % 
  Similarly, an odd-power coefficient $a_{2k+1}$ or $b_{2k+1}$ will
  appear in $s_{ij}$ for which the sum of indices $i+j$ is odd.
  % 
  Now recall that the determinant of $S$ is defined as
  \[ \sum_{\sigma\in S_{n+m}}\sgn(\sigma)s_{1,\sigma(1)}\cdot
    s_{2,\sigma(2)}\cdots s_{m+n,\sigma(m+n)}. \]
  % 
  We note that the sum of the indices of the factors is
  \[ \sum_i^{m+n}i+\sigma(i)=(m+n)(m+n+1), \] which is always even.
  % 
  Therefore, the odd-power coefficients must appear an even number of
  times, thus completing the proof.
  % 
  \qed
  % 
\end{proof}
%
\begin{lemma}
  % 
  \label{th:summation-polynomial}
  %
  Let $\mathcal E$ be a family of elliptic curves such that its 3rd
  summation polynomial $f_{3,\mathcal E}(X_1,X_2,X_3)$ is of degree 2
  in each variable $X_i$ and consists only of monomials of homogeneous
  parity.
  % 
  Then any subsequent summation polynomial
  $f_{m,\mathcal E}(X_1,\ldots,X_m)$ for $m>3$ consists only of
  monomials of homogeneous parity.
  % 
\end{lemma}
% 
\begin{proof}
  %
  As the summation polynomial $f_{m+1,\mathcal E}$ for
  $m\geq 3$ is defined recursively from $f_{m,\mathcal E}$ and
  $f_{3,\mathcal E}$ via taking resultants
  \[ f_{m+1,\mathcal E}(X_1,\dots,X_{m+1}) = \res_X\left(f_{m,\mathcal
        E}(X_1,\dots,X_{m-1},X),f_{3,\mathcal
        E}(X_m,X_{m+1},X)\right), \]
  % 
  we shall prove this lemma by induction on $m$.
  %
  Let
  $f_{m,\mathcal
    E}(X_1,\ldots,X_{m-1},X)=a_{2^{m-2}}X^{2^{m-2}}+\cdots+a_1X+a_0$
  and $f_{3,\mathcal E}(X_m,X_{m+1},X)=b_2X^2+b_1X+b_0$.
  % 
  By the premises that $f_{3,\mathcal E}$ consists only of monomials
  of homogeneous parity, $b_0$ and $b_2$ must consist only of
  monomials (in $X_m$ and $X_{m+1}$) of homogeneous even parity.
  % 
  Furthermore, $b_1=X_mX_{m+1}$.
  % 
  This is because the definition of monomials of homogeneous odd
  parity depends on the number of variables.
  % 
  In this case, the only such monomial is $X_mX_{m+1}X$.

  Now consider a term $c_kX_{m+1}^k$ of
  \[ f_{m+1,\mathcal
      E}(X_1,\ldots,X_m,X_{m+1})=c_{2^{m-1}}X_{m+1}^{2^{m-1}}+\cdots+c_1X_{m+1}+c_0 \]
  as a univariate polynomial in $X_{m+1}$.
  % 
  \begin{enumerate}
    %
  \item If $k$ is odd, then $b_1$ must appear an odd number of times
    in $c_kX_{m+1}^k$ to produce an odd power of $X_{m+1}$ because
    $b_0$ and $b_2$ can only produce even powers of $X_{m+1}$.
    % 
    \emph{As $b_1=X_mX_{m+1}$, $c_k$ must consist only of odd powers
      of $X_m$.}

    By induction hypothesis, $f_{m,\mathcal E}$ consists only of
    monomials of homogeneous parity.
    %
    Now consider an odd-power coefficient $a_{2\ell+1}$ of
    $f_{m,\mathcal E}$ as univariate polynomial in $X$.
    %
    This means that $a_{2\ell+1}$ consists only of monomials in
    $X_1,\ldots,X_{m-1}$ of homogeneous odd parity.
    % 
    By Lemma~\ref{th:resultant}, these odd-power coefficients appear
    in total an odd number of times in $c_kX_{m+1}^k$.
    % 
    \emph{Therefore, $c_k$ must consist only of odd powers of
      $X_1,\ldots,X_{m-1}$, and $c_kX_{m+1}^k$ is a monomial of
      homogeneous odd parity.}
    %
  \item By changing ``odd'' to ``even'' in the above argument, we
    conclude that \emph{for $k$ even, $c_kX_{m+1}^k$ is a monomial of
      homogeneous even parity.}
      % 
  \end{enumerate}
  % 
  \qed
  % 
\end{proof}

We are now ready to complete the proof of Theorem~\ref{th:main}.
%
By Lemma~\ref{th:summation-polynomial},
$g_{m,\mathcal E}(X_1,\ldots,X_m)=f_{m+1,\mathcal
  E}(X_1,\ldots,X_m,x)$ consists only of monomials of homogeneous
parity.
%
Obviously, the monomials of homogeneous even parity will remain of
even power after the substitution of $x$, as the exponent of $x$ must
be even as well.
%
If $m$ is even, then the monomials of homogeneous odd parity in
$f_{m+1,\mathcal E}$ will be an even-power monomial after the
substitution of $x$.
%
For odd $m$, the monomials of homogeneous odd parity will be of even
power after the substitution of $x$.
%
However, as $f_{m+1,\mathcal E}$ consists only of monomials of
homogeneous parity, it is impossible for some odd-power monomials to
appear in $g_{m,\mathcal E}$ after the substitution of $x$, thus
completing the proof of Theorem~\ref{th:main}.
