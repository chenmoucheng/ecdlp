%
% Isomorphisms among curves in various forms
%

\section{Isomorphisms among curves in various forms}
\label{sec:isomorphisms}

We study elliptic curves in four different forms: Hessian ($H$),
Weierstrass ($W$), Montgomery ($M$), and twisted Edwards ($tE$).
%
In order to make an apple-to-apple comparison, we will experiment on
curves that are isomorphic to one another over \F{p^n}.
%
That is, $H(\F{p^n})\cong W(\F{p^n})\cong M(\F{p^n})\cong tE(\F{p^n})$
as groups.
%
In this section, we present the isomorphisms that we will use in our
experiments.
%
We will also state whether the factor base will be invariant under the
action of addition of 2-torsion points for each of the four forms
under investigation.

We start from a Hessian curve $H_d$ satisfying $x^3 + y^3 + 1 = 3dxy$
for $d\in\F{p^n}$ such that the number of its rational points
$\#H_d(\F{p^n})$ is divisible by 12.
%
As we have seen in Section~\ref{sec:hessian}, the factor base of $H_d$
is in general not invariant under the addition of 2-torsion points.
%
From $H_d$, we can obtain an isomorphic Weierstrass curve $W_{a,b}$
satisfying $y^2 = x^3 + ax + b$ for $a = - 27d(d^3 + 8)$ and
$b = 54(d^6 - 20d^3 - 8)$~\cite{DBLP:conf/ches/Smart01}.
%
The isomorphism $\phi_{W,H}$ from $W_{a,b}(\F{p^n})$ to $H_d(\F{p^n})$
is defined over $\F{p^n}$ and is given by sending $(u,v)\in W_{a,b}$
to $(x,y)\in H_d$ where
\[ \left\{\begin{aligned}
x = & \frac{36(d^3 - 1) - v}{6(u + 9d^2)} - \frac{d}{2}, \\
y = & \frac{36(d^3 - 1) + v}{6(u + 9d^2)} - \frac{d}{2}.
\end{aligned}\right. \]
%
The inverse $\phi_{H,W}$ is given by
\[ \left\{\begin{aligned}
u = & \frac{12(d^3 - 1)}{d + x + y} - 9d^2, \\
v = & \frac{36(d^3 - 1)(y - x)}{d + x + y}.
\end{aligned}\right. \]
%
The factor base of $W_{a,b}$ is in general not invariant under the
addition of 2-torsion points~\cite{DBLP:journals/joc/FaugereGHR14}.

With a high probability, we can obtain a Montgomery curve $M_{A,B}$
satisfying $By^2 = x^3 + Ax^2 + x$ from $W_{a,b}$ by solving the
following equations
%
\[ \left\{\begin{aligned}
a = & \frac{3 - A^2}{3B^2}, \\
b = & \frac{2A^3 - 9A}{27B^3}.
\end{aligned}\right. \]
%
The isomorphism $\phi_{W,M}$ is defined over
$\F{p^n}$ and is given by sending $(u,v)\in W_{a,b}$ to
$(x,y)\in M_{A,B}$ where $x = Bu - 1/3A$ and $y = Bv$.
%
The inverse $\phi_{M,W}$ can be obtained by equation solving.
%
As we have seen in Section~\ref{sec:montgomery-symmetry}, the factor
base is invariant under the addition of a particular 2-torsion point
$(0,0)$, though we are not able to exploit this symmetry in general.

Finally, we can obtain a twisted Edwards curve $tE_{a',d'}$ satisfying
$a'x^2 + y^2 = 1 + d'x^2y^2$ from $M_{A,B}$ by taking
\[ \left\{\begin{aligned}
a' = & \frac{A + 2}{B}, \\
d' = & \frac{A - 2}{B}.
\end{aligned}\right. \]
%
Again we let $a_0=1/(a' - d')$ be the amount of quadratic twist.
%
The isomorphism $\phi_{W,tE}$ is defined over $\F{p^n}$ and given by
sending $(u,v)\in W_{a,b}$ to $(x,y)\in tE_{a',d'}$ where
\[ \left\{\begin{aligned}
x = & \frac{2a_0u}{v}, \\
y = & \frac{u - a_0}{u + a_0}.
\end{aligned}\right. \]
%
The inverse $\phi_{tE,W}$ is given by
\[ \left\{\begin{aligned}
u = & \frac{a_0(1 + y)}{1 - y}, \\
v = & \frac{2a_0^2(1 + y)}{x(1 - y)}.
\end{aligned}\right. \]
%
As shown by Faug\`ere, Gaudry, Hout, and
Renault~\cite{DBLP:journals/joc/FaugereGHR14}, the factor base is
invariant under the addition of the 2-torsion point $(0,-1)$.
