%
% Index calculus for ECDLP
%

\section{Index calculus for ECDLP}
%
\label{sec:index-calculus-ecdlp}
%
Let $E$ be an elliptic curve defined over a finite field \F{p^n}.
%
For cryptographic applications, we are mostly interested in a
prime-order subgroup generated by a rational point $P\in E(\F{p^n})$.
%
To find an integer $\alpha$ such that $Q=\alpha P$ for
$Q\in\langle P\rangle$ using an index calculus algorithm, one
typically works as follows.
%
\begin{enumerate}
%
\item Determine a \emph{factor base} $\mathcal F\subset E(\F{p^n})$.
%
\item Collect a set $\mathcal R$ of \emph{relations} by decomposing
  random points $a_iP+b_iQ$ into a sum of points from $\mathcal F$,
  i.e.,
  \[ \mathcal
    R=\left\{a_iP+b_iQ=\sum_{j=1}^mP_{i,j}:P_{i,j}\in\mathcal
      F\right\} \]
%
\item When $|\mathcal R|\approx|\mathcal F|$, eliminate the righthand
  side using linear algebra to obtain an equation in the form
  $aP+bQ=\mathcal O$, and $\alpha=-a/b\bmod\ord(P)$.
%
\end{enumerate} 

\subsection{Semaev's summation polynomials}
%
\label{sec:summation-polynomials}
%
As we can see, an important step in index calculus algorithms for
solving ECDLP is point decomposition on an elliptic curve.
%
It is straightforward that if two points sum to zero, then their
$x$-coordinates must be equal.
%
Let us now consider the simplest nontrivial case where three points
sum to the point at infinity.
%
Let $Z=\{(x_1,y_1),(x_2,y_2),(x_3,y_3)\in E(\F{p^n})$ and
$(x_1,y_1)+(x_2,y_2)+(x_3,y_3)=\mathcal O:(x_i,y_i)\in E(\F{p^n})\}$.
%
Clearly, $Z$ is in the variety of the ideal
$I\subset\F{p^n}[X_1,Y_1,X_2,Y_2,X_3,Y_3]$ generated by
\[ \left\{\begin{aligned}
      &  (X_3 - X_1)(Y_2 - Y_1) - (X_2 - X_1)(Y_3 - Y_1),\\
      & Y_i^2 - (X_i^3 + aX_i + b),i=1,2,3
    \end{aligned}\right\}. \]
%
Now let $J=I\cap\F{p^n}[X_1,X_2,X_3]$.
%
Using MAGMA's \texttt{EliminationIdeal} function, we obtain that $J$
is actually a principal ideal generated by the polynomial
$(X_2 - X_3)(X_1 - X_3)(X_1 - X_2)f_3$, where \[ \begin{aligned}
    f_3 = & X_1^2X_2^2 - 2X_1^2X_2X_3 + X_1^2X_3^2 - 2X_1X_2^2X_3 - 2X_1X_2X_3^2 - 2aX_1X_2 - 2aX_1X_3 \\
    & - 4bX_1 + X_2^2X_3^2 - 2aX_2X_3 - 4bX_2 - 4bX_3 + a^2.
  \end{aligned} \]
%
Clearly, the linear factors of the generator correspond to the
degenerated case where two or more points are the same, and $f_3$ is
the 3rd \emph{summation polynomial}, that is, the summation polynomial
for three distinct points summing to zero.

Starting from the 2nd and 3rd summation polynomials, one can
recursively obtain the subsequent summation polynomials via taking
resultants.
%
This is the observation Semaev made in his seminal
work~\cite{DBLP:journals/iacr/Semaev04}.
%
In short, his proposal is to consider factor bases of the following
form: $\{(x,y)\in E(\F{p^n}):x\in V\subset\F{p^n})\}$, where $V$ is a
subset of \F{p^n}.

\subsection{Weil restriction}
%
Restricting $x$-coordinates of the points in factor base to a subset
of \F{p^n} is important from a viewpoint of polynomial system solving.
%
Take $f_3$ as an example.
%
When decomposing a random point $aP+bQ$, we first substitute its
$x$-coordinate into say $X_3$, projecting $J$ onto $\F{p^n}[X_1,X_2]$.
%
The dimension of the variety of this ideal is nonzero.
%
Therefore, we would like to pose some restrictions on $X_1$ and $X_2$
to reduce the dimension to zero and make the solving time more
manageable.

When looking for solutions to a polynomial in $\F{p^n}[X]$ in \F{p^n},
we can view it as a commutative affine algebra
$\mathcal A=\F{p^n}/(X^{p^n} - X)\cong\F{p^n}[X_1,\ldots,X_n]/(X_1^p -
X_1,\ldots,X_n^p - X_n)$.
%
This can be done by identifying the indeterminate $X$ as
$X_1\theta_1+\cdots+X_n\theta_n$, where $(\theta_1,\ldots,\theta_n)$
is a basis for \F{p^n} over \F p.
%
Hence, a polynomial $f=\sum a_iX^i\in\F{p^n}[X]$ can be identified as
a polynomial $f_1\theta_1+\cdots+f_n\theta_n$, where
$f_1,\ldots,f_n\in\mathcal A'=\F p[X_1,\ldots,X_n]/(X_1^p -
X_1,\ldots,X_n^p - X_n)$, by appropriately sending any coefficient
$a\in\F{p^n}$ to $a_1\theta_1+\cdots+a_n\theta_n$ for
$a_1,\ldots,a_n\in\F p$.
%
Therefore, an equation $f=0$ over \F{p^n} will give rise to a system
of equations $f_1=\cdots=f_n=0$ over \F p.
%
This technique is known as the \emph{Weil restriction} and is used in
the Gaudry-Diem attack, in which the factor base is chosen to consist
of points whose $x$-coordinates lie in a subspace $V$ of \F{p^n} over
\F p~\cite{DBLP:journals/jsc/Gaudry09,DBLP:journals/moc/Diem11}.

\subsection{Exploiting symmetry}
%
\label{sec:exploit-symmetry}
%
Naturally, the symmetric group $S_m$ acts on a point decomposition
$P_1+\ldots+P_m$ because elliptic curve groups are abelian.
%
As noted by Gaudry in his seminal work, we can therefore rewrite the
variables $x_1,\ldots,x_m\in\F{p^n}$ by elementary symmetric
polynomials $e_1,\ldots,e_m$, where $e_1=\sum x_i$,
$e_2=\sum_{i\neq j}x_ix_j$,
$e_3=\sum_{i\neq j,i\neq k,j\neq k}x_ix_jx_k$,
etc~\cite{DBLP:journals/jsc/Gaudry09}.
%
Such rewriting can reduce the degree of summation polynomials, as well
as significantly speeding up the solving
time~\cite{DBLP:conf/eurocrypt/FaugerePPR12,DBLP:conf/iwsec/HuangPST13}.

One might be able to exploit further symmetry using actions by other
groups.
%
For example, further speed-up has been reported for point
decomposition by making the factor base invariant under addition of
some small torsion
points~\cite{DBLP:conf/eurocrypt/FaugereHJRV14,DBLP:conf/indocrypt/GalbraithG14}.
%
Such kind of speed-up is, of course, curve-specific in nature.
%
Furthermore, even if the factor base is invariant under addition of
small torsion points, we may or may not be able to exploit such
symmetry to speed up the solving time depending on whether the action
of these small torsion points are ``easy to handle in the polynomial
system solving process''~\cite{DBLP:conf/eurocrypt/FaugereHJRV14}.
