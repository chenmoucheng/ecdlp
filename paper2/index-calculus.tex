%
% Index calculus for ECDLP
%

\section{Previous works}
\label{sec:previous-work}

\subsection{Index calculus for ECDLP}
%
\label{sec:index-calculus-ecdlp}
%
Let $E$ be an elliptic curve defined over a finite field \F{p^n}.
%
For cryptographic applications, we are mostly interested in a
prime-order subgroup generated by a rational point $P\in E(\F{p^n})$.
%
Here we first give a high-level overview of a typical index calculus
algorithm for finding an integer $\alpha$ such that $Q=\alpha P$ for
$Q\in\langle P\rangle$.
%
\begin{enumerate}
%
\item Determine a \emph{factor base} $\mathcal F\subset E(\F{p^n})$.
%
\item Collect a set $\mathcal R$ of \emph{relations} by decomposing
  random points $a_iP+b_iQ$ into a sum of points from $\mathcal F$,
  i.e.,
  \[ \mathcal R=\left\{a_iP+b_iQ=\sum_jP_{i,j}:P_{i,j}\in\mathcal
      F\right\}. \]
%
\item When $|\mathcal R|\approx|\mathcal F|$, eliminate the righthand
  side using linear algebra to obtain an equation in the form
  $aP+bQ=\mathcal O$, and $\alpha=-a/b\bmod\ord(P)$.
%
\end{enumerate}
%
The last step of linear algebra is relatively well studied in the
literature, so we will focus on the subproblem in the second step,
namely, the point decomposition problem (PDP) on an elliptic curve in
the rest of this paper.
%
\begin{definition}[Point Decomposition Problem of $m$-th order]
\label{def:pdp}
% 
Given a rational point $R\in E(\F{p^n})$ on an elliptic curve $E$ and
a factor base $\mathcal F\subset E(\F{p^n})$, find, if they exist,
$P_1,\dots,P_m\in\mathcal F$ such that \[ R=P_1 + \cdots + P_m. \]
%
\end{definition}

\subsection{Semaev's summation polynomials}
%
\label{sec:summation-polynomial}
%
We can solve PDP by considering when a set of points sum to zero on an
elliptic curve.
%
It is straightforward that if two points sum to zero on an elliptic
curve $E: y^2=x^3+ax+b$ in Weierstrass form, then their
$x$-coordinates must be equal.
%
Let us now consider the simplest yet nontrivial case where three
points on $E$ sum to zero.
%
Let
\[ Z=\left\{\begin{aligned}
      (x_1,y_1,x_2,y_2,x_3,y_3)&\in\F{p^n}^6:(x_i,y_i)\in E(\F{p^n}),i=1,2,3; \\
      & (x_1,y_1)+(x_2,y_2)+(x_3,y_3)=\mathcal O
    \end{aligned} \right\}. \]
%
Clearly, $Z$ is in the variety of the ideal
$I\subset\F{p^n}[X_1,Y_1,X_2,Y_2,X_3,Y_3]$ generated by
\[ \left\{\begin{aligned}
      & Y_i^2 - (X_i^3 + aX_i + b),i=1,2,3; \\
      &  (X_3 - X_1)(Y_2 - Y_1) - (X_2 - X_1)(Y_3 - Y_1)\\
    \end{aligned}\right\}. \]
%
Now let $J=I\cap\F{p^n}[X_1,X_2,X_3]$.
%
Using MAGMA's \texttt{EliminationIdeal} function, we obtain that $J$
is actually a principal ideal generated by the polynomial
$(X_2 - X_3)(X_1 - X_3)(X_1 - X_2)f_3$, where
%
\begin{align*}
  f_3 = & X_1^2X_2^2 - 2X_1^2X_2X_3 + X_1^2X_3^2 - 2X_1X_2^2X_3 - 2X_1X_2X_3^2 - 2aX_1X_2 - 2aX_1X_3 \\
        & - 4bX_1 + X_2^2X_3^2 - 2aX_2X_3 - 4bX_2 - 4bX_3 + a^2.
\end{align*}
%
Clearly, the linear factors of this generator correspond to the
degenerated case where two or more points are the same or of opposite
signs, and $f_3$ is the 3rd \emph{summation polynomial}, that is, the
summation polynomial for three distinct points summing to zero.

Starting from the 3rd summation polynomial, we can recursively
construct the subsequent summation polynomials $f_m$ for $m>3$ via
taking resultants.
%
As a result, the degree of each variable in $f_m$ is $2^{m-2}$, which
grows exponentially as $m$.
%
This is the observation Semaev made in his seminal
work~\cite{DBLP:journals/iacr/Semaev04}.
%
In short, his proposal is to consider factor bases of the following
form:
\[ \mathcal F=\Big\{(x,y)\in E(\F{p^n}):x\in V\subset\F{p^n}\Big\}, \]
where $V$ is a subset of \F{p^n}.
%
Then we solve PDP of $m$-th order via solving the corresponding
$(m+1)$-th summation polynomial $f_{m+1}(X_1,\ldots,X_m,\tilde x)=0$,
where $\tilde x$ is the $x$-coordinate of the point to be decomposed.

Note that this factor base is naturally invariant under point
negation.
%
That is, $P_i\in\mathcal F$ implies $-P_i\in\mathcal F$.
%
In this case, we have about $|\mathcal F|/2$ (trivial) relations
$P_i+(-P_i)=\mathcal O$ for free, so we just need to find the other
$|\mathcal F|/2$ nontrivial relations.
%
In general, we will only discuss factor bases that are invariant under
point negation, so by abuse of language, both $\mathcal F$ and
$\mathcal F$ modulo point negation may be referred to as a factor base
in the rest of this paper.

\subsection{Weil restriction}
\label{sec:weil-restriction}
%
Restricting the $x$-coordinates of the points in a factor base to a
subset of \F{p^n} is important from a viewpoint of polynomial system
solving.
%
Take $f_3$ as an example.
%
When decomposing a random point $aP+bQ$, we first substitute its
$x$-coordinate into say $X_3$, projecting the ideal onto
$\F{p^n}[X_1,X_2]$.
%
The dimension of the variety of this ideal is nonzero.
%
Therefore, we would like to pose some restrictions on $X_1$ and $X_2$
to reduce the dimension to zero so that the solving time can be more
manageable.

When looking for solutions to a polynomial
$f=\sum a_iX^i\in\F{p^n}[X]$ in \F{p^n}, we can view $\F{p^n}[X]$ as a
commutative affine algebra
$\mathcal A=\F{p^n}[X]/(X^{p^n} -
X)\cong\F{p^n}[X_1,\ldots,X_n]/(X_1^p - X_1,\ldots,X_n^p - X_n)$.
%
This can be done by identifying the indeterminate $X$ as
$X_1\theta_1+\cdots+X_n\theta_n$, where $(\theta_1,\ldots,\theta_n)$
is a basis for \F{p^n} over \F p.
%
Hence, $f$ can be identified as a polynomial
$f_1\theta_1+\cdots+f_n\theta_n$, where
$f_1,\ldots,f_n\in\mathcal A'=\F p[X_1,\ldots,X_n]/(X_1^p -
X_1,\ldots,X_n^p - X_n)$, by appropriately sending each coefficient
$a_i\in\F{p^n}$ to $a_i^{(1)}\theta_1+\cdots+a_i^{(n)}\theta_n$ for
$a_i^{(1)},\ldots,a_i^{(n)}\in\F p$.
%
Therefore, an equation $f=0$ over \F{p^n} will give rise to a system
of equations $f_1=\cdots=f_n=0$ over \F p.
%
This technique is known as the \emph{Weil restriction} and is used in
the Gaudry-Diem attack, in which the factor base is chosen to consist
of points whose $x$-coordinates lie in a subspace $V$ of \F{p^n} over
\F p~\cite{DBLP:journals/jsc/Gaudry09,DBLP:journals/moc/Diem11}.

\subsection{Exploiting symmetry}
%
\label{sec:exploit-symmetry}
%
Naturally, the symmetric group $S_m$ acts on a point decomposition
$P_1+\ldots+P_m$ because elliptic curve groups are abelian.
%
As noted by Gaudry in his seminal
work~\cite{DBLP:journals/jsc/Gaudry09}, we can therefore rewrite the
variables $x_1,\ldots,x_m\in\F{p^n}$ by elementary symmetric
polynomials $e_1,\ldots,e_m$, where $e_1=\sum x_i$,
$e_2=\sum_{i\neq j}x_ix_j$,
$e_3=\sum_{i\neq j,i\neq k,j\neq k}x_ix_jx_k$, etc.
%
Such rewriting can reduce the degree of summation polynomials and
significantly speed up point
decomposition~\cite{DBLP:conf/eurocrypt/FaugerePPR12,DBLP:conf/iwsec/HuangPST13}.

We might be able to exploit additional symmetry brought by actions of
other groups, e.g., when the factor base is invariant under addition
of small torsion points.
%
For example, consider a decomposition of a point $R$ under the action
of addition of a 2-torsion point $T_2$:
\[ R = P_1+\cdots+P_n =
  (P_1+u_1T_2)+\cdots+(P_{n-1}+u_{n-1}T_2)+\left(P_n+\left(\sum_{i=1}^{n-1}u_i\right)T_2\right). \]
%
Clearly this holds for any $u_1,\ldots,u_{n-1}\in\{0,1\}$, so a
decomposition can give rise to $2^{n-1-1}$ other decompositions.
%
Similar to rewriting using the elementary symmetric polynomials for
the action of $S_m$, we can also take advantage of this additional
symmetry by appropriate
rewriting~\cite{DBLP:journals/joc/FaugereGHR14}.

Naturally, such kind of speed-up is curve-specific.
%
Furthermore, even if the factor base is invariant under additional
group actions, we may or may not be able to exploit such symmetry to
speed up point decomposition depending on whether the action is ``easy
to handle in the polynomial system solving
process''~\cite{DBLP:journals/joc/FaugereGHR14}.


%------------------------------
\subsection{PDP on (twisted) Edwards curves}
%------------------------------------
\label{sec:twisted-edwards}


Faug\`ere, Gaudry, Hout, and Renault studied PDP on twisted Edwards,
twisted Jacobi intersections, and Weierstrass
curves~\cite{DBLP:journals/joc/FaugereGHR14}.
%
For the sake of completeness, we include some of their results here.
%
An Edwards curve over \F{p^n} for $p\neq 2$ is defined by the equation
$x^2+y^2=1+dx^2y^2$ for certain
$d\in\F{p^n}$~\cite{DBLP:journals/iacr/BernsteinL07}.
%
A twisted Edwards curve $tE_{a,d}$ over \F{p^n} for $p\neq 2$ is
defined by the equation $ax^2+y^2=1+dx^2y^2$ for certain
$a,d\in\F{p^n}$~\cite{DBLP:journals/iacr/BernsteinBJLP08}.
%
A twisted Edwards curve is a quadratic twist of an Edwards curve by
$a_0=1/(a-d)$.
%
For $P=(x,y)\in tE_{a,d}$, $-P=(-x,y)$.
%
Furthermore, the addition and doubling formulae for
$(x_3,y_3)=(x_1,y_1)+(x_2,y_2)$ are given as follows.
%
\begin{flalign*}
  \text{When }(x_1,y_1)\neq (x_2,y_2):\left\{\begin{aligned}
      x_3 & = \frac{x_1y_2 + y_1x_2}{1 + dx_1x_2y_1y_2}, \\
      y_3 & = \frac{y_1y_2 - ax_1x_2}{1 - dx_1x_2y_1y_2}.
    \end{aligned}\right. &&
\end{flalign*}
%
\begin{flalign*}
  \text{When }(x_1,y_1)=(x_2,y_2):\left\{\begin{aligned}
      x_3 & = \frac{2x_1y_1}{1 + dx_1^2y_1^2}, \\
      y_3 & = \frac{y_1^2 - ax_1^2}{1 - dx_1^2y_1^2}.
    \end{aligned}\right. &&
\end{flalign*}
%
The 3rd summation polynomial for twisted Edwards curves is~\cite{DBLP:journals/joc/FaugereGHR14}:
%
\begin{align*}
  f_{tE, 3}(Y_1,Y_2,Y_3) = & \left(Y_1^2Y_2^2 - Y_1^2 - Y_2^2 +
                             \frac{a}{d}\right)Y_3^2  \\
                           & + 2\frac{d-a}{d}Y_1Y_2Y_3 +
                             \frac{a}{d}\left(Y_1^2 + Y_2^2 - 1\right)
                             -
                             Y_1^2Y_2^2.
\end{align*}
%
Again subsequent summation polynomials are obtained by taking
resultants.



\subsection{Symmetry and decomposition probability}
\label{sec:symmetry-decomposition-probability}

Symmetry brought by group action on point decomposition will
inevitably be accompanied by \emph{decrease in decomposition
  probability}.
%
For example, if a factor base $\mathcal F$ is invariant under addition
of a 2-torsion point, then the decomposition probability for PDP of
$m$-th order should decrease by a factor of $2^{m-1}$.
%
This is due to the same reason that the decomposition probability
decreases by a factor of $m!$ because the symmetric group $S_m$ acts
on $\mathcal F$.

However, this simple fact seems to have been largely ignored in the
literature.
%
For example, Faug\`ere, Gaudry, Hout, and Renault explicitly stated in
Section~5.3 of their paper that ``[the] probability to decompose a
point [into a sum of $n$ points from the factor base] is
$\frac{1}{n!}$'' for twisted Edwards or twisted Jacobi intersections
curves, despite the fact that the factor base is invariant under
addition of 2-torsion points~\cite{DBLP:journals/joc/FaugereGHR14}.
%
At a first glance, this may not seem a problem, as we would expect to
obtain $2^{n-1}$ solutions if we can successfully solve a PDP
instance.
%
(Unfortunately this is also \emph{not true} in general.  We will come
back to it in more detail in Section~\ref{sec:price}.)
% 
However, when estimating the cost of a complete ECDLP attack, they
proposed to \emph{collapse} these $2^{n-1}$ relations into one in
order to reduce the size of the factor base and thus the cost of the
linear algebra, cf.~Remark 5 of the paper.
%
In this case, the decrease in decomposition probability \emph{does}
have an adverse effect, and their estimation for the overall ECDLP
cost ended up being overoptimistic by a factor of at least $2^{n-1}$.

