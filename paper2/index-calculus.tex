%
% Index calculus for ECDLP
%

\section{Index calculus for ECDLP}
%
\label{sec:index-calculus-ecdlp}
%
Let $E$ be an elliptic curve defined over a finite field \F{p^n}.
%
For cryptographic applications, we are mostly interested in a
prime-order subgroup generated by a rational point $P\in E(\F{p^n})$.
%
To find an integer $\alpha$ such that $Q=\alpha P$ for
$Q\in\langle P\rangle$ using an index calculus algorithm, one
typically works as follows.
%
\begin{enumerate}
%
\item Determine a \emph{factor base} $\mathcal F\subset E(\F{p^n})$.
%
\item Collect a set $\mathcal R$ of \emph{relations} by decomposing
  random points $a_iP+b_iQ$ into a sum of points from $\mathcal F$,
  i.e.,
  \[ \mathcal
    R=\left\{a_iP+b_iQ=\sum_{j=1}^mP_{i,j}:P_{i,j}\in\mathcal
      F\right\} \]
%
\item When $|\mathcal R|\approx|\mathcal F|$, eliminate the righthand
  side using linear algebra to obtain an equation in the form
  $aP+bQ=\mathcal O$, and $\alpha=-a/b\bmod\ord(P)$.
%
\end{enumerate} 

\subsection{Semaev's summation polynomials}
%
\label{sec:summation-polynomials}
%
As we can see, an important step in index calculus algorithms for
solving ECDLP is point decomposition on an elliptic curve.
%
It is straightforward that if two points sum to zero on a Weierstrass
curve, then their $x$-coordinates must be equal.
%
Let us now consider the simplest nontrivial case where three points on
a Weierstrass curve $y^2=x^3+ax+b$ sum to the point at infinity.
%
Let
\[ Z=\left\{\begin{aligned}
      (x_1,y_1,x_2,y_2,x_3,y_3)&\in\F{p^n}^6:(x_i,y_i)\in E(\F{p^n}),i=1,2,3; \\
      & (x_1,y_1)+(x_2,y_2)+(x_3,y_3)=\mathcal O
    \end{aligned} \right\}. \]
%
Clearly, $Z$ is in the variety of the ideal
$I\subset\F{p^n}[X_1,Y_1,X_2,Y_2,X_3,Y_3]$ generated by
\[ \left\{\begin{aligned}
      &  (X_3 - X_1)(Y_2 - Y_1) - (X_2 - X_1)(Y_3 - Y_1),\\
      & Y_i^2 - (X_i^3 + aX_i + b),i=1,2,3
    \end{aligned}\right\}. \]
%
Now let $J=I\cap\F{p^n}[X_1,X_2,X_3]$.
%
Using MAGMA's \texttt{EliminationIdeal} function, we obtain that $J$
is actually a principal ideal generated by the polynomial
$(X_2 - X_3)(X_1 - X_3)(X_1 - X_2)f_3$, where
%
\begin{align*}
  f_3 = & X_1^2X_2^2 - 2X_1^2X_2X_3 + X_1^2X_3^2 - 2X_1X_2^2X_3 - 2X_1X_2X_3^2 - 2aX_1X_2 - 2aX_1X_3 \\
        & - 4bX_1 + X_2^2X_3^2 - 2aX_2X_3 - 4bX_2 - 4bX_3 + a^2.
\end{align*}
%
Clearly, the linear factors of the generator correspond to the
degenerated case where two or more points are the same or of opposite
signs, and $f_3$ is the 3rd \emph{summation polynomial}, that is, the
summation polynomial for three distinct points summing to zero.

Starting from the 2nd and 3rd summation polynomials, one can
recursively obtain the subsequent summation polynomials via taking
resultants.
%
This is the observation Semaev made in his seminal
work~\cite{DBLP:journals/iacr/Semaev04}.
%
In short, his proposal is to consider factor bases of the following
form:
\[ \mathcal F=\Big\{(x,y)\in E(\F{p^n}):x\in
  V\subset\F{p^n})\Big\}, \] where $V$ is a subset of \F{p^n}.
%
Note that this factor base is invariant under negation.
%
That is, $P_i\in\mathcal F$ implies $-P_i\in\mathcal F$.
%
In this case, we will have about $|\mathcal F|/2$ (trivial) relations
$P_i+(-P_i)=\mathcal O$ for free, so we just need to find about
$|\mathcal F|/2$ nontrivial relations.
%
In general, we will only discuss factor bases that are invariant under
negation, so by abuse of language, both $\mathcal F$ and $\mathcal F$
modulo negation may be referred to as a factor base in the rest of
this paper.

\subsection{Weil restriction}
%
Restricting $x$-coordinates of the points in factor base to a subset
of \F{p^n} is important from a viewpoint of polynomial system solving.
%
Take $f_3$ as an example.
%
When decomposing a random point $aP+bQ$, we first substitute its
$x$-coordinate into say $X_3$, projecting $J$ onto $\F{p^n}[X_1,X_2]$.
%
The dimension of the variety of this ideal is nonzero.
%
Therefore, we would like to pose some restrictions on $X_1$ and $X_2$
to reduce the dimension to zero and make the solving time more
manageable.

When looking for solutions to a polynomial in $\F{p^n}[X]$ in \F{p^n},
we can view it as a commutative affine algebra
$\mathcal A=\F{p^n}[X]/(X^{p^n} -
X)\cong\F{p^n}[X_1,\ldots,X_n]/(X_1^p - X_1,\ldots,X_n^p - X_n)$.
%
This can be done by identifying the indeterminate $X$ as
$X_1\theta_1+\cdots+X_n\theta_n$, where $(\theta_1,\ldots,\theta_n)$
is a basis for \F{p^n} over \F p.
%
Hence, a polynomial $f=\sum a_iX^i\in\F{p^n}[X]$ can be identified as
a polynomial $f_1\theta_1+\cdots+f_n\theta_n$, where
$f_1,\ldots,f_n\in\mathcal A'=\F p[X_1,\ldots,X_n]/(X_1^p -
X_1,\ldots,X_n^p - X_n)$, by appropriately sending any coefficient
$a_i\in\F{p^n}$ to $a_i^{(1)}\theta_1+\cdots+a_i^{(n)}\theta_n$ for
$a_i^{(1)},\ldots,a_i^{(n)}\in\F p$.
%
Therefore, an equation $f=0$ over \F{p^n} will give rise to a system
of equations $f_1=\cdots=f_n=0$ over \F p.
%
This technique is known as the \emph{Weil restriction} and is used in
the Gaudry-Diem attack, in which the factor base is chosen to consist
of points whose $x$-coordinates lie in a subspace $V$ of \F{p^n} over
\F p~\cite{DBLP:journals/jsc/Gaudry09,DBLP:journals/moc/Diem11}.

\subsection{Exploiting symmetry}
%
\label{sec:exploit-symmetry}
%
Naturally, the symmetric group $S_m$ acts on a point decomposition
$P_1+\ldots+P_m$ because elliptic curve groups are abelian.
%
As noted by Gaudry in his seminal
work~\cite{DBLP:journals/jsc/Gaudry09}, we can therefore rewrite the
variables $x_1,\ldots,x_m\in\F{p^n}$ by elementary symmetric
polynomials $e_1,\ldots,e_m$, where $e_1=\sum x_i$,
$e_2=\sum_{i\neq j}x_ix_j$,
$e_3=\sum_{i\neq j,i\neq k,j\neq k}x_ix_jx_k$, etc.
%
Such rewriting can reduce the degree of summation polynomials and
significantly speed up point
decomposition~\cite{DBLP:conf/eurocrypt/FaugerePPR12,DBLP:conf/iwsec/HuangPST13}.

One might be able to exploit additional symmetry using actions of
other groups, e.g., when the factor base is invariant under addition
of small torsion points~\cite{DBLP:conf/eurocrypt/FaugereHJRV14}.
%
For example, consider a relation $R$ under the action of addition of a
2-torsion point $T_2$:
\begin{align*}
  R & = P_1+\cdots+P_n \\
    & =
      (P_1+u_1T_2)+\cdots+(P_{n-1}+u_{n-1}T_2)+\left(P_n+\left(\sum_{i=1}^{n-1}u_i\right)T_2\right).
\end{align*}
%
Clearly this holds for any $u_1,\ldots,u_{n-1}\in\{0,1\}$, and
$P_i\in\mathcal F$ implies that $P_i+u_iT_2\in\mathcal F$ when the
factor base $\mathcal F$ is invariant under addition of $T_2$.

Of course such kind of speed-up is curve-specific in nature.
%
Furthermore, even if the factor base is invariant under addition of
small torsion points, we may or may not be able to exploit such
symmetry to speed up point decomposition depending on whether the
action of addition of these small torsion points is ``easy to handle
in the polynomial system solving
process''~\cite{DBLP:conf/eurocrypt/FaugereHJRV14}.

Finally, this potential speed-up will be counteracted by
\emph{decrease in decomposition probability}, which seems to have been
largely ignored in the literature.
%
For example, Faug\`ere, Gaudry, Hout, and Renault \emph{did not} take
it into account in their estimation of the cost of a complete ECDLP
attack on twisted Edwards or twisted Jacobi intersections
curves~\cite{DBLP:conf/eurocrypt/FaugereHJRV14}.
%
Specifically in the Section 5.3 of their paper, they explicitly stated
that ``[the] probability to decompose a point [into a sum of $n$
points from the factor base] is $\frac{1}{n!}$'' for curves in various
forms, regardless of whether the factor base is invariant under
addition of small torsion points.
%
However, this decomposition probability will decrease due to the
symmetry brought by additional group action on a factor base
$\mathcal F$, similar to the fact that the probability decreases by a
factor of $n!$ due to the symmetry of the symmetric group $S_n$ acting
on $\mathcal F$.
%
For example, the decomposition probability should further decrease by
a factor of $2^{n-1}$ in the earlier example of this section.
