\section{Experiments on PDP solving}
\label{sec:experiment}

This section shows our experimental results, which we have conducted to
compare the computational complexity of PDP on four different curves:
Hessian($H$), Weierstrass($W$), Montgomery($M$), and twisted
Edwards($tE$).

\subsection{Experimental setup}
\label{subsec:conditions}

%
% Isomorphisms among curves in various forms
%

% \section{Isomorphisms among curves in various forms}
% \label{sec:isomorphism}

%
To make a fair comparison, we use curves in different forms but are
nonetheless isomorphic to one another over \F{p^n}.
%
That is, $H(\F{p^n})\cong W(\F{p^n})\cong M(\F{p^n})\cong tE(\F{p^n})$
as groups, and we consider ECDLP in the same largest prime-order
subgroup.
%
We will also explicitly state whether the factor base is invariant
under addition of 2-torsion points for each of the four forms under
investigation.

We start from a Hessian curve $H_d$ satisfying $x^3 + y^3 + 1 = 3dxy$
for $d\in\F{p^n}$ such that the number of its rational points
$\#H_d(\F{p^n})$ is divisible by 12.
%
As we have seen in Section~\ref{sec:hessian-t2}, the factor base of
$H_d$ is in general not invariant under addition of 2-torsion points.
%
From $H_d$, we can obtain an isomorphic Weierstrass curve $W_{a,b}$
satisfying $y^2 = x^3 + ax + b$ for $a = - 27d(d^3 + 8)$ and
$b = 54(d^6 - 20d^3 - 8)$~\cite{DBLP:conf/ches/Smart01}.
%
The isomorphism $\phi_{W,H}$ from $W_{a,b}(\F{p^n})$ to $H_d(\F{p^n})$
is defined over $\F{p^n}$ and is given by sending $(u,v)\in W_{a,b}$
to $(x,y)\in H_d$, where
\[ \left\{\begin{aligned}
      x = & \frac{36(d^3 - 1) - v}{6(u + 9d^2)} - \frac{d}{2}, \\
      y = & \frac{36(d^3 - 1) + v}{6(u + 9d^2)} - \frac{d}{2}.
    \end{aligned}\right. \text{The inverse $\phi_{H,W}$ is given by }
  \left\{\begin{aligned}
      u = & \frac{12(d^3 - 1)}{d + x + y} - 9d^2, \\
      v = & \frac{36(d^3 - 1)(y - x)}{d + x + y}.
    \end{aligned}\right. \]
%
The factor base of $W_{a,b}$ is in general not invariant under
addition of 2-torsion points~\cite{DBLP:journals/joc/FaugereGHR14}.

With a high probability, we can obtain a Montgomery curve $M_{A,B}$
satisfying $By^2 = x^3 + Ax^2 + x$ from $W_{a,b}$ by solving the
following equations
%
\[ \left\{\begin{aligned}
a = & \frac{3 - A^2}{3B^2}, \\
b = & \frac{2A^3 - 9A}{27B^3}.
\end{aligned}\right. \]
%
The isomorphism $\phi_{W,M}$ is defined over $\F{p^n}$ and is given by
sending $(u,v)\in W_{a,b}$ to $(x,y)\in M_{A,B}$ for $x = Bu - 1/3A$
and $y = Bv$.
%
The inverse $\phi_{M,W}$ can be obtained by equation solving.
%
As we have seen in Section~\ref{sec:montgomery-symmetry}, the factor
base is invariant under addition of a particular 2-torsion point
$(0,0)$, though we are not able to exploit this symmetry in general.

Finally, we can obtain a twisted Edwards curve $tE_{a',d'}$ satisfying
$a'x^2 + y^2 = 1 + d'x^2y^2$ from $M_{A,B}$ by taking $a' = (A + 2)/B$
and $d' = (A - 2)/B$.
%
Again we let $a_0=1/(a' - d')$ be the amount of quadratic twist.
%
The isomorphism $\phi_{W,tE}$ is defined over $\F{p^n}$ and given by
sending $(u,v)\in W_{a,b}$ to $(x,y)\in tE_{a',d'}$, where
\[ \left\{\begin{aligned}
      x = & \frac{2a_0u}{v}, \\
      y = & \frac{u - a_0}{u + a_0}.
    \end{aligned}\right. \text{The inverse $\phi_{tE,W}$ is given by }
  \left\{\begin{aligned}
      u = & \frac{a_0(1 + y)}{1 - y}, \\
      v = & \frac{2a_0^2(1 + y)}{x(1 - y)}.
    \end{aligned}\right. \]
% 
As shown by Faug\`ere, Gaudry, Hout, and
Renault~\cite{DBLP:journals/joc/FaugereGHR14}, the factor base is
invariant under addition of the 2-torsion point $(0,-1)$.


As explained in Section~\ref{sec:index-calculus-ecdlp}, we focus on
PDP in these experiments, as the linear algebra step is already well
understood.
%
Furthermore, we focus on the bottleneck computation in PDP, namely,
the cost of the F4 algorithm for computing Gr\"obner bases of the
polynomial systems obtained after rewriting using the elementary
symmetric polynomials and applying the Weil restriction technique to
summation polynomials.
%
This way we will be taking advantage of the symmetry of $S_m$ acting
on point decompositions.
% 
However, we \emph{did not} exploit symmetry of any other group
actions.
%
This is because we want to compare the \emph{intrinsic} computational
complexity of PDP and hence only consider the symmetry that is present
in \emph{all} curves.
%
Exploiting further curve-specific symmetry whenever possible will
result in further speed-up, but it would be independent of our
findings here.


\subsection{Experimental results}
%
\label{sec:experiment-result}

Table\ref{tb:pdpsolving} presents our experimental results for the case of $n=5$.
%
Here we choose our factor base by taking $V$ as the base field \F p of
\F{p^n}.
%
All our experiment are done using the MAGMA computation algebra system
(version 2.23-1) on a single core of an Intel Xeon CPU E7-4830 v4
running at 2~GHz.
%
Comparisons to solve each PDP are done by running time (in second),
Dreg, Matcost, and Rank.
%
the ``Dreg'' is the maximum step degree reached during the execution of the F4 algorithm, 
which referred to as the ``degree of regularity''
in the literature~\cite{DBLP:conf/indocrypt/GalbraithG14}, and
provides an upper bound for the sizes of the Macaulay submatrices
involved in the computation, 
%
the ``Matcost'' is a number output by the MAGMA implementation of the
F4 algorithm and provides an estimate of the linear algebra cost
during the execution of the F4 algorithm,
%
and finally the ``Rank'' is the number of linearly independent
relations we obtain once successfully solving a PDP instance.
%
It is an important factor to consider, as it determines how many PDP
instances we need to successfully solve in order to have enough
relations for a complete ECDLP attack using index calculus.


\begin{table}[!h]
\centering
\caption{$m=3$}
\label{tb:m=3}
\begin{tabular}{llrrrr}
\hline
\multicolumn{1}{|c|}{$q$}                    & \multicolumn{1}{l|}{Curve}       & \multicolumn{1}{l|}{Time} & \multicolumn{1}{l|}{Dreg} & \multicolumn{1}{l|}{Matcost} & \multicolumn{1}{l|}{Rank} \\ \hline
\multicolumn{1}{|l|}{\multirow{4}{*}{251}} & \multicolumn{1}{l|}{Hessian}     & \multicolumn{1}{r|}{0}    & \multicolumn{1}{r|}{6}    & \multicolumn{1}{r|}{41420.4} & \multicolumn{1}{r|}{1}    \\ \cline{2-6} 
\multicolumn{1}{|l|}{}                     & \multicolumn{1}{l|}{Weierstrass} & \multicolumn{1}{r|}{0}    & \multicolumn{1}{r|}{6}    & \multicolumn{1}{r|}{42132.0} & \multicolumn{1}{r|}{1}    \\ \cline{2-6} 
\multicolumn{1}{|l|}{}                     & \multicolumn{1}{l|}{Montgomery}  & \multicolumn{1}{r|}{0}    & \multicolumn{1}{r|}{6}    & \multicolumn{1}{r|}{61127.9} & \multicolumn{1}{r|}{4}    \\ \cline{2-6} 
\multicolumn{1}{|l|}{}                     & \multicolumn{1}{l|}{tEdwards}    & \multicolumn{1}{r|}{0}    & \multicolumn{1}{r|}{6}    & \multicolumn{1}{r|}{6308.4}  & \multicolumn{1}{r|}{4}    \\ \hline \vspace{-3mm}
                                           &                                  & \multicolumn{1}{l}{}      & \multicolumn{1}{l}{}      & \multicolumn{1}{l}{}         & \multicolumn{1}{l}{}      \\ \hline
\multicolumn{1}{|l|}{\multirow{4}{*}{239}} & \multicolumn{1}{l|}{Hessian}     & \multicolumn{1}{r|}{0}    & \multicolumn{1}{r|}{6}    & \multicolumn{1}{r|}{42336.8} & \multicolumn{1}{r|}{1}    \\ \cline{2-6} 
\multicolumn{1}{|l|}{}                     & \multicolumn{1}{l|}{Weierstrass} & \multicolumn{1}{r|}{0}    & \multicolumn{1}{r|}{6}    & \multicolumn{1}{r|}{41259}   & \multicolumn{1}{r|}{1}    \\ \cline{2-6} 
\multicolumn{1}{|l|}{}                     & \multicolumn{1}{l|}{Montgomery}  & \multicolumn{1}{r|}{0}    & \multicolumn{1}{r|}{6}    & \multicolumn{1}{r|}{61239}   & \multicolumn{1}{r|}{4}    \\ \cline{2-6} 
\multicolumn{1}{|l|}{}                     & \multicolumn{1}{l|}{tEdwards}    & \multicolumn{1}{r|}{0}    & \multicolumn{1}{r|}{6}    & \multicolumn{1}{r|}{6308.36} & \multicolumn{1}{r|}{4}    \\ \hline
\end{tabular}
\end{table}

\begin{table}[!h]
\centering
\caption{$m=4$}
\label{tb:m=4}
\begin{tabular}{clrrrr}
\hline
\multicolumn{1}{|c|}{$q$}                  & \multicolumn{1}{l|}{Curve}       & \multicolumn{1}{l|}{Time}  & \multicolumn{1}{l|}{Dreg} & \multicolumn{1}{l|}{Matcost}     & \multicolumn{1}{l|}{Rank} \\ \hline
\multicolumn{1}{|l|}{\multirow{4}{*}{251}} & \multicolumn{1}{l|}{Hessian}     & \multicolumn{1}{r|}{3.459} & \multicolumn{1}{r|}{19}   & \multicolumn{1}{r|}{12069800000} & \multicolumn{1}{r|}{1}    \\ \cline{2-6} 
\multicolumn{1}{|l|}{}                     & \multicolumn{1}{l|}{Weierstrass} & \multicolumn{1}{r|}{3.659} & \multicolumn{1}{r|}{19}   & \multicolumn{1}{r|}{12066400000} & \multicolumn{1}{r|}{1}    \\ \cline{2-6} 
\multicolumn{1}{|l|}{}                     & \multicolumn{1}{l|}{Montgomery}  & \multicolumn{1}{r|}{3.280} & \multicolumn{1}{r|}{18}   & \multicolumn{1}{r|}{11401700000} & \multicolumn{1}{r|}{5}    \\ \cline{2-6} 
\multicolumn{1}{|l|}{}                     & \multicolumn{1}{l|}{tEdwards}    & \multicolumn{1}{r|}{0.119} & \multicolumn{1}{r|}{18}   & \multicolumn{1}{r|}{54102900}    & \multicolumn{1}{r|}{5}    \\ \hline
 \vspace{-3mm}                                          &                                  & \multicolumn{1}{l}{}       & \multicolumn{1}{l}{}      & \multicolumn{1}{l}{}             & \multicolumn{1}{l}{}      \\ \hline
\multicolumn{1}{|l|}{\multirow{4}{*}{239}} & \multicolumn{1}{l|}{Hessian}     & \multicolumn{1}{r|}{3.990} & \multicolumn{1}{r|}{19}   & \multicolumn{1}{r|}{12066100000} & \multicolumn{1}{r|}{1}    \\ \cline{2-6} 
\multicolumn{1}{|l|}{}                     & \multicolumn{1}{l|}{Weierstrass} & \multicolumn{1}{r|}{3.680} & \multicolumn{1}{r|}{19}   & \multicolumn{1}{r|}{12064700000} & \multicolumn{1}{r|}{1}    \\ \cline{2-6} 
\multicolumn{1}{|l|}{}                     & \multicolumn{1}{l|}{Montgomery}  & \multicolumn{1}{r|}{3.489} & \multicolumn{1}{r|}{18}   & \multicolumn{1}{r|}{11399100000} & \multicolumn{1}{r|}{5}    \\ \cline{2-6} 
\multicolumn{1}{|l|}{}                     & \multicolumn{1}{l|}{tEdwards}    & \multicolumn{1}{r|}{0.150} & \multicolumn{1}{r|}{18}   & \multicolumn{1}{r|}{54093000}    & \multicolumn{1}{r|}{5}    \\ \hline
\end{tabular}
\end{table}


% \begin{table}[!h]
% \centering
% \caption{$m=3$}
% \label{tb:m=3}
% \begin{tabular}{llrrr}
% \hline
% \multicolumn{1}{|l|}{$q$}                  & \multicolumn{1}{l|}{Curve}       & \multicolumn{1}{l|}{Time} & \multicolumn{1}{l|}{Dreg} & \multicolumn{1}{l|}{Matcost} \\ \hline
% \multicolumn{1}{|l|}{\multirow{4}{*}{251}} & \multicolumn{1}{l|}{Hessian}     & \multicolumn{1}{r|}{0}    & \multicolumn{1}{r|}{6}    & \multicolumn{1}{r|}{41420.4} \\ \cline{2-5} 
% \multicolumn{1}{|l|}{}                     & \multicolumn{1}{l|}{Weierstrass} & \multicolumn{1}{r|}{0}    & \multicolumn{1}{r|}{6}    & \multicolumn{1}{r|}{42132.0} \\ \cline{2-5} 
% \multicolumn{1}{|l|}{}                     & \multicolumn{1}{l|}{Montgomery}  & \multicolumn{1}{r|}{0}    & \multicolumn{1}{r|}{6}    & \multicolumn{1}{r|}{61127.9} \\ \cline{2-5} 
% \multicolumn{1}{|l|}{}                     & \multicolumn{1}{l|}{tEdwards}    & \multicolumn{1}{r|}{0}    & \multicolumn{1}{r|}{6}    & \multicolumn{1}{r|}{6308.4}  \\ \hline \vspace{-3mm}
%                                            &                                  &                           &                           &                              \\ \hline
% \multicolumn{1}{|l|}{\multirow{4}{*}{239}} & \multicolumn{1}{l|}{Hessian}     & \multicolumn{1}{r|}{0}    & \multicolumn{1}{r|}{6}    & \multicolumn{1}{r|}{42336.8} \\ \cline{2-5} 
% \multicolumn{1}{|l|}{}                     & \multicolumn{1}{l|}{Weierstrass} & \multicolumn{1}{r|}{0}    & \multicolumn{1}{r|}{6}    & \multicolumn{1}{r|}{41259.0} \\ \cline{2-5} 
% \multicolumn{1}{|l|}{}                     & \multicolumn{1}{l|}{Montgomery}  & \multicolumn{1}{r|}{0}    & \multicolumn{1}{r|}{6}    & \multicolumn{1}{r|}{61239.0} \\ \cline{2-5} 
% \multicolumn{1}{|l|}{}                     & \multicolumn{1}{l|}{tEdwards}    & \multicolumn{1}{r|}{0}    & \multicolumn{1}{r|}{6}    & \multicolumn{1}{r|}{6308.4}  \\ \hline
% \end{tabular}
% \end{table}


% \begin{table}[!h]
% \centering
% \caption{$m=4$}
% \label{tb:m=4}
% \begin{tabular}{llrrr}
% \hline
% \multicolumn{1}{|l|}{$q$}                  & \multicolumn{1}{l|}{Curve}       & \multicolumn{1}{l|}{Time}  & \multicolumn{1}{l|}{Dreg} & \multicolumn{1}{l|}{Matcost}     \\ \hline
% \multicolumn{1}{|l|}{\multirow{4}{*}{251}} & \multicolumn{1}{l|}{Hessian}     & \multicolumn{1}{r|}{3.459} & \multicolumn{1}{r|}{19}   & \multicolumn{1}{r|}{12069800000} \\ \cline{2-5} 
% \multicolumn{1}{|l|}{}                     & \multicolumn{1}{l|}{Weierstrass} & \multicolumn{1}{r|}{3.659} & \multicolumn{1}{r|}{19}   & \multicolumn{1}{r|}{12066400000} \\ \cline{2-5} 
% \multicolumn{1}{|l|}{}                     & \multicolumn{1}{l|}{Montgomery}  & \multicolumn{1}{r|}{3.280} & \multicolumn{1}{r|}{18}   & \multicolumn{1}{r|}{11401700000} \\ \cline{2-5} 
% \multicolumn{1}{|l|}{}                     & \multicolumn{1}{l|}{tEdwards}    & \multicolumn{1}{r|}{0.119} & \multicolumn{1}{r|}{18}   & \multicolumn{1}{r|}{54102900}    \\ \hline \vspace{-3mm}
%                                            &                                  &                            &                           &                                  \\ \hline
% \multicolumn{1}{|l|}{\multirow{4}{*}{239}} & \multicolumn{1}{l|}{Hessian}     & \multicolumn{1}{r|}{3.990} & \multicolumn{1}{r|}{19}   & \multicolumn{1}{r|}{12066100000} \\ \cline{2-5} 
% \multicolumn{1}{|l|}{}                     & \multicolumn{1}{l|}{Weierstrass} & \multicolumn{1}{r|}{3.680} & \multicolumn{1}{r|}{19}   & \multicolumn{1}{r|}{12064700000} \\ \cline{2-5} 
% \multicolumn{1}{|l|}{}                     & \multicolumn{1}{l|}{Montgomery}  & \multicolumn{1}{r|}{3.489} & \multicolumn{1}{r|}{18}   & \multicolumn{1}{r|}{11399100000} \\ \cline{2-5} 
% \multicolumn{1}{|l|}{}                     & \multicolumn{1}{l|}{tEdwards}    & \multicolumn{1}{r|}{0.150} & \multicolumn{1}{r|}{18}   & \multicolumn{1}{r|}{54093000}    \\ \hline
% \end{tabular}
% \end{table}

We can clearly see that the PDP solving time and Matcost for twisted
Edwards curves are much smaller than those for the other curves.
%
In contrast, the degree of regularity for Montgomery and twisted
Edwards curves is smaller than that of the other curves in the case of
$m=4$.
%
Also, we can see that the rank for Hessian and Weierstrass curves is 1
in all cases, whereas for Montgomery and twisted Edwards curves, it is
4 and 5 in the case of $m=3$ and $m=4$, respectively.
%
Last but not least, although we only present the results for small $p$
(around 8-bit long) here, we have some preliminary results for larger
$p$ (around 16-bit and 32-bit long).
%
Aside from slight difference in absolute running time, all other
results such as Dreg, Matcost, and Rank are similar, so we do not
repeat them here.
%
% We further analyze which terms are missing from the summation
% polynomials for curves in different forms.
% %
% We classify terms into odd vs.~even degrees.
% %
% Recall that the polynomials are expressed using elementary symmetric polynomials 
% $e_1, \dots ,e_m$.
% %
% Therefore, a monomial $e_i^j$ has odd degree only if both $i$ and $j$
% are odd.
% %
% For $m=2$, it is interesting that there are no monomials of odd degree
% only for twisted Edwards curve.
% %
% In other words, there are no terms originated from monomial such as
% $e_1$ or $e_1e_2$.
% %
% This is because the addition formula for twisted Edwards curve, based
% on which the summation polynomial is derived, consists of only even
% terms.
% %
% Therefore, when one of the 3 variables in the summation polynomial is
% substituted with the point we want to decompose, the odd-degree terms
% all vanish.

% For $m=3$, we use summation polynomial with 4 variables, which is
% obtained by taking the resultant of two summation polynomials 
% with 3 variables.
% %
% Therefore, there are some odd terms such as $e_3$ or $e_2^2e_3$.
% %
% For $m=4$, there is also no terms with odd degree.
% %
% Through calculating resultant recursively, two variables are
% eliminated, and it preserves all terms to even degree.






