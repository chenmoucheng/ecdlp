%
% Montgomery curves
%

\subsection{Montgomery curves}
\label{sec:montgomery-symmetry}
%
A Montgomery curve $M_{A,B}$ over \F{p^n} for $p\neq 2$ is defined by
the equation \begin{equation}
  By^2=x^3+Ax^2+x \label{eq:montgomery-curve} \end{equation} for
$A,B\in\F{p^n}$ such that $A\neq\pm 2$, $B\neq 0$, and
$B(A^2-4)\neq 0$~\cite{1987-montgomery}.
%
For $P=(x,y)\in M_{A,B}$, $-P=(x,-y)$.
%
Furthermore, the addition and doubling formulae for
$(x_3,y_3)=(x_1,y_1)+(x_2,y_2)$ are given as follows.
%
When $(x_1,y_1)\neq(x_2,y_2)$: \[ \left\{\begin{aligned}
      x_3 & = B\left(\frac{y_2 - y_1} {x_2 - x_1}\right)^2 - A - x_1 - x_2 = \frac{B(x_2y_1 - x_1y_2)^2} {x_1x_2(x_2 - x_1)^2}, \\
      y_3 & = \frac{(2x_1 + x_2 + A)(y_2 - y_1)} {x_2 - x_1} -
      \frac{B(y_2 - y_1)^3} {(x_2 - x_1)^3} - y_1.
    \end{aligned}\right. \]
%
When $(x_1,y_1)=(x_2,y_2)$: \[ \left\{\begin{aligned}
      x_3 & = \frac{(x_1^2 - 1)^2} {4x_1(x_1^2 + Ax_1 + 1)},  \\
      y_3 & = \frac{(2x_1 + x_1 + A)(3x_1^2 + 2Ax_1 + 1)} {2By_1} -
      \frac{B(3x_1^2 + 2Ax_1 + 1)^3} {(2By_1)^3} - y_1.
  \end{aligned}\right. \]
%
It was noted by Montgomery himself in his original paper that such
curves can give rise to efficient scalar multiplication
algorithms~\cite{1987-montgomery}.
%
That is, consider a random point $P\in M_{A,B}(\F{p^n})$ and
$nP=(X_n:Y_n:Z_n)$ in projective coordinates for some integer $n$.
%
Then:
%
\[ \left\{\begin{aligned}
      X_{m+n} & = Z_{m-n}[(X_m - Z_m)(X_n + Z_n) + (X_m + Z_m)(X_n - Z_n)]^2, \\
      Z_{m+n} & = X_{m-n}[(X_m - Z_m)(X_n + Z_n) - (X_m + Z_m)(X_n -
      Z_n)]^2.
    \end{aligned}\right. \]
%
In particular, when $m=n$:
\[ \left\{\begin{aligned}
      X_{2n} & = (X_n + Z_n)^2(X_n - Z_n)^2, \\
      Z_{2n} & = (4X_nZ_n)\left((X_n - Z_n)^2 + ((A+2)/4)(4X_nZ_n)\right), \\
      4X_nZ_n & = ( X_n + Z_n)^2 - (X_n - Z_n)^2.
    \end{aligned}\right. \]
%
In this way, scalar multiplication on Montgomery curve can be
performed without using $y$-coordinates, leading to fast
implementation.

\subsection{Summation polynomials for Montgomery curves}

Following Semaev's approach~\cite{DBLP:journals/iacr/Semaev04}, we can
construct summation polynomials for Montgomery curves.
%
Like Weierstrass curves, the 2nd summation polynomial for Montgomery
curves is simply $f_{M,2} = X_1 - X_2$.
%
Now consider $P,Q\in M_{A, B}$ for $P=(x_1, y_1)$ and $Q=(x_2, y_2)$.
%
Let $P+Q=(x_3, y_3)$ and $P-Q=(x_4, y_4)$.
%
By the addition formula, we have
\[ x_3 = \frac{B(x_2y_1 - x_1y_2)^2} {x_1x_2(x_2 - x_1)^2},
  x_4 =\frac{B(x_2y_1 - x_1y_2)^2} {x_1x_2(x_2 + x_1)^2}. \]
%
It follows that
%
\[ \left\{\begin{aligned} x_3 + x_4&=\frac{2\left((x_1 + x_2)(x_1x_2 + 1) + 2Ax_1x_2\right)}{(x_1 - x_2)^2}, \\
      x_3x_4&=\frac{(1 - x_1x_2)^2}{(x_1 - x_2)^2}.
    \end{aligned}\right. \]
%
Using the relationship between the roots of a quadratic polynomial and
its coefficients, we obtain
\[ (x_1 - x_2)^2x^2 - 2\left((x_1 + x_2)(x_1x_2 + 1) +
    2Ax_1x_2\right)x + (1 - x_1x_2)^2. \]
%
From here, we can obtain for Montgomery curve the 3rd summation
polynomial:
\[ f_{M,3}(X_1,X_2,X_3) = (X_1 - X_2)^2X_3^2 - 2\left((X_1 +
    X_2)(X_1X_2 + 1) + 2AX_1X_2\right)X_3 + (1-X_1X_2)^2, \]
%
as well as the subsequent summation polynomials via taking resultants:
\[ f_{M,m}(X_1,\ldots,X_m) =
  \res_X\left(f_{M,m-k}(X_1,\ldots,X_{m-k-1},X),f_{M,k+2}(X_{m-k},\ldots,X_m,X)\right). \]

  
%--------------------------------
\subsection{Small torsion points on Montgomery
  curves} \label{subsec:TSPL}
%------------------------
A Montgomery curve always contains an affine 2-torsion point $T_2$.
%
Since $T_2+T_2=2T_2=\mathcal O$, it follows that $-T_2=T_2$.
%
If we write $T_2=(x,y)$, then we can see that $y=0$ in order for
$-T_2=T_2$, as $p\neq 2$.
%
Substituting $y=0$ into Equation~(\ref{eq:montgomery-curve}),
we get an equation $x^3+Ax^2+x=0$.
%
The lefthand side factors into $x(x^2+Ax+1)=0$, so we get \[
  x=0,\frac{-A\pm\sqrt{A^2 - 4}}{2}. \]
%
Therefore, the set of rational points over the definition field
$F_{p^n}$ of a Montgomery curve includes at least two 2-torsion
points, namely, $\mathcal O$ and $(0,0)$.
%
The other 2-torsion points may or may not be rational, so we will
focus on $(0,0)$ in this paper.
%
Substituting $(x_2,y_2)=(0,0)$ into the addition formula for
Montgomery curves, we get that for any point $P=(x,y)\in M_{A,B}$,
$P+(0,0)=(1/x,-y/x^2)$.

To be able to exploit the symmetry of addition of $T_2=(0,0)$, we need
to choose the factor base
$\mathcal F=\{(x,y)\in E(\F{p^n}):x\in V\subset\F{p^n}\}$ invariant
under addition of $T_2$.
%
This means that $V$ needs to be closed under taking multiplicative
inverses.
%
In other words, $V$ needs to be a \emph{subfield} of $\F{p^n}$, i.e.,
$V=\F{p^\ell}$ for some integer $\ell$ that divides $n$.
%
In this case, $f_m$ is invariant under the action of
$x_i\mapsto 1/x_i$.
%
Unfortunately, such an action is not linear and hence not easy to
handle in polynomial system solving.
%
How to take advantage of such kind of symmetry in PDP is still an open
research problem.

