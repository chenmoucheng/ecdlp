%
% Introduction
%

\section{Introduction}
%
In recent years, elliptic curve cryptography is gaining momentum in
deployment, as it can achieve the same level of security as RSA using
much shorter keys and ciphertexts.
%
The security of elliptic curve cryptography is closely related to the
computational complexity of the elliptic curve discrete logarithm
problem (ECDLP).
%
Let $p$ be a prime number and $E$, a nonsingular elliptic curve over
\F{p^n}, the finite field of $p^n$ elements.
%
That is, $E$ is a plane algebraic curve defined by the equation
$y^2=x^3+ax+b$ for $a,b\in\F{p^n}$ such that
$\Delta=-16(4a^3+27b^2)\neq 0$.
%
Along with a point $\mathcal O$ at infinity, the set of rational
points $E(\F{p^n})$ forms an abelian group with $\mathcal O$ as the
identity.
%
Given $P\in E(\F{p^n})$ and $Q$ in the subgroup generated by $P$,
ECDLP is the problem of finding an integer $\alpha$ such that
$Q=\alpha P$.

Today, the best practical attacks against ECDLP are exponential-time,
generic discrete logarithm algorithms such as Pollard's rho
method~\cite{1978-pollard-kangaroo}.
%
However, recently there is a line of research on index calculus for
ECDLP started by Semaev, Gaudry, and
Diem~\cite{DBLP:journals/iacr/Semaev04,DBLP:journals/jsc/Gaudry09,DBLP:journals/moc/Diem11}.
%
Under certain heuristic assumptions, such algorithms could lead to
subexponential attacks to ECDLP in some
cases~\cite{DBLP:conf/eurocrypt/FaugerePPR12,DBLP:journals/iacr/PetitQ12,DBLP:conf/iwsec/HuangPST13}.
%
The interested reader is referred to a survey paper by Galbraith and
Gaudry for a more comprehensive and in-depth account of the recent
development of ECDLP algorithms along various
directions~\cite{DBLP:journals/dcc/GalbraithG16}.

In this paper, we investigate the computational complexity of ECDLP
for elliptic curves in various forms---including
Hessian~\cite{DBLP:conf/ches/Smart01},
Montgomery~\cite{1987-montgomery}, (twisted)
Edwards~\cite{DBLP:journals/iacr/BernsteinL07,DBLP:journals/iacr/BernsteinBJLP08},
and Weierstrass using index calculus.
%
Recently, elliptic curves of various forms such as
Curve25519~\cite{DBLP:conf/pkc/Bernstein06} have been drawing a lot of
attention in deployment, partly because some of them allow for fast
implementation and security against timing-based side channel attacks.
%
Furthermore, we can construct these curves not only over prime fields
(such as the field of $2^{255} - 19$ elements as used in Curve25519)
but also extension fields.
%
In this paper, we will focus on curves over optimal extension fields
(OEFs)~\cite{DBLP:conf/crypto/BaileyP98}.
%
An OEF is an extension field from a prime field \F p with $p$ close to
$2^8, 2^{16}, 2^{32}, 2^{64}$, etc.
%
Such primes fit nicely into the processor words of 8, 16, 32, or
64-bit microprocessors and hence are particularly suitable for
software implementation, allowing for efficient utilization of fast
integer arithmetics on modern
microprocessors~\cite{DBLP:conf/crypto/BaileyP98}.
%
As we will see, our experimental results show quite significant
difference in the computational complexity of ECDLP for elliptic
curves in various forms over OEFs.

The rest of this paper is organized as follows.
%
In Section~\ref{sec:previous-work}, we will review the relevant
literature, giving an high-level overview of attacking ECDLP using
index calculus and the state of the art in this research direction.
%
In Section~\ref{sec:montgomery-hessian}, we will present how we can
attack ECDLP using index calculus for elliptic curves in Montgomery
and Hessian forms.
%
In Section~\ref{sec:experiment}, we will experimentally compare its
computational complexity for elliptic curves in various forms.
%
Finally, we will conclude this paper by analyzing why ECDLP for
elliptic curves in certain forms may be ``easier'' to attack using
index calculus in Section~\ref{sec:analysis}.
%
