%
% Introduction
%

\section{Introduction}
%
In recent years, elliptic curve cryptography is gaining momentum in
deployment, as it can achieve the same level of security as RSA using
much shorter keys and ciphertexts.
%
The security of elliptic curve cryptography is closely related to the
complexity of solving the elliptic curve discrete logarithm problem
(ECDLP).
%
Let $p$ be a prime number, and $E$, a nonsingular elliptic curve over
\F{p^n}, the finite field of $p^n$ elements for some positive integer
$n$.
%
That is, $E$ is a plane algebraic curve defined by the equation
$y^2=x^3+ax+b$ for $a,b\in\F{p^n}$ and $\Delta=-16(4a^3+27b^2)\neq 0$.
%
Along with a point at infinity $\mathcal O$, the set of rational
points $E(\F{p^n})$ forms an abelian group with $\mathcal O$ being the
identity.
%
Given $P\in E(\F{p^n})$ and $Q\in\langle P\rangle$, ECDLP is the
problem of finding an integer $\alpha$ such that $Q=\alpha P$.

Today, the best practical attacks against ECDLP are exponential-time,
generic discrete logarithm algorithms such as Pollard's rho
method~\cite{1978-pollard-kangaroo}.
%
However, recently there is a line of research on index calculus
algorithms for ECDLP started by Semaev, Gaudry, and
Diem~\cite{DBLP:journals/iacr/Semaev04,DBLP:journals/jsc/Gaudry09,DBLP:journals/moc/Diem11}.
%
Under certain heuristic assumptions, such algorithms could lead to
subexponential attacks to ECDLP in some
cases~\cite{DBLP:conf/eurocrypt/FaugerePPR12,DBLP:journals/iacr/PetitQ12,DBLP:conf/iwsec/HuangPST13}.
%
The interested reader is referred to a survey paper by Galbraith and
Gaudry for a more comprehensive and in-depth account of the recent
development of ECDLP algorithms along various
directions~\cite{DBLP:journals/dcc/GalbraithG16}.

In this paper, we consider the complexity of solving ECDLP for
elliptic curves in various forms---including
Hessian~\cite{DBLP:conf/ches/Smart01},
Montgomery~\cite{1987-montgomery}, (twisted)
Edwards~\cite{DBLP:journals/iacr/BernsteinL07,DBLP:journals/iacr/BernsteinBJLP08},
and Weierstrass---over optimal extension fields
(OEFs)~\cite{DBLP:conf/crypto/BaileyP98} using index calculus
algorithms.
%
Recently, elliptic curves of various forms such as the (Montgomery)
Curve25519~\cite{DBLP:conf/pkc/Bernstein06} have been drawing a lot of
attention in deployment, partly because some of them allow fast
implementation that is secure against timing-based side-channel
attacks.
%
Furthermore, we can construct these curves not only over prime fields
such as the field of $2^{255} - 19$ elements used in Curve25519 but
also extension fields.
%
An OEF is an extension field from a prime field \F p with $p$ close to
$2^8, 2^{16}, 2^{32}, 2^{64}$, etc.
%
Such primes fit nicely into the processor words of 8, 16, 32, or
64-bit microprocessors and hence are particularly suitable for
software implementation, allowing for efficient utilization of fast
integer arithmetics on modern
microprocessors~\cite{DBLP:conf/crypto/BaileyP98}.
%
As we will see, our experimental results show quite significant
difference in the computational complexity of solving ECDLP for
elliptic curves in various forms over OEFs.

The rest of this paper is organized as follows.
%
In Section~\ref{sec:index-calculus-ecdlp}, we will give an high-level
overview of the index calculus algorithm for attacking ECDLP.
%
In Section~\ref{sec:montgomery-symmetry} and \ref{sec:hessian}, we
will describe curves of various forms and how we exploit the symmetry
for speeding up index calculus on them.
