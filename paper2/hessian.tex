%
% Hessian curves
%

\section{Hessian and (twisted) Edwards curves}

\subsection{Hessian curves}
\label{sec:hessian}

%
A Hessian curve $H_d$ over \F{p^n} for $p^n=2\bmod 3$ is defined by
the equation \begin{equation}
  x^3+y^3+1=3dxy \label{eq:hessian-curve} \end{equation} for
$d\in\F{p^n}$ such that $27d^3\neq 1$~\cite{DBLP:conf/ches/Smart01}.
%
For $P=(x,y)\in H_d$, $-P=(y,x)$.
%
Furthermore, the addition and doubling formulae for
$(x_3,y_3)=(x_1,y_1)+(x_2,y_2)$ are given as follows.
%
\begin{itemize}
\item When $(x_1,y_1)\neq(x_2,y_2)$:
  \begin{align*}
    x_3 & = \frac{y_1^2x_2 - y_2^2x_1}{x_2y_2 - x_1y_1} \\
    y_3 & = \frac{x_1^2y_2 - x_2^2y_1}{x_2y_2 - x_1y_1}
  \end{align*}
\item When $(x_1,y_1)=(x_2,y_2)$:
  \begin{align*}
    x_3 & = \frac{y_1(1 - x_1^3)}{x_1^3 - y_1^3} \\
    y_3 & = \frac{x_1(y_1^3 - 1)}{x_1^3 - y_1^3}
  \end{align*}
\end{itemize}


% \subsection{Summation polynomials for Hessian curves}

Following a similar approach outlined by Galbraith and
Gebregiyorgis~\cite{DBLP:conf/indocrypt/GalbraithG14}, we can
construct summation polynomials for Hessian curves.
%
First, we introduce a new variable $T=X+Y$, which is invariant under
negation of a point.
%
The 2nd summation polynomial for Hessian curves is simply
$f_{H,2} = T_1 - T_2$.
%
Now let
\[ Z=\left\{\begin{aligned}
      (x_1,y_1,t_1,&x_2,y_2,t_2,x_3,y_3,t_3)\in\F{p^n}^9:(x_i,y_i)\in H_d(\F{p^n}),i=1,2,3; \\
      & (x_1,y_1)+(x_2,y_2)+(x_3,y_3)=\mathcal O; x_i+y_i=t_i,i=1,2,3
    \end{aligned} \right\}. \]
%
Clearly, $Z$ is in the variety of the ideal
$I\subset\F{p^n}[X_1,Y_1,T_1,X_2,Y_2,T_2,X_3,Y_3,T_3]$ generated by
\[ \left\{\begin{aligned}
      & X_i^3 + Y_i^3 + 1 - 3dX_iY_i,i=1,2,3; \\
      &  (X_3 - X_1)(Y_2 - Y_1) - (X_2 - X_1)(Y_3 - Y_1); \\
      & X_i + Y_i - T_i,i=1,2,3
    \end{aligned}\right\}. \]
%
Again we compute the elimination ideal $I\cap\F{p^n}[T_1,T_2,T_3]$ and
obtain a principal ideal generated by some polynomial.
%
After removing the degenerate factors, we can obtain for Hessian curve
the 3rd summation polynomial:
\begin{align*}
  f_{H,3}(T_1,T_2,T_3) = & T_1^2T_2^2T_3 + dT_1^2T_2^2 + T_1^2T_2T_3^2
                           + dT_1^2T_2T_3 + dT_1^2T_3^2 - T_1^2 + \\
                         & T_1T_2^2T_3^2 + dT_1T_2^2T_3 + dT_1T_2T_3^2
                           + 3d^2T_1T_2T_3 + 2T_1T_2 + 2T_1T_3 + 2dT_1
                           + \\
                         & dT_2^2T_3^2 - T_2^2 + 2T_2T_3 + 2dT_2 - T_3^2 + 2dT_3 + 3d^2,
\end{align*}
%
as well as the subsequent summation polynomials via taking resultants:
\[ f_{H,m}(T_1,\ldots,T_m) =
  \res_T\left(f_{H,m-k}(T_1,\ldots,T_{m-k-1},T),f_{H,k+2}(T_{m-k},\ldots,T_m,T)\right). \]

A Hessian curve can contain an affine 2-torsion point $T_2$.
%
Since $T_2+T_2=2T_2=\mathcal O$, it follows that $-T_2=T_2$.
%
If we write $T_2=(x,y)$, then we can see that $x=y$ in order for
$-T_2=T_2$, as $-T_2=(y,x)$.
%
Substituting $x=y$ into Equation~(\ref{eq:hessian-curve}), we get an
equation $2x^3-3dx^2+1=0$.
%
Therefore, a Hessian curve $H_d(\F{p^n})$ has a 2-torsion point
$(\zeta,\zeta)$ if the polynomial $2X^3 - 3dX^2 + 1$ has a root
$\zeta$ in $\F{p^n}$.
%
In this case, the addition of this 2-torsion point to a point $(x,y)$
would give a point $(x',y')$ where
\[ \left\{\begin{aligned}
x' = & \frac{\zeta y^2 - \zeta^2x}{\zeta^2 - xy}, \\
y' = & \frac{\zeta x^2 - \zeta^2y}{\zeta^2 - xy}.
\end{aligned}\right. \]
%
Obviously, the factor base is not invariant under the action of this
2-torsion point in general.
%
Therefore, it is not clearly how to exploit such symmetry for Hessian
curves.

