
\section{Analysis, discussion, and concluding remarks}
\label{sec:analysis}

\subsection{What makes PDP on (twisted) Edwards curves easier to
  solve?}

As we have seen in Section~\ref{sec:experiment-result}, PDP on
(twisted) Edwards curves seems easier to solve than on other curves.
%
The explanation offered by Faug\`ere, Gaudry, Hout, and Renault is
``due to the smaller degree appearing in the computation of Gr\"obner
basis of $\mathscr S_{D_n}$ in comparison with the Weierstrass case,''
cf.~Section~4.1.1 of their
paper~\cite{DBLP:journals/joc/FaugereGHR14}.
%
Unfortunately, this \emph{cannot} explain the difference between
(twisted) Edwards and Montgomery curves, as the highest degrees
appearing in the computation of Gr\"obner bases are \emph{the same}
for these two curves.
%
Therefore, there must be other reasons.
%
Table~\ref{tb:terms} shows the sparsity of the polynomials before and
after Weil restriction for each of the four curves under investigation
in the case of $m=2,3,4$.
%
As there are $n$ polynomials after Weil restriction, we only show the
average numbers here.
%
We also include the maximum number of terms for a polynomial of degree
$2^{(m+1)-2}$ in $m$ variables.
%

\begin{table}[!h]
\centering
\caption{number of terms (experimental/theoretical) in polynomial systems before and after Weil decent}
\label{tb:terms}
\begin{tabular}{llllllll}
\hline
\multicolumn{1}{|l|}{\multirow{2}{*}{$m$}} & \multicolumn{1}{l|}{\multirow{2}{*}{Curve}} & \multicolumn{3}{l|}{before Weil decent}                                                    & \multicolumn{3}{l|}{after Weil decent}                                                           \\ \cline{3-8} 
\multicolumn{1}{|l|}{}                     & \multicolumn{1}{l|}{}                       & \multicolumn{1}{l|}{total}   & \multicolumn{1}{l|}{odd}     & \multicolumn{1}{l|}{even}    & \multicolumn{1}{l|}{total}     & \multicolumn{1}{l|}{odd}       & \multicolumn{1}{l|}{even}      \\ \hline
\multicolumn{1}{|l|}{\multirow{4}{*}{2}}   & \multicolumn{1}{l|}{Hessian}                & \multicolumn{1}{l|}{6/6}     & \multicolumn{1}{l|}{2/2}     & \multicolumn{1}{l|}{4/4}     & \multicolumn{1}{l|}{5.2/6}     & \multicolumn{1}{l|}{2/2}       & \multicolumn{1}{l|}{3.2/4}     \\ \cline{2-8} 
\multicolumn{1}{|l|}{}                     & \multicolumn{1}{l|}{Weierstrass}            & \multicolumn{1}{l|}{6/6}     & \multicolumn{1}{l|}{2/2}     & \multicolumn{1}{l|}{4/4}     & \multicolumn{1}{l|}{5.2/6}     & \multicolumn{1}{l|}{2/2}       & \multicolumn{1}{l|}{3.2/4}     \\ \cline{2-8} 
\multicolumn{1}{|l|}{}                     & \multicolumn{1}{l|}{Montgomery}             & \multicolumn{1}{l|}{6/6}     & \multicolumn{1}{l|}{2/2}     & \multicolumn{1}{l|}{4/4}     & \multicolumn{1}{l|}{5.2/6}     & \multicolumn{1}{l|}{2/2}       & \multicolumn{1}{l|}{3.2/4}     \\ \cline{2-8} 
\multicolumn{1}{|l|}{}                     & \multicolumn{1}{l|}{tEdwards}               & \multicolumn{1}{l|}{4/6}     & \multicolumn{1}{l|}{0/2}     & \multicolumn{1}{l|}{4/4}     & \multicolumn{1}{l|}{3.2/6}     & \multicolumn{1}{l|}{0/2}       & \multicolumn{1}{l|}{3.2/4}     \\ \hline \vspace{-3mm}
                                           &                                             &                              &                              &                              &                                &                                &                                \\ \hline
\multicolumn{1}{|l|}{\multirow{4}{*}{3}}   & \multicolumn{1}{l|}{Hessian}                & \multicolumn{1}{l|}{35/35}   & \multicolumn{1}{l|}{16/16}   & \multicolumn{1}{l|}{19/19}   & \multicolumn{1}{l|}{34.2/35}   & \multicolumn{1}{l|}{16/16}     & \multicolumn{1}{l|}{18.2/19}   \\ \cline{2-8} 
\multicolumn{1}{|l|}{}                     & \multicolumn{1}{l|}{Weierstrass}            & \multicolumn{1}{l|}{35/35}   & \multicolumn{1}{l|}{16/16}   & \multicolumn{1}{l|}{19/19}   & \multicolumn{1}{l|}{34/35}     & \multicolumn{1}{l|}{16/16}     & \multicolumn{1}{l|}{18/19}     \\ \cline{2-8} 
\multicolumn{1}{|l|}{}                     & \multicolumn{1}{l|}{Montgomery}             & \multicolumn{1}{l|}{35/35}   & \multicolumn{1}{l|}{16/16}   & \multicolumn{1}{l|}{19/19}   & \multicolumn{1}{l|}{33.4/35}   & \multicolumn{1}{l|}{16/16}     & \multicolumn{1}{l|}{17.4/19}   \\ \cline{2-8} 
\multicolumn{1}{|l|}{}                     & \multicolumn{1}{l|}{tEdwards}               & \multicolumn{1}{l|}{25/35}   & \multicolumn{1}{l|}{6/16}    & \multicolumn{1}{l|}{19/19}   & \multicolumn{1}{l|}{23.4/35}   & \multicolumn{1}{l|}{6/16}      & \multicolumn{1}{l|}{17.4/19}   \\ \hline  \vspace{-3mm}
                                           &                                             &                              &                              &                              &                                &                                &                                \\ \hline
\multicolumn{1}{|l|}{\multirow{4}{*}{4}}   & \multicolumn{1}{l|}{Hessian}                & \multicolumn{1}{l|}{495/495} & \multicolumn{1}{l|}{240/240} & \multicolumn{1}{l|}{255/255} & \multicolumn{1}{l|}{493.2/495} & \multicolumn{1}{l|}{239.4/240} & \multicolumn{1}{l|}{253.8/255} \\ \cline{2-8} 
\multicolumn{1}{|l|}{}                     & \multicolumn{1}{l|}{Weierstrass}            & \multicolumn{1}{l|}{495/495} & \multicolumn{1}{l|}{240/240} & \multicolumn{1}{l|}{255/255} & \multicolumn{1}{l|}{492/495}   & \multicolumn{1}{l|}{238.4/240} & \multicolumn{1}{l|}{253.6/255} \\ \cline{2-8} 
\multicolumn{1}{|l|}{}                     & \multicolumn{1}{l|}{Montgomery}             & \multicolumn{1}{l|}{495/495} & \multicolumn{1}{l|}{240/240} & \multicolumn{1}{l|}{255/255} & \multicolumn{1}{l|}{492.2/495} & \multicolumn{1}{l|}{239.2/240} & \multicolumn{1}{l|}{253/255}   \\ \cline{2-8} 
\multicolumn{1}{|l|}{}                     & \multicolumn{1}{l|}{tEdwards}               & \multicolumn{1}{l|}{255/495} & \multicolumn{1}{l|}{0/240}   & \multicolumn{1}{l|}{255/255} & \multicolumn{1}{l|}{253/495}   & \multicolumn{1}{l|}{0/240}     & \multicolumn{1}{l|}{253/255}   \\ \hline
\end{tabular}
\end{table}


% \begin{table}[!h]
% \centering
% \caption{\#terms in polynomial systems before/after Weil decent}
% \label{tb:terms}
% \begin{tabular}{llllllll}
% \hline
% \multicolumn{1}{|l|}{m}                  & \multicolumn{1}{l|}{Curve}       & \multicolumn{1}{l|}{\begin{tabular}[c]{@{}l@{}}before\\ Weil decent\end{tabular}} & \multicolumn{1}{l|}{odd}     & \multicolumn{1}{l|}{even}    & \multicolumn{1}{l|}{\begin{tabular}[c]{@{}l@{}}after \\ Weil decent\end{tabular}} & \multicolumn{1}{l|}{odd}           & \multicolumn{1}{l|}{even}            \\ \hline
% \multicolumn{1}{|l|}{\multirow{4}{*}{2}} & \multicolumn{1}{l|}{Hessian}     & \multicolumn{1}{l|}{6/6}                                                          & \multicolumn{1}{l|}{2/2}     & \multicolumn{1}{l|}{4/4}     & \multicolumn{1}{l|}{58/65}                                                        & \multicolumn{1}{l|}{10/10}         & \multicolumn{1}{l|}{48/55}           \\ \cline{2-8} 
% \multicolumn{1}{|l|}{}                   & \multicolumn{1}{l|}{Weierstrass} & \multicolumn{1}{l|}{6/6}                                                          & \multicolumn{1}{l|}{2/2}     & \multicolumn{1}{l|}{4/4}     & \multicolumn{1}{l|}{58/65}                                                        & \multicolumn{1}{l|}{10/10}         & \multicolumn{1}{l|}{48/55}           \\ \cline{2-8} 
% \multicolumn{1}{|l|}{}                   & \multicolumn{1}{l|}{Montgomery}  & \multicolumn{1}{l|}{6/6}                                                          & \multicolumn{1}{l|}{2/2}     & \multicolumn{1}{l|}{4/4}     & \multicolumn{1}{l|}{58/65}                                                        & \multicolumn{1}{l|}{10/10}         & \multicolumn{1}{l|}{48/55}           \\ \cline{2-8} 
% \multicolumn{1}{|l|}{}                   & \multicolumn{1}{l|}{tEdwards}    & \multicolumn{1}{l|}{4/6}                                                          & \multicolumn{1}{l|}{0/2}     & \multicolumn{1}{l|}{4/4}     & \multicolumn{1}{l|}{28/65}                                                        & \multicolumn{1}{l|}{5/10}          & \multicolumn{1}{l|}{23/55}           \\ \hline \vspace{-5mm}
%                                          &                                  &                                                                                   &                              &                              &                                                                                   &                                    &                                      \\ \hline
% \multicolumn{1}{|l|}{\multirow{4}{*}{3}} & \multicolumn{1}{l|}{Hessian}     & \multicolumn{1}{l|}{35/35}                                                        & \multicolumn{1}{l|}{16/16}   & \multicolumn{1}{l|}{19/19}   & \multicolumn{1}{l|}{3856/3875}                                                    & \multicolumn{1}{l|}{695/695}       & \multicolumn{1}{l|}{3161/3180}       \\ \cline{2-8} 
% \multicolumn{1}{|l|}{}                   & \multicolumn{1}{l|}{Weierstrass} & \multicolumn{1}{l|}{35/35}                                                        & \multicolumn{1}{l|}{16/16}   & \multicolumn{1}{l|}{19/19}   & \multicolumn{1}{l|}{3854/3875}                                                    & \multicolumn{1}{l|}{694/695}       & \multicolumn{1}{l|}{3160/3180}       \\ \cline{2-8} 
% \multicolumn{1}{|l|}{}                   & \multicolumn{1}{l|}{Montgomery}  & \multicolumn{1}{l|}{35/35}                                                        & \multicolumn{1}{l|}{16/16}   & \multicolumn{1}{l|}{19/19}   & \multicolumn{1}{l|}{3838/3875}                                                    & \multicolumn{1}{l|}{695/695}       & \multicolumn{1}{l|}{3143/3180}       \\ \cline{2-8} 
% \multicolumn{1}{|l|}{}                   & \multicolumn{1}{l|}{tEdwards}    & \multicolumn{1}{l|}{25/35}                                                        & \multicolumn{1}{l|}{6/16}    & \multicolumn{1}{l|}{19/19}   & \multicolumn{1}{l|}{2235/3875}                                                    & \multicolumn{1}{l|}{550/695}       & \multicolumn{1}{l|}{1685/3180}       \\ \hline \vspace{-5mm}
%                                          &                                  &                                                                                   &                              &                              &                                                                                   &                                    &                                      \\ \hline
% \multicolumn{1}{|l|}{\multirow{4}{*}{4}} & \multicolumn{1}{l|}{Hessian}     & \multicolumn{1}{l|}{495/495}                                                      & \multicolumn{1}{l|}{240/240} & \multicolumn{1}{l|}{255/255} & \multicolumn{1}{l|}{3095460/3108104}                                              & \multicolumn{1}{l|}{700485/701864} & \multicolumn{1}{l|}{2394975/2406240} \\ \cline{2-8} 
% \multicolumn{1}{|l|}{}                   & \multicolumn{1}{l|}{Weierstrass} & \multicolumn{1}{l|}{495/495}                                                      & \multicolumn{1}{l|}{240/240} & \multicolumn{1}{l|}{255/255} & \multicolumn{1}{l|}{3090557/3108104}                                              & \multicolumn{1}{l|}{697526/701864} & \multicolumn{1}{l|}{2393031/2406240} \\ \cline{2-8} 
% \multicolumn{1}{|l|}{}                   & \multicolumn{1}{l|}{Montgomery}  & \multicolumn{1}{l|}{495/495}                                                      & \multicolumn{1}{l|}{240/240} & \multicolumn{1}{l|}{255/255} & \multicolumn{1}{l|}{3101711/3108104}                                              & \multicolumn{1}{l|}{699180/701864} & \multicolumn{1}{l|}{2402531/2406240} \\ \cline{2-8} 
% \multicolumn{1}{|l|}{}                   & \multicolumn{1}{l|}{tEdwards}    & \multicolumn{1}{l|}{255/495}                                                      & \multicolumn{1}{l|}{0/240}   & \multicolumn{1}{l|}{255/255} & \multicolumn{1}{l|}{1548859/3108104}                                              & \multicolumn{1}{l|}{350847/701864} & \multicolumn{1}{l|}{1198012/2406240} \\ \hline
% \end{tabular}
% \end{table}

We can see that total number of terms for twisted Edwards curves is
significantly fewer than that for the other curves in all cases.
%
Naturally, this could lead to faster solving time with the F4
algorithm.
%
We also note that, except for twisted Edwards curves, the summation
polynomials before Weil restriction for the other curves are all 100\%
dense without any missing terms.

% We further analyze which terms are missing from the summation
% polynomials for curves in different forms.
% %
% We classify terms into odd vs.~even degrees.
% %
% From Table~\ref{tb:terms}, missing of only odd terms on twisted Edwards
% curve reduce the number of terms.

%
% (twisted) Edwards curve
%

%------------------------------
\subsection{(Twisted) Edwards curves}
%------------------------------------
\label{sec:twisted-edwards}


Faug\`ere, Gaudry, Hout, and Renault studied PDP on twisted Edwards,
twisted Jacobi intersections, and Weierstrass
curves~\cite{DBLP:journals/joc/FaugereGHR14}.
%
For the sake of completeness, we include some basic facts about
(twisted) Edwards curves here.
%
%
An Edwards curve over \F{p^n} for $p\neq 2$ is defined by the
equation \begin{equation*}
  x^2+y^2=1+dx^2y^2 \label{eq:edwards-curve} \end{equation*} for
$d\in\F{p^n}$~\cite{DBLP:journals/iacr/BernsteinL07}.
%
A twisted Edwards curve $tE_{a',d'}$ over \F{p^n} for $p\neq 2$ is
defined by the equation \begin{equation}
  a'x^2+y^2=1+d'x^2y^2 \label{eq:twisted-edwards-curve} \end{equation}
for $a',d'\in\F{p^n}$~\cite{DBLP:journals/iacr/BernsteinBJLP08}.
%
A twisted Edwards curve is a quadratic twist of an Edwards curve by
$a_0=1/(a'-d')$.
%
For $P=(x,y)\in tE_{a',d'}$, $-P=(-x,y)$.
%
Furthermore, the addition and doubling formulae for
$(x_3,y_3)=(x_1,y_1)+(x_2,y_2)$ are given as follows.
%
\begin{itemize}
\item When $(x_1,y_1)\neq(x_2,y_2)$:
  \begin{align*}
    x_3 & = \frac{x_1y_2 + y_1x_2}{1 + d'x_1x_2y_1y_2} \\
    y_3 & = \frac{y_1y_2 - a'x_1x_2}{1 - d'x_1x_2y_1y_2}
  \end{align*}
\item When $(x_1,y_1)=(x_2,y_2)$:
  \begin{align*}
    x_3 & = \frac{2x_1y_1}{1 + d'x_1^2y_1^2} \\
    y_3 & = \frac{y_1^2 - a'x_1^2}{1 - d'x_1^2y_1^2}
  \end{align*}
\end{itemize}
%
As given by Faug\`ere, Gaudry, Hout, and
Renault~\cite{DBLP:journals/joc/FaugereGHR14}, the 3rd summation
polynomial for twisted Edwards curve is
%
\begin{align*}
  f_{tE, 3}(Y_1,Y_2,Y_3) = & \left(Y_1^2Y_2^2 - Y_1^2 - Y_2^2 + \frac{a}{d}\right)Y_3^2  + 2\frac{d-a}{d}Y_1Y_2Y_3 +
                             \frac{a}{d}\left(Y_1^2 + Y_2^2 - 1\right)
                             - Y_1^2Y_2^2.
\end{align*}
%
As usual, subsequent summation polynomials can be obtained by using
resultants.



\section{Observations of summation polynomial on twisted Edwards
  curve}
\label{sec:twisted-edwards-summation-polynomial}

We classify variables and monomials in from the viewpoint of parity. 
%
We refer to variable whose exponent is even number and odd number 
as respectively \emph{even variable} and \emph{odd variable}
%
Similarly, we refer to monomial whose degree is even number and odd number 
as respectively \emph{even monomial} and \emph{odd monoimal}


\begin{description}
  \item [Lemma 1]~\\
  %
  Considering resultant of two multivariate polynomial $a_1$ and $a_2$,
  you will consider the polynomials as univariate polynomial of the variable
  to eliminate $T$, such as $f_1 = a_mT^m + \dots +a_1T^1 + a_0$ 
  and $f_2 = b_nT^n + \dots +b_1T^1 + b_0$.
  %
  If degree of both of $m$ and $n$ is even, 
  then coefficients of odd monomial, $a_{2k+1}$ and $b_{2k+1}$ for integer $k$,
  is contained even number times in the term of resultant.
  %
\end{description}

\noindent
\emph{Proof.}
%
Resultant of $a_1$ and $a_2$ is 
determinant of following square matrix X of size $m+n$.
%
\begin{equation*}
X =
\begin{bmatrix}
  a_m & a_{m-1} & \ldots & & a_0 & & &  \\
          & a_m & a_{m-1} & \ldots & & a_0  & &  \\
  & & \ddots & & & & \ddots &  \\
    & & & a_m & a_{m-1} & \ldots & & a_0  \\
  b_n & b_{n-1} & \ldots & & b_0 & & &  \\
       & b_n & b_{n-1} & \ldots & & b_0  & &  \\
  & & \ddots & & & & \ddots &  \\
    & & & b_n & b_{n-1} & \ldots & & b_0
\end{bmatrix}
\end{equation*}

Let components in the $i$-th row and $j$-th column in the matrix $X$ 
be represented to $x_{ij}$. 
%
From assumption, $a_m$ and $b_n$ is the coefficients of even monomial
and the number of row which include $a_i$ and $b_i$ is both even.
%
Consequently, coefficients of odd monomials, $a_{2k+1}$ and $b_{2k+1}$ 
for integer $k$, correspond components whose sum of index (i+j) is odd.
%
Also, coefficients of even monomials, $a_{2k}$ and $b_{2k}$ for 
integer $k$, correspond components whose sum of index is even.

Here, determinant of $X$ is defined as
%
\begin{align*}
  %
  \mbox{det}(X) = \sum_{\sigma \in S_{n+m}} \mbox{sgn}(\sigma)x_{1\sigma(1)}
                   x_{2\sigma(2)}\cdots x_{m+n\sigma(m+n)}
\end{align*}
%
Under any permutation, total summation of sum of index remain even number,
that is $2*(1+ \cdots +n)$.
%
Hence, components whose sum of index is odd is contained even number times.
%
Therefore, coefficients of odd monoials is contained even number times
in the term of resultant.
\qed \\

We focus on each variable which is consisted term
is whether even or odd variable.
%
We reffer to the term which consist of only the same parity of variables
as \emph{well-regulated term}.


\begin{description}
  %
  \item [Lemma 2]~\\
  %
  If all terms in 3rd summation polynomial are well-regulated terms,
  all terms in summation polynomial with any variables are also
  well-regulated.
  %
\end{description}

\noindent
\emph{Proof.}
%
We prove that summation polynomial $f_{n}$ for $n\geq3$ satisfy the
proposion by mathematical inducution.

Let $n=3$, from assumption, summation polynomial $f_3(Y_1, Y_2, Y_3)$
satisfy proposion.
%
\begin{align*}
  %
  f_{3}(Y_1, Y_2, Y_3)=a_2Y_3^2 + a_1Y_3 + a_0
  %
\end{align*}
%

Let $n=k$, we assume that $f_{k}(Y_1, \dots , Y_{k})$ satisfy the proposion.
% 
we can express the polynomial as univariate polynomial of $Y_{k}$ 
with $2^{k-2}$ degree. 
%
\begin{align*}
  %
  f_{k}(Y_1, \dots , Y_{k}) & = 
  b_{2^{k-2}}Y_{k}^{2^{k-2}} + b_{2^{k-2}-1}Y_{k}^{2^{k-2}-1} + 
   \cdots + b_2Y_{k}^2 + b_1Y_{k} + b_0
  %
\end{align*}
%
In other words, we assume polynomial $b_{2k}$ consist of only even variables 
and $b_{2k+1}$ consist of only odd variables for integer $k$.

Let $n=k+1$, we will consider about $f_{k+1}(Y_1, \dots , Y_{k}, Y_{k+1})$.
%
From the definition of summation polynomial, we can express the polynomial
by using above coefficient $a_i$ and $b_i$.
%
\begin{align*}
  %
  f_{k+1}(Y_1, \dots , Y_{k}, Y_{k+1}) &= 
   Res\left(f_{k}(Y_1, \dots , Y_{k-1}, T), f_3(Y_{k}, Y_{k+1}, T)\right)\\
  %
   & = Res(b_{2^{k-2}}T^{2^{k-2}} + \cdots + b_1T + b_0,~ a_2T^2 + a_1T + a_0)
  %
\end{align*}
%
Similarly, we can express as univariate polynoimal of $Y_{k+1}$ 
with $2^{k-1}$ degree.
%
\begin{align*}
  f_{k+1}(Y_1, \dots , Y_{k}, Y_{k+1}) =
  c_{2^{k-1}}Y_{k+1}^{2^{k-1}} + c_{2^{k-1}-1}Y_{k+1}^{2^{k-1}-1} + 
   \cdots + c_2Y_{k+1}^2 + c_1Y_{k+1} + c_0
\end{align*}
%
All odd terms include multiplication of coefficient $a_1$ odd times.
%
Because degree of all summation polynomial are even, 
we can consider Lemma 1 in this case.
%
Hence, coefficients of all odd terms $c_{2k+1}$ include 
multiplication of coefficient $b_{2k+1}$ odd times.
%
From the assumption, $b_{2k+1}$ consist of only odd variables.
%
Therefore, all variables in all odd terms are odd variables.
%
Similarly, we can prove proposion in the case of even terms.
%
\qed \\


\begin{description}
  \item [theorem 1]~\\
  %
  Let us think about point decomposition of $m$ point 
  by using summation polynomial with $m+1$ variables, 
  $f_{m+1}(Y_0, \dots, Y_m, Y)$.
  %
  $Y$ is associated with point to decompose 
  and will be substituted by some constant.

  Suppose 3rd summation polynomial consists of only well-regulated terms.
  %
  When $m$ is even number,
  polynomial after substituting consist of only even monomials.

  When $m$ is odd number,
  polynomial after substituting includes some odd monomials.
  %
\end{description}

\noindent
\emph{Proof.}
%
Because max degree of each variables of 
$f_{m+1}(Y_0, \dots, Y_m, Y)$ are $2^{m-1}$,
it can be written as univariate polynoimal of $Y$.
%
\begin{align*}
  a_{2^{m-1}}Y^{2^{m-1}} + a_{2^{m-1}-1}Y^{2^{m-1}-1} + 
  \cdots + a_2Y^2 + a_1Y + a_0
\end{align*}
%
The coefficients $c_i$ are a polynomial 
with $m$ variables ($Y_0, \dots, Y_m$).
%
After substituting of $Y$, summation polynomial is composed of $c_i$.
%
Let $k$ be integers. 
From Lemma 2, all monomials in $a_{2k+1}$ consist of odd degree variables.
%
If $m$ is even number, there are all even monomials in $a_{2k+1}$
because even times summation of odd numbers are even.
%
Similarly, if $m$ is odd number, all monomials in $a_{2k+1}$ are odd monomials.
\qed \\

\begin{description}
  \item [Corollary 1]~\\
    As you see above of this section, 
    in twisted Edwards curve, 3rd summation polynomial consists of
    only well-regulated terms.
    %
    Therefore, twisted Edwards curve is the case of Theorem 1.
\end{description}






\subsection{What price for a highly symmetric factor base?}
\label{sec:price}

Last but not least, we discuss the price needed to pay to have a
highly symmetric factor base $\mathcal F$ that is invariant under more
group actions in addition to that of the symmetric group $S_m$.
%
As previewed in Section~\ref{sec:symmetry-decomposition-probability},
we would expect that the effect of the decrease in decomposition
probability due to additional symmetry in $\mathcal F$ could be offset
by that of the increase in number of solutions.
%
For example, let us reconsider the group action of addition of $T_2$
in Section~\ref{sec:exploit-symmetry}.
%
If we could get $2^{m-1}$ solutions, then the loss of the factor of
$2^{m-1}$ in decomposition probability would be compensated.
%
This way everything would be the same as if there were no such
symmetry, and we could exploit the additional symmetry at no cost.

Unfortunately, this proposition is \emph{false} in general.
%
Consider an example of $m=4$.
%
Let $Q_i=P_i+ T_2$ for $i=1,2,3,4$.
%
We can write down all $2^{m-1}=8$ possible ways of a point
decomposition under this group action:
%
\[ \begin{aligned}
P_1 + P_2 + P_3 + P_4 = & Q_1 + Q_2 + P_3 + P_4 \\
= Q_1 + P_2 + Q_3 + P_4 = & Q_1 + P_2 + P_3 + Q_4 \\
= P_1 + Q_2 + Q_3 + P_4 = & P_1 + Q_2 + P_3 + Q_4 \\
= P_1 + P_2 + Q_3 + Q_4 = & Q_1 + Q_2 + Q_3 + Q_4.
\end{aligned} \]
%
It is easy to find that we have only 5 linearly independent relations
from these 8 relations, as there are nontrivial linear combinations
summing to zero, e.g.:
\[ (P_1 + P_2 + P_3 + P_4) - (Q_1 + Q_2 + P_3 + P_4) - (P_1 + P_2 +
  Q_3 + Q_4) + (Q_1 + Q_2 + Q_3 + Q_4) = \mathcal O.\]


As explained in Section~\ref{subsec:conditions}, the factor bases for
Montgomery and twisted Edwards curves are invariant under addition of
2-torsion points.
%
For $m=3$, we achieve maximum rank of $2^{m-1}=4$.
%
For $m=4$, as we have explained above, we can only have rank $5$, 
which is strictly less than the maximum possible rank $2^{m-1}=8$.

Finally, we note that we have not exploited any symmetry for Hessian
curves in our experiments.
%
However, the rank for Hessian curves is always 1 in all our
experiments.
%
This shows that the factor base we have chosen for Hessian curves is
\emph{not} invariant under addition of small torsion points, as the
rank would be $>1$ otherwise.





