%
\section{Experimental results on OEF security}
%
\label{sec:experiments-OEF}
%
\subsection{Security consideration of ECDLP over OEF}
%
An OEF $\Fqn$ is typically used to achieve highly efficient software
implementation.
%
From the security point of view, $n$ is recommended to be a prime
since $E(\Fqm) \subset E(\Fqn)$ yields $\#E(\Fqm) \mid \# E(\Fqn)$ for
$m \mid n$.
%
However, we have not seen relation between parameters $(p, n)$ of OEF,
and the security of ECDLP over an OEF $\Fqn$.
%
For example, suppose that we use OEFs of $\F{{p_{32}}^5}$,
$\F{{p_{16}}^{11}}$, and $\F{{p_{8}}^{19}}$, where $p_{32}, p_{16}$,
and $p_{8}$ are primes close to $2^8, 2^{16}$, and $2^{32}$,
respectively.
%
These OEFs satisfy a condition of prime extension degree of 5, 11, and
19, respectively, and have almost the same number of rational points,
$\#E(\F{{p_{32}}^5}) \sim \#E(\F{{p_{16}}^{11}}) \sim
\#E(\F{{p_{8}}^{19}})$.
%
Now we pose a question: Using index calculus algorithm to Montgomery 
curves described in Section~\ref{subsec:TSPL}, is there any significant 
difference in the
computational complexity of solving ECDLP for $\F{{p_{32}}^5}$,
$\F{{p_{16}}^{11}}$, and $\F{{p_{8}}^{19}}$?
%
Such a difference occurs in the case of Weierstrass curves?
We will provide some empirical evidence showing an affirmative answer
in the next section.


\subsection{Experiment 2: Security evaluation for various OEFs}
%
As we discuss in the previous section, we examine how OEF influences the
security of solving ECDLP.
We choose several OEFs suggested by Bailey and
Paar~\cite{DBLP:conf/crypto/BaileyP98}.
In this experiment, we would like to evaluate the security of
Montgomery curves over various OEFs.  By using the same OEFs,  
the security of Weierstrass curves is also examined.
To set the security level of both curves over the same OEF equal to 
each other, the following elliptic curves are used for experiments:
1. Montomery curves $E_M$ with cofactor = 2.  \\
2. Weierstrass curves $E$ with prime order

The results are summarized in Table~\ref{tab:experimental-results}.
%
All timing results are reported in seconds unless otherwise specified.
%
%
%
We then decide a factor base $\mathcal F$ using a vector subspace $V$
of \F{p^n} over \F p, i.e., $|\mathcal F|\approx p^{\dim V}$, which,
together with the linear algebra constant, determines the linear
algebra time (``T\textsubscript L'' in
Table~\ref{tab:experimental-results}).
%
The bulk of our experimentation then involves generating random points
in $\langle P\rangle$ \emph{known} to be decomposable into a sum of
$m$ points from $\mathcal F$.
%
As noted by Galbraith and Gebregiyorgis, the time for solving such
instances tends to be larger than that for attempting to decompose
points that \emph{cannot} be decomposed
so~\cite{DBLP:conf/indocrypt/GalbraithG14}.
%
In general, the success probability of such point decomposition is
roughly $\mathcal O(p^{m\dim V - n}/m!)$ for $m\dim V<n$, which is
taken into account in estimating the relation collection time
(``T\textsubscript R'' in Table~\ref{tab:experimental-results}).
%
\begin{table}
  \begin{center}
    \begin{tabular}{llcr}
      OEF & $\ord(P)$ & $\dim V$ & T\textsubscript R$+$T\textsubscript L \\ \hline
      \F{241^{19}} & $\mathcal O(2^{147})$ & 5 & $212.4\times\mathcal O(2^{72})+\mathcal O(2^{80})$ \\
      \F{65371^{11}} & $\mathcal O(2^{173})$ & 4 & $37.6\times\mathcal O(2^{49})+\mathcal O(2^{129})$ \\
      \F{4294967291^5} & $\mathcal O(2^{154})$ & 2 & $0.04\times\mathcal O(2^{65})+\mathcal O(2^{129})$
    \end{tabular}
  \end{center}
  \caption{Security evaluation of Montgomery curves of various OEFs for $m=2$.}
  \label{tab:experimental-results}
\end{table}
%
It is interesting to note that the complexity for the relation
collection phase is quite different for different OEFs, even though we
use the same Weil descent algorithm.
%
However, as we can see from Table~\ref{tab:experimental-results}, such
index calculus algorithm cannot yet outperform the generic algorithms
such as Pollard's rho method, as the linear algebra phase takes
significantly longer time.

% \section{Concluding remarks}
% %
% \label{sec:conclusions}
%

