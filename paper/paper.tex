%
%
% This is LLNCS.DEM the demonstration file of
% the LaTeX macro package from Springer-Verlag
% for Lecture Notes in Computer Science,
% version 2.4 for LaTeX2e as of 16. April 2010
%
\documentclass{llncs}
%
\usepackage{algpseudocode,amsfonts,amsmath,amssymb}
%
\DeclareMathOperator{\ord}{ord}
\DeclareMathOperator{\res}{Res}
\newcommand{\F}[1]{\ensuremath{\mathbb F_{#1}}}
%\def\F{{\mathbb F}}
\def\Fqn{{\mathbb F}_{p^n}}
\def\Fqm{{\mathbb F}_{p^m}}
\long\def\comment#1{}
%
\begin{document}
%
\title{Security evaluation of Montgomery-form elliptic curve
  cryptography over optimal extension fields}
%
%\titlerunning{}
%
%\author{Kenta~Kodera \and Atsuko~Miyaji \and Chen-Mou~Cheng}
%
%\authorrunning{}
%
%\institute{Osaka University, Japan}

\maketitle              % typeset the title of the contribution

\begin{abstract}
%
  The security of elliptic curve cryptography is closely related to
  the complexity of solving the elliptic curve discrete logarithm
  problem (ECDLP).
%
  Today, the best practical attacks against ECDLP are
  exponential-time, generic discrete logarithm algorithms such as
  Pollard's rho method~\cite{1978-pollard-kangaroo}.
%
  However, recently there is a line of research on index calculus
  algorithms for ECDLP started by Semaev, Gaudry, and
  Diem~\cite{DBLP:journals/iacr/Semaev04,DBLP:journals/jsc/Gaudry09,DBLP:journals/moc/Diem11}.
%
  Under certain heuristic assumptions, such algorithms could lead to
  subexponential attacks to ECDLP in some
  cases~\cite{DBLP:conf/eurocrypt/FaugerePPR12,DBLP:journals/iacr/PetitQ12,DBLP:conf/iwsec/HuangPST13}.
%
  In this paper, we consider the complexity of solving ECDLP for
  elliptic curves in Montgomery form~\cite{1987-montgomery} over
  optimal extension fields (OEFs)~\cite{DBLP:conf/crypto/BaileyP98}
  using index calculus algorithms.
%
  Recently, Montgomery curves such as
  Curve25519~\cite{DBLP:conf/pkc/Bernstein06} have been drawing a lot
  of attention in deployment, partly because their scalar
  multiplication formula involves only the $x$-coordinates, allowing
  fast implementation secure against timing-based side-channel
  attacks.
%
  An OEF is an extension field from a prime field \F p with $p$ close
  to $2^8, 2^{16}, 2^{32}, 2^{64}, \ldots$
%
  Such primes fit nicely into the processor words of 8, 16, 32, or
  64-bit microprocessors and hence are particularly suitable for
  software implementation, allowing for efficient utilization of fast
  integer arithmetics on modern
  microprocessors~\cite{DBLP:conf/crypto/BaileyP98}.
%
  When it comes to security evaluation using index calculus
  algorithms, there have been several papers on, e.g., Edwards
  curves~\cite{DBLP:journals/joc/FaugereGHR14,DBLP:conf/indocrypt/GalbraithG14},
  but to the best of our knowledge, this is the first work that
  discusses the security of Montgomery curves over OEF from such a
  viewpoint.
%
  The research question we would like to answer is: Using index
  calculus algorithm, is there any significant difference in the
  computational complexity of solving ECDLP for various OEFs?
%
  We will provide some empirical evidence showing an affirmative
  answer in this paper.
%
  \keywords{ECDLP, index calculus, Montgomery curves, optimal
    extension fields, security evaluation}
\end{abstract}

\section{Introduction}
%
In recent years, elliptic curve cryptography is gaining momentum in
deployment, as it can achieve the same level of security as RSA using
much shorter keys and ciphertexts.
%
The security of elliptic curve cryptography is closely related to the
complexity of solving the elliptic curve discrete logarithm problem
(ECDLP).
%
Let $p$ be a prime number, and $E$, a nonsingular elliptic curve over
\F{p^n}, the finite field of $p^n$ elements for some positive integer
$n$.
%
That is, $E$ is a plane algebraic curve defined by the equation
$y^2=x^3+ax+b$ for $a,b\in\F{p^n}$ and $\Delta=-16(4a^3+27b^2)\neq 0$.
%
Along with a point at infinity $\mathcal O$, the set of rational
points $E(\F{p^n})$ forms an abelian group with $\mathcal O$ being the
identity.
%
Given $P\in E(\F{p^n})$ and $Q\in\langle P\rangle$, ECDLP is the
problem of finding an integer $\alpha$ such that $Q=\alpha P$.

Today, the best practical attacks against ECDLP are exponential-time,
generic discrete logarithm algorithms such as Pollard's rho
method~\cite{1978-pollard-kangaroo}.
%
However, recently there is a line of research on index calculus
algorithms for ECDLP started by Semaev, Gaudry, and
Diem~\cite{DBLP:journals/iacr/Semaev04,DBLP:journals/jsc/Gaudry09,DBLP:journals/moc/Diem11}.
%
Under certain heuristic assumptions, such algorithms could lead to
subexponential attacks to ECDLP in some
cases~\cite{DBLP:conf/eurocrypt/FaugerePPR12,DBLP:journals/iacr/PetitQ12,DBLP:conf/iwsec/HuangPST13}.
%
The interested reader is referred to a survey paper by Galbraith and
Gaudry for a more comprehensive and in-depth account of the recent
development of ECDLP algorithms along various
directions~\cite{DBLP:journals/dcc/GalbraithG16}.

In this paper, we consider the complexity of solving ECDLP for
elliptic curves in Montgomery form~\cite{1987-montgomery} over optimal
extension fields (OEFs)~\cite{DBLP:conf/crypto/BaileyP98} using index
calculus algorithms.
%
Recently, Montgomery curves such as
Curve25519~\cite{DBLP:conf/pkc/Bernstein06} have been drawing a lot of
attention in deployment, partly because their scalar multiplication
formula involves only the $x$-coordinates, allowing fast
implementation secure against timing-based side-channel attacks.
%
In general, we can construct Montgomery curves not only over prime
fields such as \F{2^{255}-19} as used in Curve25519 but also extension
fields.
%
An OEF is an extension field from a prime field \F p with $p$ close to
$2^8, 2^{16}, 2^{32}, 2^{64}, \ldots$
%
Such primes fit nicely into the processor words of 8, 16, 32, or
64-bit microprocessors and hence are particularly suitable for
software implementation, allowing for efficient utilization of fast
integer arithmetics on modern
microprocessors~\cite{DBLP:conf/crypto/BaileyP98}.
%
When it comes to security evaluation using index calculus algorithms,
there have been several papers on, e.g., Edwards
curves~\cite{DBLP:journals/joc/FaugereGHR14,DBLP:conf/indocrypt/GalbraithG14},
but to the best of our knowledge, this is the first work that
discusses the security of Montgomery curves over OEF from such a
viewpoint.

The rest of this paper is organized as follows.
%
In Section~\ref{sec:index-calculus-ecdlp}, we will give an high-level
overview of the index calculus algorithm for attacking ECDLP.
%
In Section~\ref{sec:montgomery-symmetry}, we will describe Montgomery
curves and how we exploit the symmetry for speeding up index calculus
on them.
%
We will then present our experimental results and conclude this paper
in Section~\ref{sec:experimental-results}.

\section{Index calculus for ECDLP}
%
\label{sec:index-calculus-ecdlp}
%
Let $E$ be an elliptic curve defined over a finite field \F{p^n}.
%
For cryptographic applications, we are mostly interested in a
prime-order subgroup generated by a rational point $P\in E(\F{p^n})$.
%
To find an integer $\alpha$ such that $Q=\alpha P$ for
$Q\in\langle P\rangle$ using an index calculus algorithm, one
typically works as follows.
%
\begin{enumerate}
%
\item Determine a \emph{factor base} $\mathcal F\subset E(\F{p^n})$.
%
\item Collect a set $\mathcal R$ of \emph{relations} by decomposing
  random points $a_iP+b_iQ$ into a sum of points from $\mathcal F$,
  i.e.,
  \[ \mathcal
    R=\left\{a_iP+b_iQ=\sum_{j=1}^mP_{i,j}:P_{i,j}\in\mathcal
      F\right\} \]
%
\item When $|\mathcal R|\approx|\mathcal F|$, eliminate the righthand
  side using linear algebra to obtain an equation in the form
  $aP+bQ=\mathcal O$, and $\alpha=-a/b\bmod\ord(P)$.
%
\end{enumerate} 

\subsection{Semaev's summation polynomials}
%
\label{sec:summation-polynomials}
%
As we can see, an important step in index calculus algorithms for
solving ECDLP is point decomposition on an elliptic curve.
%
It is straightforward that if two points sum to zero, then their
$x$-coordinates must be equal.
%
Let us now consider the simplest nontrivial case where three points
sum to the point at infinity.
%
Let $Z=\{(x_1,y_1),(x_2,y_2),(x_3,y_3)\in E(\F{p^n})$ and
$(x_1,y_1)+(x_2,y_2)+(x_3,y_3)=\mathcal O:(x_i,y_i)\in E(\F{p^n})\}$.
%
Clearly, $Z$ is in the variety of the ideal
$I\subset\F{p^n}[X_1,Y_1,X_2,Y_2,X_3,Y_3]$ generated by
\[ \left\{\begin{aligned}
      &  (X_3 - X_1)(Y_2 - Y_1) - (X_2 - X_1)(Y_3 - Y_1),\\
      & Y_i^2 - (X_i^3 + aX_i + b),i=1,2,3
    \end{aligned}\right\}. \]
%
Now let $J=I\cap\F{p^n}[X_1,X_2,X_3]$.
%
Using MAGMA's \texttt{EliminationIdeal} function, we obtain that $J$
is actually a principal ideal generated by the polynomial
$(X_2 - X_3)(X_1 - X_3)(X_1 - X_2)f_3$, where \[ \begin{aligned}
    f_3 = & X_1^2X_2^2 - 2X_1^2X_2X_3 + X_1^2X_3^2 - 2X_1X_2^2X_3 - 2X_1X_2X_3^2 - 2aX_1X_2 - 2aX_1X_3 \\
    & - 4bX_1 + X_2^2X_3^2 - 2aX_2X_3 - 4bX_2 - 4bX_3 + a^2.
  \end{aligned} \]
%
Clearly, the linear factors of the generator correspond to the
degenerated case where two or more points are the same, and $f_3$ is
the 3rd \emph{summation polynomial}, that is, the summation polynomial
for three distinct points summing to zero.

Starting from the 2nd and 3rd summation polynomials, one can
recursively obtain the subsequent summation polynomials via taking
resultants.
%
This is the observation Semaev made in his seminal
work~\cite{DBLP:journals/iacr/Semaev04}.
%
In short, his proposal is to consider factor bases of the following
form: $\{(x,y)\in E(\F{p^n}):x\in V\subset\F{p^n})\}$, where $V$ is a
subset of \F{p^n}.

\subsection{Weil descent}
%
Restricting $x$-coordinates of the points in factor base to a subset
of \F{p^n} is important from a viewpoint of polynomial system solving.
%
Take $f_3$ as an example.
%
When decomposing a random point $aP+bQ$, we first substitute its
$x$-coordinate into say $X_3$, projecting $J$ to $\F{p^n}[X_1,X_2]$.
%
The dimension of the variety of this ideal is nonzero.
%
Therefore, we would like to pose some restrictions on $X_1$ and $X_2$
to reduce the dimension to zero and make the solving time more
manageable.

When looking for solutions to a polynomial in $\F{p^n}[X]$ in \F{p^n},
we can view it as a commutative affine algebra
$\mathcal A=\F{p^n}/(X^{p^n} - X)\cong\F{p^n}[X_1,\ldots,X_n]/(X_1^p -
X_1,\ldots,X_n^p - X_n)$.
%
This can be done by identifying the indeterminate $X$ as
$X_1\theta_1+\cdots+X_n\theta_n$, where $(\theta_1,\ldots,\theta_n)$
is a basis for \F{p^n} over \F p.
%
Hence, a polynomial $f=\sum a_iX^i\in\F{p^n}[X]$ can be identified as
a polynomial $f_1\theta_1+\cdots+f_n\theta_n$, where
$f_1,\ldots,f_n\in\mathcal A'=\F p[X_1,\ldots,X_n]/(X_1^p -
X_1,\ldots,X_n^p - X_n)$, by appropriately sending any coefficient
$a\in\F{p^n}$ to $a_1\theta_1+\cdots+a_n\theta_n$ for
$a_1,\ldots,a_n\in\F p$.
%
Therefore, an equation $f=0$ over \F{p^n} will give rise to a system
of equations $f_1=\cdots=f_n=0$ over \F p.
%
This technique is known as the \emph{Weil descent} and is used in the
Gaudry-Diem attack, in which the factor base is chosen to consist of
points whose $x$-coordinates lie in a subspace $V$ of \F{p^n} over \F
p~\cite{DBLP:journals/jsc/Gaudry09,DBLP:journals/moc/Diem11}.

\subsection{Exploiting symmetry}
%
\label{sec:exploit-symmetry}
%
Naturally, the symmetric group $S_m$ acts on a point decomposition
$P_1+\ldots+P_m$ because elliptic curve groups are abelian.
%
As noted by Gaudry in his seminal work, we can therefore rewrite the
variables $x_1,\ldots,x_m\in\F{p^n}$ by elementary symmetric
polynomials $e_1,\ldots,e_m$, where $e_1=\sum x_i$,
$e_2=\sum_{i\neq j}x_ix_j$,
$e_3=\sum_{i\neq j,i\neq k,j\neq k}x_ix_jx_k$,
etc~\cite{DBLP:journals/jsc/Gaudry09}.
%
Such rewriting can reduce the degree of summation polynomials, as well
as significantly speeding up the solving
time~\cite{DBLP:conf/eurocrypt/FaugerePPR12,DBLP:conf/iwsec/HuangPST13}.

One might be able to exploit further symmetry using actions by other
groups.
%
For example, further speed-up has been reported for point
decomposition by making the factor base invariant under addition of
some small torsion
points~\cite{DBLP:conf/eurocrypt/FaugereHJRV14,DBLP:conf/indocrypt/GalbraithG14}.
%
Such kind of speed-up is, of course, curve-specific in nature, and we
will come back to it in more detail how we can exploit symmetry for
Montgomery curves later in Section~\ref{sec:montgomery-symmetry}.

%------------------------------
\section{Montgomery curves}
%------------------------------------
\label{sec:montgomery-symmetry}
%
A Montgomery curve $M_{A,B}$ over \F{p^n} for $p\neq 2$ is defined by
the equation \begin{equation}
  By^2=x^3+Ax^2+x \label{eq:montgomery-curve} \end{equation} for
$A,B\in\F{p^n}$ such that $A\neq\pm 2$, $B\neq 0$, and
$B(A^2-4)\neq 0$~\cite{1987-montgomery}.
%
For $P=(x,y)\in M_{A,B}$, $-P=(x,-y)$.
%
Furthermore, the addition and doubling formulae for
$(x_3,y_3)=(x_1,y_1)+(x_2,y_2)$ are given as follows.
%
\begin{itemize}
\item When $x_1\neq x_2$:
  \begin{align*}
    x_3 & = B\left(\frac{y_2 - y_1} {x_2 - x_1}\right)^2 - A - x_1 - x_2 = \frac{B(x_2y_1 - x_1y_2)^2} {x_1x_2(x_2 - x_1)^2} \\
    y_3 & = \frac{(2x_1 + x_2 + A)(y_2 - y_1)} {x_2 - x_1} - \frac{B(y_2 - y_1)^3} {(x_2 - x_1)^3} - y_1
  \end{align*}
\item When $x_1=x_2$:
  \begin{align*}
    x_3 & = \frac{(x_1^2 - 1)^2} {4x_1(x_1^2 + Ax_1 + 1)}  \\
    y_3 & = \frac{(2x_1 + x_1 + A)(3x_1^2 + 2Ax_1 + 1)} {2By_1} - \frac{B(3x_1^2 + 2Ax_1 + 1)^3} {(2By_1)^3} - y_1
  \end{align*}
\end{itemize}
%
It was noted by Montgomery himself in his original paper that such
curves can give rise to efficient scalar multiplication
algorithms~\cite{1987-montgomery}.
%
Consider a random point $P\in M_{A,B}(\F{p^n})$ and let
$nP=(X_n:Y_n:Z_n)$ in projective coordinate for any integer $n$.
%
Then:
%
\begin{align*}
  X_{m+n} & = Z_{m-n}[(X_m - Z_m)(X_n + Z_n) + (X_m + Z_m)(X_n - Z_n)]^2 \\
  Z_{m+n} & = X_{m-n}[(X_m - Z_m)(X_n + Z_n) - (X_m + Z_m)(X_n - Z_n)]^2
\end{align*}
%
In particular, when $m=n$:
\begin{align*}
  X_{2n} & = (X_n + Z_n)^2(X_n - Z_n)^2 \\
  Z_{2n} & = (4X_nZ_n)\left((X_n - Z_n)^2 + ((A+2)/4)(4X_nZ_n)\right) \\
  4X_nZ_n & = ( X_n + Z_n)^2 - (X_n - Z_n)^2
\end{align*}
%
In this way, scalar multiplication on Montgomery curve can be
performed without using $y$-coordinates, leading to fast
implementation.

\subsection{Summation polynomials for Montgomery curves}

Following Semaev's approach~\cite{DBLP:journals/iacr/Semaev04}, we can
construct summation polynomials for Montgomery curves as follows.
%
Like Weierstrass curves, the 2nd summation polynomial for Montgomery
curves is simply $f_{M,2} = X_1 - X_2$.
%
Now consider $P,Q\in M_{A, B}$ where $P=(x_1, y_1)$, $Q=(x_2, y_2)$.
%
Let $P+Q=(x_3, y_3)$ and $P-Q=(x_4, y_4)$.
%
By addition formula, we have
\[ x_3 = \frac{B(x_2y_1 - x_1y_2)^2} {x_1x_2(x_2 - x_1)^2},
  x_4 =\frac{B(x_2y_1 - x_1y_2)^2} {x_1x_2(x_2 + x_1)^2}. \]
%
It follows that
%
\begin{align*}
  x_3 + x_4&=\frac{2\left((x_1 + x_2)(x_1x_2 + 1) + 2Ax_1x_2\right)}{(x_1 - x_2)^2},\text{ and} \\
  x_3x_4&=\frac{(1 - x_1x_2)^2}{(x_1 - x_2)^2}.
\end{align*}
%
Using the relationship between the roots of a quadratic polynomial and
its coefficients, we obtain
\[ (x_1 - x_2)^2x^2 - 2\left((x_1 + x_2)(x_1x_2 + 1) +
    2Ax_1x_2\right)x + (1 - x_1x_2)^2. \]
%
From here, we can obtain for Montgomery curve the 3rd summation
polynomial:
\[ f_{M,3}(X_1,X_2,X_3) = (X_1 - X_2)^2X_3^2 - 2\left((X_1 +
    X_2)(X_1X_2 + 1) + 2AX_1X_2\right)X_3 + (1-X_1X_2)^2, \]
%
as well as the subsequent summation polynomials via taking resultants:
\[ f_{M,m}(X_1,\ldots,X_m) =
  \res_X\left(f_{m-k}(X_1,\ldots,X_{m-k-1},X),f_{k+2}(X_{m-k},\ldots,X_m,X)\right). \]
%--------------------------------
\subsection{Exploiting symmetry for Montgomery curves} \label{subsec:TSPL}
%------------------------
A Montgomery curve always contains an affine 2-torsion point $T_2$.
%
Since $T_2+T_2=2T_2=\mathcal O$, it follows that $-T_2=T_2$.
%
If we write $T_2=(x,y)$, then we can see that $y=0$ in order for
$-T_2=T_2$, as $p\neq 2$.
%
Substituting $y=0$ into Equation~(\ref{eq:montgomery-curve}),
we get an equation $x^3+Ax^2+x=0$.
%
The lefthand side factors into $x(x^2+Ax+1)=0$, so we get \[
  x=0,\frac{-A\pm\sqrt{A^2 - 4}}{2}. \]
%
Therefore, the set of rational points on a Montgomery curve includes
at least two 2-torsion points, namely, $\mathcal O$ and $(0,0)$.
%
The other 2-torsion points may or may not be rational, so we will
first focus on the 2-torsion point $(0,0)$.
%
Substituting $(x_2,y_2)=(0,0)$ into the addition formula for
Montgomery curves, we get that for any point $P=(x,y)\in M_{A,B}$,
$P+(0,0)=(1/x,y')$ for some $y'$.

Following the approach outlined by Galbraith and
Gebregiyorgis~\cite{DBLP:conf/indocrypt/GalbraithG14}, we consider a
relation $R:=P_1+\cdots+P_m$ under the action of addition of 2-torsion
points $T_2=(0,0)$:
\[ R' :=
  (P_1+u_1T_2)+(P_2+u_2T_2)+\cdots+(P_{m-1}+u_{m-1}T_2)+\left(P_m+\left(\sum_{i=1}^{m-1}u_i\right)T_2\right). \]
%
Clearly if $R$ holds, then $R'$ also holds for any
$u_1,\ldots,u_{m-1}\in\{0,1\}$, and
$D_m=(\mathbb{Z}/2\mathbb{Z})^{m-1}\rtimes S_m$ acts on the summation
polynomial $f_m$.
%
However, this assumes that the factor base
$\mathcal F=\{(x,y)\in E(\F{p^n}):x\in V\subset\F{p^n}\}$ is invariant
under addition of the 2-torsion point, which is indeed the case for
binary Edwards curves but not necessarily true for Montgomery curves
unless $V$ is closed under taking multiplicative inverses.
%
In other words, this means that for Montgomery curves, $V$ needs to be
a \emph{subfield} of $\F{p^n}$, i.e., $V=\F{p^\ell}$ for some integer
$\ell$ that divides $n$.
%
In this case, $f_m$ is invariant under the action of
$x_i\mapsto 1/x_i$, and we should be able to rewrite $f_m$ in the
variable $x+1/x$.
%
Unlike the case for binary Edwards
curves~\cite{DBLP:conf/indocrypt/GalbraithG14}, the symmetry brought
by addition of 2-torsion points can speed up point decomposition on
Montgomery curves, as we will clearly see from the experimental
results presented in Section~\ref{sec:experimental-results}.

Finally, we can combine such variable rewriting with that for the
symmetric group action $S_m$ using the following new variables:
%
\begin{align*}
  s_1 & = (x_1+\frac{1}{x_1})+\cdots+(x_m+\frac{1}{x_m}) \\
  s_2 & = (x_1+\frac{1}{x_1})(x_2+\frac{1}{x_2})+\cdots+(x_m+\frac{1}{x_m})(x_1+\frac{1}{x_1}) \\
      &\vdots \\
  s_m & = (x_1+\frac{1}{x_1})(x_2+\frac{1}{x_2})\cdots(x_m+\frac{1}{x_m}).
\end{align*}
%
Then we can follow a similar idea outlined in
Section~\ref{sec:summation-polynomials} to combine the recursive
resultant computation in deriving the $m$-th summation polynomial with
that for variable rewriting as follows.
%
Let
$I\subset\F{p^n}[X_1,\ldots,X_m,U_1,\ldots,U_{m-2},S_1,\ldots,S_m]$ be
the ideal generated by the following polynomials:
%
\[ \left\{\begin{aligned}
      & f_{M,3}(X_1,X_2,U_1), \\
      & f_{M,3}(U_1,X_3,U_2), \\
      & \vdots \\
      & f_{M,3}(U_{m-3},X_{m-1},U_{m-2}), \\
      & f_{M,3}(U_{m-2},X_m,x), \\
      & S\left((X_1+\frac{1}{X_1})+\cdots+(X_m+\frac{1}{X_m}) -
        S_1\right), \\
      &
      S\left((X_1+\frac{1}{X_1})(X_2+\frac{1}{X_2})+\cdots+(X_m+\frac{1}{X_m})(X_1+\frac{1}{X_1})
        - S_2\right), \\
      & \vdots \\
      &
      S\left((X_1+\frac{1}{X_1})(X_2+\frac{1}{X_2})\cdots(X_m+\frac{1}{X_m})
        - S_m\right)
\end{aligned}\right\}, \]
where $x\in\F{p^n}$ is the $x$-coordinate of the random point to be
decomposed, and $S=X_1X_2\cdots X_m$ for clearing the denominators.
%
Again we will compute $J=I\cap\F{p^n}[S_1,\ldots,S_m]$ using MAGMA's
\texttt{EliminationIdeal} function to arrive at the $m$-th summation
polynomial in the rewriting variables $S_1,\ldots,S_m$.

\section{Experimental results and concluding remarks}\label{sec:experimental-results}
%
\subsection{Security consideration of ECDLP over OEF}
%
An OEF $\Fqn$ is typically used to achieve highly efficient software
implementation.
%
From the security point of view, $n$ is recommended to be a prime
since $E(\Fqm) \subset E(\Fqn)$ yields $\#E(\Fqm) \mid \# E(\Fqn)$ for
$m \mid n$.
%
However, we have not seen relation between parameters $(p, n)$ of OEF,
and the security of ECDLP over an OEF $\Fqn$.
%
For example, suppose that we use OEFs of $\F{{p_{32}}^5}$,
$\F{{p_{16}}^{11}}$, and $\F{{p_{8}}^{19}}$, where $p_{32}, p_{16}$,
and $p_{8}$ are primes close to $2^8, 2^{16}$, and $2^{32}$,
respectively.
%
These OEFs satisfy a condition of prime extension degree of 5, 11, and
19, respectively, and have almost the same number of rational points,
$\#E(\F{{p_{32}}^5}) \sim \#E(\F{{p_{16}}^{11}}) \sim
\#E(\F{{p_{8}}^{19}})$.
%
Now we pose a question: Using index calculus algorithm described in
Section~\ref{subsec:TSPL}, is there any significant difference in the
computational complexity of solving ECDLP for $\F{{p_{32}}^5}$,
$\F{{p_{16}}^{11}}$, and $\F{{p_{8}}^{19}}$?
%
We will provide some empirical evidence showing an affirmative answer
in the next section.

\subsection{Experiment 1: Exploiting the symmetry}
%
We conducted two experiments using (single-threaded) MAGMA on Intel
Xeon CPU E7-4830 v4 running at 2~GHz.
%
First, we show that using symmetry on Montgomery curves can
significantly speed up relation collection in index calculus.
%
Specifically, we choose $\F{2^{16} - 165}$, over which we generate
several random Weierstrass and Montgomery curves.
%
We then try to decompose several random points into $m$ points from a
factor base consisting of points whose $x$-coordinates are in \F p.
%
The results of the first experiment are summarized in
Table~\ref{tab:exp-sym}.
%
\begin{table}
  \begin{center}
    \begin{tabular}{|r||r|r|r|r|r|r|}
      \hline
      & \multicolumn{2}{c|}{Weierstrass} & \multicolumn{2}{c|}{Montgomery (plain)} & \multicolumn{2}{c|}{Montgomery (sym.)} \\ \hline\hline
      m & Dreg     & Matcost      & Dreg     & Matcost      & Dreg     & Matcost      \\ \hline
      2 & 2        & 2406.6      & 3        & 3206.7      & 2        & 2406.6      \\ \hline
      3 & 6        & 3466021.3    & 6        & 4344579.2    & 6        & 3465410.9    \\ \hline
    \end{tabular}
  \end{center}
  \caption{Experimental results for symmetry exploiting on Montgomery curves.}
  \label{tab:exp-sym}
\end{table}
%
In Table~\ref{tab:exp-sym}, we report the degree of regularity (Dreg)
and the matrix cost in MAGMA's F4 algorithm (Matcost).
%
As we can see, without using the symmetry, it generally takes longer
time to decompose over Montgomery curves, whereas it roughly takes the
same amount of time for Weierstrass curves and Montgomery curves after
taking symmetry into account.

\subsection{Experiment 2: Security evaluation for various OEFs}
%
In this experiment, we would like to evaluate the security of
Montgomery curves over various OEFs; the results are summarized in
Table~\ref{tab:experimental-results}.
%
All timing results are reported in seconds unless otherwise specified.
%
We choose several OEFs suggested by Bailey and
Paar~\cite{DBLP:conf/crypto/BaileyP98}, over which we study the
complexity of the relation collection phase of the index calculus
algorithm for solving ECDLP on several random Montgomery curves.
%
Specifically, for each curve $E$ we first choose a random rational
point $P$ whose order is prime and close to the size of the underlying
OEF \F{p^n}.
%
We then decide a factor base $\mathcal F$ using a vector subspace $V$
of \F{p^n} over \F p, i.e., $|\mathcal F|\approx p^{\dim V}$, which,
together with the linear algebra constant, determines the linear
algebra time (``T\textsubscript L'' in
Table~\ref{tab:experimental-results}).
%
The bulk of our experimentation then involves generating random points
in $\langle P\rangle$ \emph{known} to be decomposable into a sum of
$m$ points from $\mathcal F$.
%
As noted by Galbraith and Gebregiyorgis, the time for solving such
instances tends to be larger than that for attempting to decompose
points that \emph{cannot} be decomposed
so~\cite{DBLP:conf/indocrypt/GalbraithG14}.
%
In general, the success probability of such point decomposition is
roughly $\mathcal O(p^{m\dim V - n}/m!)$ for $m\dim V<n$, which is
taken into account in estimating the relation collection time
(``T\textsubscript R'' in Table~\ref{tab:experimental-results}).
%
\begin{table}
  \begin{center}
    \begin{tabular}{llcr}
      OEF & $\ord(P)$ & $\dim V$ & T\textsubscript R$+$T\textsubscript L \\ \hline
      \F{241^{19}} & $\mathcal O(2^{147})$ & 5 & $212.4\times\mathcal O(2^{72})+\mathcal O(2^{80})$ \\
      \F{65371^{11}} & $\mathcal O(2^{173})$ & 4 & $37.6\times\mathcal O(2^{49})+\mathcal O(2^{129})$ \\
      \F{4294967291^5} & $\mathcal O(2^{154})$ & 2 & $0.04\times\mathcal O(2^{65})+\mathcal O(2^{129})$
    \end{tabular}
  \end{center}
  \caption{Security evaluation of Montgomery curves of various OEFs for $m=2$.}
  \label{tab:experimental-results}
\end{table}
%
It is interesting to note that the complexity for the relation
collection phase is quite different for different OEFs, even though we
use the same Weil descent algorithm.
%
However, as we can see from Table~\ref{tab:experimental-results}, such
index calculus algorithm cannot yet outperform the generic algorithms
such as Pollard's rho method, as the linear algebra phase takes
significantly longer time.

% \section{Concluding remarks}
% %
% \label{sec:conclusions}
%

%
% ---- Bibliography ----
%
\bibliographystyle{abbrv}
\bibliography{dblp,local}

\end{document}
